\documentclass{IFES-beamer}
\usepackage{pgfplots}
\pgfplotsset{compat=1.15}
\usepackage{amssymb}
\usepackage{graphicx,xcolor}
\usepackage{color}
\definecolor{ccqqqq}{rgb}{1,0,0}
\definecolor{darkGreen}{rgb}{0,.5,0}

\usepackage[brazilian]{babel}
\usepackage{amsthm}
\newtheorem*{teorema}{Teorema}


%%%%%%%%%%%%%%%%%%%%%%%%%%%%%%%%%%%%
%  Defini o estlo no algorithmic
%%%%%%%%%%%%%%%%%%%%%%%%%%%%%%%%%%%%%
\newcommand{\R}{\mathbb{R}}
\newcommand{\red}{\color{red}}
\newcommand{\green}{\color{green}}
\newcommand{\TODO}[1]{{{\red #1}}}
%\newcommand{\var}[1]{$#1$} %define estilo de nomes de variaveis
\newcommand{\defi}[1]{\textbf{#1}} % defini estilo ao definir algo no texto
\def\Nil{\text{NIL}}

\newcommand\circledmark{%
  \ooalign{%
    \hidewidth
    \kern-0.4ex\raisebox{-2.1ex}{\scalebox{5.5}{\textcolor{darkGreen}{\textbullet}}}
    \hidewidth\cr
    \kern-.6ex\raisebox{.6ex}{\color{white}$\checkmark$}\cr
  }%
}
\newcommand\custommark{%
  \ooalign{%
    \hidewidth
    \kern-0.4ex\raisebox{-2.1ex}{\scalebox{5.5}{\textcolor{white}{\textbullet}}}
    \hidewidth\cr
    \kern-.6ex{\color{black}$\checkmark$}\cr
  }%
}


\usepackage[Algoritmo]{algorithm}
\usepackage[noend]{algpseudocode}
\algrenewcommand\algorithmicif{\textbf{se}}
\algrenewcommand\algorithmicfor{\textbf{para}}
\algrenewcommand\algorithmicelse{\textbf{senão}}
\algrenewcommand\algorithmicwhile{\textbf{enquanto}}
\algrenewcommand\algorithmicdo{\textbf{faça}}
\algrenewcommand\algorithmicend{\textbf{fim}}
\algrenewcommand\algorithmicthen{\textbf{então}}
\algrenewcommand\algorithmicreturn{\textbf{retorne}}



\newcommand{\AlgoName}[1]{\text{\scshape #1}}

%%%%%%%%%%%%%%%%%%%%%%%%%%%%%%%%%%%%%
% Comandos de notação assintotica   %
%%%%%%%%%%%%%%%%%%%%%%%%%%%%%%%%%%%%%
\renewcommand{\O}[1]{\text{O}(#1)}
\newcommand{\OTheta}[1]{\Theta(#1)}

%%%%%%%%%%%%%%%%%%%%%%%%%%%%%%
%    Nomes de váriaveis      %
%%%%%%%%%%%%%%%%%%%%%%%%%%%%%%
\newcommand{\varname}[1]{\textit{#1}}
\newcommand{\altvarname}[1]{$#1$}
\newcommand{\node}{\textit{nó}}
\newcommand{\var}{\mathit{var}}

%%%%%%%%%%%%%%%%%%%%%%%%%%%%%%
%     Métodos de Grafos      %
%%%%%%%%%%%%%%%%%%%%%%%%%%%%%%
\newcommand{\graphCreate}{\AlgoName{novoGrafo}}
\newcommand{\graphAdd}{\AlgoName{ligueGLA}}
\newcommand{\graphDel}{\AlgoName{removaGLA}}

%%%%%%%%%%%%%%%%%%%%%%%%%%%%%%
%     Métodos de Treap       %
%%%%%%%%%%%%%%%%%%%%%%%%%%%%%%
\newcommand{\treapCreate}{\AlgoName{novoNó}}
\newcommand{\treapSearch}{\AlgoName{busca}}
\newcommand{\treapGetLast}{\AlgoName{último}}
\newcommand{\treapGetRoot}{\AlgoName{raiz}}
\newcommand{\treapOrder}{\AlgoName{ordem}}
\newcommand{\treapJoin}{\AlgoName{junta}}
\newcommand{\treapSplit}{\AlgoName{corta}}
\newcommand{\treapSplitRight}{\AlgoName{cortaDireita}}
\newcommand{\treapGetSize}{\AlgoName{tamanho}}
\newcommand{\treapGetEdgesLevel}{\AlgoName{arestasDeNível}}

\newcommand{\treapFirst}{\AlgoName{primeiro}}
\newcommand{\treapLast}{\AlgoName{último}}
\newcommand{\treapPredecessor}{\AlgoName{\AlgoName{Pred}}}    %(F, u, v)

%%%%%%%%%%%%%%%%%%%%%%%%%%%%%%
% Métodos de Euler Tour Tree %
%%%%%%%%%%%%%%%%%%%%%%%%%%%%%%
\newcommand{\ETTCreate}{\AlgoName{novoETT}}     % (v)
\newcommand{\ETTAddEdge}{\AlgoName{ligueETT}} % ($F$, $uu$, $vv$)
\newcommand{\ETTDelEdge}{\AlgoName{removaETT}} % ($F$, $uu$, $vv$)
\newcommand{\ETTQuery}{\AlgoName{conectadoETT}} % ($F$, $uu$, $vv$)
\newcommand{\ETmovetofront}{\AlgoName{movaInício}} % ($F$, $uu$)


%%%%%%%%%%%%%%%%%%%%%%%%%%%%%%%%%%
% Métodos da tabela hash         %
%%%%%%%%%%%%%%%%%%%%%%%%%%%%%%%%%%
\newcommand{\dymForestHash}{$H$}  %simbolo que identifica a matriz/hash da floresta
\newcommand{\nivel}{\AlgoName{nível}} 
\newcommand{\hashCreate}{\AlgoName{novoDicio}}     

%%%%%%%%%%%%%%%%%%%%%%%%%%%%%%%%%%
% Métodos de Florestas dinamicas %
%%%%%%%%%%%%%%%%%%%%%%%%%%%%%%%%%%
\newcommand{\dymForestCreate}{\AlgoName{novaFD}}  %(n)
\newcommand{\dymForestAddEdge}{\AlgoName{ligueFD}}    %(F, u, v)
\newcommand{\dymForestDelEdge}{\AlgoName{removaFD}}    %(F, u, v)
\newcommand{\dymForestQuery}{\AlgoName{\AlgoName{conectadoFD}}}    %(F, u, v)

%%%%%%%%%%%%%%%%%%%%%%%%%%%%%%%%%%
% Métodos de Grafos dinamicas %
%%%%%%%%%%%%%%%%%%%%%%%%%%%%%%%%%%
\newcommand{\dymGraphCreate}{\AlgoName{novoGD}}    %(n)
\newcommand{\dymGraphAddEdge}{\AlgoName{ligueGD}} %(G, u, v)
\newcommand{\dymGraphDelEdge}{\AlgoName{removaGD}} %(G, u, v)
\newcommand{\dymGraphQuery}{\AlgoName{\AlgoName{conectadoGD}}} %(G, u, v)
\newcommand{\dymGraphReplace}{\AlgoName{\AlgoName{substituaGD}}} %(G, u, v, i)
\newcommand{\dymGraphHash}{$H$}  %simbolo que identifica a matriz/hash da floresta

%%%%%%%%%%%%%%%%%%%%%%%%%%%%%%
% Métodos de Link/Cut Tree   %
%%%%%%%%%%%%%%%%%%%%%%%%%%%%%%

\newcommand{\linkcutCreate}{\AlgoName{newLCT}}
\newcommand{\linkcutDestroy}{\AlgoName{delLCT}}
\newcommand{\linkcutAddEdge}{\AlgoName{link}}    %(F, u, v)
\newcommand{\linkcutDelEdge}{\AlgoName{cut}}    %(F, u, v)
\newcommand{\linkcutEvert}{\AlgoName{\AlgoName{evert}}}    %(v)
\newcommand{\linkcutMax}{\AlgoName{\AlgoName{max}}}    %(F, u, v)
\newcommand{\linkcutMin}{\AlgoName{\AlgoName{min}}}    %(F, u, v)
\newcommand{\linkcutParent}{\AlgoName{\AlgoName{parent}}}    %(F, u, v)
\newcommand{\linkcutQuery}{\AlgoName{\AlgoName{conectadoLC}}}    %(F, u, v)
\newcommand{\linkcutWeight}{\AlgoName{\AlgoName{set weight}}}    %(F, u, v)
\newcommand{\linkcutRoot}{\AlgoName{\AlgoName{get Root}}}    %(F, u, v)

%%%%%%%%%%%%%%%%%%%%%%%%%%%%%%%%%%%%%%%
% Métodos de Link/Cut Tree com ordem  %
%%%%%%%%%%%%%%%%%%%%%%%%%%%%%%%%%%%%%%%

\newcommand{\LCOMakeOcto}{\AlgoName{Create Octo}}
\newcommand{\LCODestroyOcto}{\AlgoName{Destroy Octo}}

\newcommand{\LCOMakeNode}{\AlgoName{Make edge}}
\newcommand{\LCODestroyNode}{\AlgoName{Make edge}}
\newcommand{\LCOLink}{\AlgoName{Link}}
\newcommand{\LCOMerge}{\AlgoName{Merge}}
\newcommand{\LCOSplit}{\AlgoName{Split}}
\newcommand{\LCOCycle}{\AlgoName{Cycle}}
\newcommand{\LCOParent}{\AlgoName{\AlgoName{Parent}}}    %(F, u, v)
\newcommand{\LCORoot}{\AlgoName{Root}}
\newcommand{\LCOAddCost}{\AlgoName{Set weight}}
\newcommand{\LCOMax}{\AlgoName{\AlgoName{Find max}}}    %(F, u, v)
\newcommand{\LCOMin}{\AlgoName{\AlgoName{Find min}}}    %(F, u, v)
\newcommand{\LCOEvert}{\AlgoName{\AlgoName{Evert}}}    %(F, u, v)
\newcommand{\LCOConnected}{\AlgoName{\AlgoName{Connected}}}    %(F, u, v)
\newcommand{\LCOFindNode}{\AlgoName{Find node}}

%%%%%%%%%%%%%%%%%%%%%%%%%%%%%%
%     Métodos de MSF         %
%%%%%%%%%%%%%%%%%%%%%%%%%%%%%%
\newcommand{\MSFCreate}{\AlgoName{novoGDP}} %(n)
\newcommand{\MSFupdate}{\AlgoName{mudaPesoGDP}} %(n)
\newcommand{\MSFaddEdge}{\AlgoName{ligueGDP}}    %(G, u, v, w)
\newcommand{\MSFdelEdge}{\AlgoName{removaGDP}}    %(G, u, v)
\newcommand{\MSFweight}{\AlgoName{pesoGDP}}    %(G)

\newcommand{\dymGraphReplaceMSF}{\AlgoName{substituaGDP}} %(G, u, v, i)
\newcommand{\treapGetEdgeMinWeight}{\AlgoName{arestaMinPesoGDP}}


%%%%%%%%%%%%%%%%%%%%%%%%%%%%%%
%     Métodos de VPSP
% Verify partial sum of permutations%
%%%%%%%%%%%%%%%%%%%%%%%%%%%%%%

\newcommand{\VPSPconvert}{\AlgoName{converta}}
\newcommand{\VPSPupdate}{\AlgoName{substitua}}
\newcommand{\VPSPverify}{\AlgoName{verifique}}




% --------------------------------------------------- %
%                  Presentation info	              %
% --------------------------------------------------- %
\title[Algorit. em conexidade dinâmica]{Algoritmos para conexidade em grafos dinâmicos}
%\subtitle{Subtitle}
\author[Arthur Rodrigues]{Arthur Henrique Dias Rodrigues\\{\footnotesize sob orientação de}\\Cristina Gomes Fernandes}
\institute[IME-USP]{
  Instituto de Matemática e Estatística\\
  USP
}
\day=22
\month=11
\year=2024
\subject{Algorit. em conexidade dinâmica} % metadata

% --------------------------------------------------- %
%                    Title + Schedule                 %
% --------------------------------------------------- %


\begin{document}
\begin{frame}
  \titlepage
\end{frame}

\begin{frame}{Sumário}
  \tableofcontents
\end{frame}

\iffalse
% --------------------------------------------------- %
%                      Presentation                   %
% --------------------------------------------------- %

\section{Problemas}
\subsection{Definições}
\begin{frame}{Problema de conexidade dinâmicas}
\begin{block}{Contexto}
\begin{itemize}
    \item $G$: grafo;
    \item $n$: número de vértices em~$G$;
    \item $F$: floresta;
    \item $u$,$v$: vértices.
    \end{itemize}
\end{block}

\begin{exampleblock}{Problema de conexidade dinâmicas}
\begin{itemize}
\item \dymGraphCreate($n$): retorna um grafo dinâmico com $n$ vértices isolados;
\item \dymGraphAddEdge($G$, $u$, $v$): adiciona aresta $uv$ a~$G$;
\item \dymGraphDelEdge($G$, $u$, $v$): remove $uv$ de $G$; e
\item \dymGraphQuery($G$, $u$, $v$): retorna verdadeiro se $u$ e~$v$ estão na mesma componente conexa de~$G$ e falso, caso contrário.
\end{itemize}
\end{exampleblock}

\end{frame}

\begin{frame}{Problemas MSF}
\boxblue{MSF: floresta maximal de peso mínimo}
\begin{block}{Grafo ponderado}
Cada aresta possui um peso associado.
\end{block}

\begin{exampleblock}{Problema de floresta maximal de peso mínimo em grafos ponderados planos dinâmicos}
\begin{itemize}
\item \MSFCreate($n$): devolve um grafo ponderado com $n$ vértices isolados;
\item \MSFaddEdge($G$, $u$, $v$, $w$): adiciona a aresta~$uv$ com peso~$w$ em~$G$;
\item \MSFdelEdge($G$, $u$, $v$): remove a aresta $uv$ de $G$; e
\item \MSFweight($G$): devolve o peso de uma MSF de $G$.
\end{itemize}
\end{exampleblock}
\end{frame}

\subsection{Resultados conhecidos}
\begin{frame}{Resultados conhecidos}
\begin{alertblock}{Limitante inferior (Patrascu e Demaine)~\cite{lowerBoundPatrascu}, 2006}
Todo algoritmo que resolve os problemas de conexidade dinâmica e de MSF dinâmica possuem consumo de tempo~$\Omega(\lg n)$.
\end{alertblock}
\begin{block}{Conexidade dinâmicas~(Holm et al.~\cite{poly_log}, 2001)}
Para árvores,~$\O{\lg n}$;\\
Grafos gerais,~$\O{\lg^2 n}$ amortizado.
\end{block}

\begin{block}{MSF~(Holm et al.~\cite{poly_log}, 2001)}
Para grafos gerais, amortizado~$\O{\lg^4 n}$.
\end{block}
\end{frame}
\fi

% --------------------------------------------------- %
%      Conexidade em florestas dinâmicas              %
% --------------------------------------------------- %
\section{Conexidade em florestas dinâmicas}
\subsection{Definição}
\begin{frame}{Conexidade em florestas dinâmicas}

\begin{exampleblock}{Problema de conexidade em florestas dinâmicas}
\begin{itemize}
\item \dymForestCreate($n$): retorna uma floresta dinâmica com $n$ vértices isolados;
\item \dymForestAddEdge($F$,$u$,$v$): adiciona $uv$ a~$F$;
\item \dymForestDelEdge($F$,$u$,$v$): remove $uv$ de $F$; e
\item \dymForestQuery($F$,$u$,$v$): retorna verdadeiro se $u$ e~$v$ estão na mesma componente conexa de~$F$ e falso, caso contrário.
\end{itemize}
\end{exampleblock}
Para solucionar esse problema, vamos apresentar a estrutura de dados proposta por Henzinger e King.
\end{frame}

\subsection{Euler Tour Trees}

\begin{frame}{Sequência Euleriana}
\begin{figure}[htb]
\centering
\scalebox{.9}{
\begin{tikzpicture}[dot/.style={draw,circle,fill,inner sep=1.5pt},line width=1pt,x=1.5cm,y=1.5cm]
\clip(.5,.8) rectangle (3.3,3.5);
\draw [line width=1pt] (1.7420645075484014,2.9548225063123694)-- (0.6109449986613309,3.1228105521866865);
\draw [line width=1pt] (1.7420645075484014,2.9548225063123694)-- (2,2);
\draw [line width=1pt] (2,2)-- (0.9693194965265414,1.7453085760172875);
\draw [line width=1pt] (2,2)-- (3,1);
\draw [line width=1pt] (2,2)-- (2.5036103155119798,2.742037648204901);
\begin{scriptsize}
\draw [fill=black,bend left] (1.7420645075484014,2.9548225063123694) circle (1.5pt);
\draw[color=black] (1.8316581320147085,3.195605372065557) node {$1$};
\draw [fill=black] (0.6109449986613309,3.1228105521866865) circle (1.5pt);
\draw[color=black] (0.7005386231276353,3.363593417939874) node {$2$};
\draw [fill=black] (2,2) circle (1.5pt);
\draw[color=black] (1.9,1.8) node {$4$};
\draw [fill=black] (0.9693194965265414,1.7453085760172875) circle (1.5pt);
\draw[color=black] (1,2) node {$5$};
\draw [fill=black] (3,1) circle (1.5pt);
\draw[color=black] (3.0859688745429477,1.2357448368651927) node {$3$};
\draw [fill=black] (2.5036103155119798,2.742037648204901) circle (1.5pt);
\draw[color=black] (2.5932039399782822,2.9828205139580892) node {$0$};
\end{scriptsize}
\end{tikzpicture}
%\documentclass[border=5pt,tikz]{standalone}
%\usetikzlibrary{positioning}
%\begin{document}
\begin{tikzpicture}[dot/.style={draw,circle,fill,inner sep=1.5pt},line width=.7pt,x=1.5cm,y=1.5cm]
\clip(-1,.8) rectangle (3.3,3.5);
\begin{scriptsize}
\node[label=above:{\large 0}] (r0) at (2.7,2.4) [dot] {};

\node[label=above:{\large 1}] (r1) at (1.7420645075484014,2.9548225063123694) [dot] {};
%\draw[color=black] (1.8316581320147085,3.195605372065557) node {$1$};


\node[label=above:{\large 2}] (r2) at (0.6109449986613309,3.1228105521866865) [dot] {};
%\draw[color=black] (0.7005386231276353,3.363593417939874) node {$2$};

\node[label=right:{\large 3}] (r3) at (2.6,1) [dot] {};
%\draw[color=black] (2.6,1.2357448368651927) node {$3$};

\node (r4) at (2,2) [dot] {};
\draw[color=black] (1.9,1.8) node {{\large 4}};

\node[label=above:{\large 5}] (r5) at (0.9693194965265414,1.7453085760172875) [dot] {};
%\draw[color=black] (1,2) node {$5$};

\draw[->] (r0) to[bend left] (r4);
\draw[->] (r4) to[bend left] (r0);

\draw[->] (r3) to[bend left] (r4);
\draw[->] (r4) to[bend left] (r3);

\draw[->] (r5) to[bend left] (r4);
\draw[->] (r4) to[bend left] (r5);

\draw[->] (r1) to[bend left] (r4);
\draw[->] (r4) to[bend left] (r1);

\draw[->] (r0) to[bend left] (r4);
\draw[->] (r4) to[bend left] (r0);

\draw[->] (r1) to[bend left] (r2);
\draw[->] (r2) to[bend left] (r1);

\end{scriptsize}
\end{tikzpicture}
%\end{document}

	}
\end{figure}
\begin{center}
{\large 30~00~04~41~12~22~21~11~14~44~45~55~54~40~03~33}
\end{center}
\end{frame}

\begin{frame}{Euler Tour Trees}
\begin{figure}[htb]
\centering
\scalebox{.7}{
\begin{tikzpicture}[line cap=round,line join=round,x=1cm,y=1cm]
\clip(-.6,-1) rectangle (14.1,4.6);
\draw [line width=1pt] (1,1) circle (0.5cm);
\draw [line width=1pt] (2,2) circle (0.5cm);
\draw [line width=1pt] (3,1) circle (0.5cm);
\draw [line width=1pt] (5.5,2) circle (0.5cm);
\draw [line width=1pt] (6.5,1) circle (0.5cm);
\draw [line width=1pt] (3.5,3) circle (0.5cm);
\draw [line width=1pt] (7,4) circle (0.5cm);
\draw [line width=1pt] (10.5,3) circle (0.5cm);
\draw [line width=1pt] (9.5,2) circle (0.5cm);
\draw [line width=1pt] (8.5,1) circle (0.5cm);
\draw [line width=1pt] (11.5,2) circle (0.5cm);
\draw [line width=1pt] (10.5,1) circle (0.5cm);
\draw [line width=1pt] (12.5,1) circle (0.5cm);
\draw [line width=1pt] (4.5,1) circle (0.5cm);
\draw [line width=1pt] (7.5,0) circle (0.5cm);
\draw [line width=1pt] (9.5,0) circle (0.5cm);
\draw [line width=1pt] (1.3535533905932737,1.3535533905932737)-- (1.646446609406726,1.646446609406726);
\draw [line width=1pt] (2.353553390593274,1.646446609406726)-- (2.646446609406726,1.353553390593274);
i%\draw [line width=1pt] (2.646446609406726,0.646446609406726)-- (2.353553390593274,0.35355339059327395);
\draw [line width=1pt] (2.416025147168923,2.2773500981126156)-- (3.0839748528310773,2.722649901887385);
\draw [line width=1pt] (3.947213595499958,2.776393202250021)-- (5.052786404500042,2.223606797749979);
\draw [line width=1pt] (5.146446609406726,1.646446609406726)-- (4.853553390593274,1.353553390593274);
\draw [line width=1pt] (5.853553390593274,1.646446609406726)-- (6.146446609406726,1.353553390593274);
\draw [line width=1pt] (10.01923802617959,3.137360563948689)-- (7.480761973820412,3.8626394360513108);
\draw [line width=1pt] (6.519238026179588,3.8626394360513108)-- (3.9807619738204116,3.137360563948689);
\draw [line width=1pt] (10.146446609406727,2.646446609406727)-- (9.853553390593273,2.353553390593273);
\draw [line width=1pt] (9.146446609406727,1.646446609406727)-- (8.853553390593273,1.353553390593273);
\draw [line width=1pt] (8.146446609406727,0.6464466094067269)-- (7.853553390593274,0.35355339059327395);
\draw [line width=1pt] (8.853553390593273,0.6464466094067269)-- (9.146446609406727,0.35355339059327306);
\draw [line width=1pt] (10.853553390593273,1.353553390593273)-- (11.146446609406727,1.646446609406727);
\draw [line width=1pt] (11.146446609406727,2.353553390593273)-- (10.853553390593273,2.646446609406727);
\draw [line width=1pt] (11.853553390593273,1.646446609406727)-- (12.146446609406727,1.353553390593273);


\draw[color=black] (1,1) node {$30$};
\draw[color=black] (1,.2) node {1};
\draw[color=black] (2,2) node {$00$};
\draw[color=black] (2,1.2) node {2};
\draw[color=black] (3,1) node {$04$};
\draw[color=black] (3,.2) node {3};
\draw[color=black] (3.5,3) node {$41$};
\draw[color=black] (3.5,2.2) node {4};
\draw[color=black] (4.5,1) node {$12$};
\draw[color=black] (4.5,.2) node {5};
\draw[color=black] (5.5,2) node {$22$};
\draw[color=black] (5.5,1.2) node {6};
\draw[color=black] (6.5,1) node {$21$};
\draw[color=black] (6.5,.2) node {7};
\draw[color=black] (7,4) node {$11$};
\draw[color=black] (7,3.2) node {8};
\draw[color=black] (7.5,0) node {$14$};
\draw[color=black] (7.5,-.8) node {9};
\draw[color=black] (8.5,1) node {$44$};
\draw[color=black] (8.5,.2) node {10};
\draw[color=black] (9.5,0) node {$45$};
\draw[color=black] (9.5,-.8) node {11};
\draw[color=black] (9.5,2) node {$55$};
\draw[color=black] (9.5,1.2) node {12};
\draw[color=black] (10.5,3) node {$54$};
\draw[color=black] (10.5,2.2) node {13};
\draw[color=black] (10.5,1) node {$40$};
\draw[color=black] (10.5,.2) node {14};
\draw[color=black] (11.5,2) node {$03$};
\draw[color=black] (11.5,1.2) node {15};
\draw[color=black] (12.5,1) node {$33$};
\draw[color=black] (12.5,.2) node {16};


\end{tikzpicture}
}
\end{figure}
\begin{tabular}{| c p{0.13cm} p{0.13cm} p{0.13cm} p{0.13cm} p{0.13cm} p{0.13cm} p{0.13cm} p{0.13cm} p{0.13cm} p{0.13cm} p{0.13cm} p{0.13cm} p{0.13cm} p{0.13cm} p{0.13cm} p{0.13cm} |} 
 \hline
	arco &30&00&04&41&12&22&21&11&14&44&45&55&54&40&03&33\\
 \hline
	índice &1&2&3&4&5&6&7&8&9&10&11&12&13&14&15&16 \\ 
 \hline
\end{tabular}
\end{frame}



\begin{frame}{Chaves implícitas}
\begin{figure}[htb]
\scalebox{.7}{
\centering
\begin{tikzpicture}[line cap=round,line join=round,>=triangle 45,x=1cm,y=1cm]
\clip(-.6,-1) rectangle (14.1,4.6);
\draw [line width=1pt] (1,1) circle (0.5cm);
\draw [line width=1pt] (2,2) circle (0.5cm);
\draw [line width=1pt] (3,1) circle (0.5cm);
\draw [line width=1pt] (5.5,2) circle (0.5cm);
\draw [line width=1pt] (6.5,1) circle (0.5cm);
\draw [line width=1pt] (3.5,3) circle (0.5cm);
\draw [line width=1pt] (7,4) circle (0.5cm);
\draw [line width=1pt] (10.5,3) circle (0.5cm);
\draw [line width=1pt] (9.5,2) circle (0.5cm);
\draw [line width=1pt] (8.5,1) circle (0.5cm);
\draw [line width=1pt] (11.5,2) circle (0.5cm);
\draw [line width=1pt] (10.5,1) circle (0.5cm);
\draw [line width=1pt] (12.5,1) circle (0.5cm);
\draw [line width=1pt] (4.5,1) circle (0.5cm);
\draw [line width=1pt] (7.5,0) circle (0.5cm);
\draw [line width=1pt] (9.5,0) circle (0.5cm);
\draw [line width=1pt] (1.3535533905932737,1.3535533905932737)-- (1.646446609406726,1.646446609406726);
\draw [line width=1pt] (2.353553390593274,1.646446609406726)-- (2.646446609406726,1.353553390593274);
i%\draw [line width=1pt] (2.646446609406726,0.646446609406726)-- (2.353553390593274,0.35355339059327395);
\draw [line width=1pt] (2.416025147168923,2.2773500981126156)-- (3.0839748528310773,2.722649901887385);
\draw [line width=1pt] (3.947213595499958,2.776393202250021)-- (5.052786404500042,2.223606797749979);
\draw [line width=1pt] (5.146446609406726,1.646446609406726)-- (4.853553390593274,1.353553390593274);
\draw [line width=1pt] (5.853553390593274,1.646446609406726)-- (6.146446609406726,1.353553390593274);
\draw [line width=1pt] (10.01923802617959,3.137360563948689)-- (7.480761973820412,3.8626394360513108);
\draw [line width=1pt] (6.519238026179588,3.8626394360513108)-- (3.9807619738204116,3.137360563948689);
\draw [line width=1pt] (10.146446609406727,2.646446609406727)-- (9.853553390593273,2.353553390593273);
\draw [line width=1pt] (9.146446609406727,1.646446609406727)-- (8.853553390593273,1.353553390593273);
\draw [line width=1pt] (8.146446609406727,0.6464466094067269)-- (7.853553390593274,0.35355339059327395);
\draw [line width=1pt] (8.853553390593273,0.6464466094067269)-- (9.146446609406727,0.35355339059327306);
\draw [line width=1pt] (10.853553390593273,1.353553390593273)-- (11.146446609406727,1.646446609406727);
\draw [line width=1pt] (11.146446609406727,2.353553390593273)-- (10.853553390593273,2.646446609406727);
\draw [line width=1pt] (11.853553390593273,1.646446609406727)-- (12.146446609406727,1.353553390593273);


\draw[color=black] (1,1) node {$30$};
\draw[color=black] (1,.2) node {1};
\draw[color=black] (2,2) node {$00$};
\draw[color=black] (2,1.2) node {3};
\draw[color=black] (3,1) node {$04$};
\draw[color=black] (3,.2) node {1};
\draw[color=black] (3.5,3) node {$41$};
\draw[color=black] (3.5,2.2) node {7};
\draw[color=black] (4.5,1) node {$12$};
\draw[color=black] (4.5,.2) node {1};
\draw[color=black] (5.5,2) node {$22$};
\draw[color=black] (5.5,1.2) node {3};
\draw[color=black] (6.5,1) node {$21$};
\draw[color=black] (6.5,.2) node {1};
\draw[color=black] (7,4) node {$11$};
\draw[color=black] (7,3.2) node {16};
\draw[color=black] (7.5,0) node {$14$};
\draw[color=black] (7.5,-.8) node {1};
\draw[color=black] (8.5,1) node {$44$};
\draw[color=black] (8.5,.2) node {3};
\draw[color=black] (9.5,0) node {$45$};
\draw[color=black] (9.5,-.8) node {1};
\draw[color=black] (9.5,2) node {$55$};
\draw[color=black] (9.5,1.2) node {4};
\draw[color=black] (10.5,3) node {$54$};
\draw[color=black] (10.5,2.2) node {8};
\draw[color=black] (10.5,1) node {$40$};
\draw[color=black] (10.5,.2) node {1};
\draw[color=black] (11.5,2) node {$03$};
\draw[color=black] (11.5,1.2) node {3};
\draw[color=black] (12.5,1) node {$33$};
\draw[color=black] (12.5,.2) node {1};


\end{tikzpicture}
}
\end{figure}
\begin{tabular}{| c p{0.13cm} p{0.13cm} p{0.13cm} p{0.13cm} p{0.13cm} p{0.13cm} p{0.13cm} p{0.13cm} p{0.13cm} p{0.13cm} p{0.13cm} p{0.13cm} p{0.13cm} p{0.13cm} p{0.13cm} p{0.13cm} |} 
 \hline
	arco &30&00&04&41&12&22&21&11&14&44&45&55&54&40&03&33\\
 \hline
	índice &1&2&3&4&5&6&7&8&9&10&11&12&13&14&15&16 \\ 
 \hline
\end{tabular}
\end{frame}


\begin{frame}{Biblioteca de Euler Tour Trees}
\begin{exampleblock}{Biblioteca de Euler Tour Trees}
\begin{itemize}
\item  \treapCreate($u$, $v$): retorna uma ABB com um único nó com valor uv;
\item \treapJoin($T$, $R$): junta as ABBs $T$ e $R$ concatenando as sequências Eulerianas armazenada nelas e retorna a raiz da árvore resultante.
\item \treapSplit($\node$): corta a ABB que contém um nó~$\node$ em três ABBs. A primeira ABB contém todos os nós com chave estritamente menor do que a chave de~$\node$, a segunda contém somente~$\node$ e a última contém todos os nós com chave estritamente maior do que a chave de~$\node$. Essa rotina retorna as raízes dessas três ABBs; e
\item \treapGetRoot($\node$): retorna a raiz da ABB que contém $\node$;
\end{itemize}
\end{exampleblock}
\boxpurple{\centering  \treapCreate{}:~$\O{1}$.\\ As demais operações :~$\O{\lg n}$.}
\end{frame}

\begin{frame}{Tabela de símbolos}
\boxblue{
\centering
Associa $(u,v) \rightarrow uv$.
}
\begin{exampleblock}{Biblioteca de tabela de símbolos}
\begin{itemize}
    \item $F \gets \hashCreate(n)$: cria e retorna um dicionário~$F$ para uma floresta dinâmica com~$n$ vértices;
    \item $F[u,v] \gets uv$: insere o nó que contém $uv$, com chave $(u,v)$ e valor associado~$uv$ na tabela~$F$.
    Se o par~$(u,v)$ já estiver presente no dicionário, então seu valor associado é substituído por~$uv$;
    \item $F[u,v] \gets \Nil{}$: remove o nó associado a~$(u,v)$ e seu valor associado do dicionário~$F$;
    \item $\var{} \gets F[u,v]$: atribui o valor associado à chave~$(u,v)$ à variável~$\var$;
	Caso a chave~$(u,v)$ não esteja presente em~$F$, atribui~$\Nil$ a~$\var{} $.
\end{itemize}
\end{exampleblock}
\boxpurple{
\centering
Consumo esperado~$\O{1}$ por rotina~\cite{CLRS}.
}
\end{frame}

\begin{frame}{Implementação da interface de floresta dinâmica}
\begin{algorithm}[H]
\caption{\dymForestCreate($n$)}
\label{Algo:dymForestCreate}
\begin{algorithmic}[1]
\State $F~\gets~\hashCreate(n)$
\For {$v$ $\gets$ 1 até $n$}\label{Algo:dymForestCreate:for}
\State $F[v,v]~\gets$ \treapCreate($v$, $v$)
\EndFor
\State \Return $F$
\end{algorithmic}
\end{algorithm}
\begin{algorithm}[H]
\caption{\dymForestQuery($F$, $u$, $v$)}
\label{Algo:dymForestQuery}
\begin{algorithmic}[1]
\State \varname{uu} $\gets$ $F[u,u]$
\State $vv$ $\gets$ $F[v,v]$
\State \Return \treapGetRoot(\varname{uu}) = \treapGetRoot($vv$)
\end{algorithmic}
\end{algorithm}
\boxpurple{
\centering
~~~~~\dymForestCreate{} : $\O{n}$\\
\dymForestQuery{}  : $\O{\lg n}$
}
\end{frame}

\begin{frame}{Implementação da interface de floresta dinâmica}
\begin{algorithm}[H]
\caption{\ETmovetofront($F$,$u$)}
\label{Algo:ETmovetofront}
\begin{algorithmic}[1]
\State \varname{uu} $\gets$ $F[u,u]$\label{Algo:ETmovetofront:1}
\State $A$, \varname{uu}, $B$ $\gets$ \treapSplit(\varname{uu})\label{Algo:ETmovetofront:2}
\State \Return \treapJoin(\varname{uu}, $B$, $A$)\label{Algo:ETmovetofront:3}
\end{algorithmic}
\end{algorithm}
\begin{algorithm}[H]
\caption{\dymForestAddEdge($F$,$u$,$v$)}
\label{Algo:dymForestAddEdge}
\begin{algorithmic}[1]
\State $U$ $\gets$ \ETmovetofront($F$, $u$)
\State $V$ $\gets$ \ETmovetofront($F$, $v$)
\State $uv$ $\gets$ \treapCreate($u$, $v$)
\State $vu$ $\gets$ \treapCreate($v$, $u$)
\State $F[u,v]$ $\gets$ $uv$
\State $F[v,u]$ $\gets$ $vu$
\State \treapJoin($U$, $uv$, $V$, $vu$)
\end{algorithmic}
\end{algorithm}
\boxpurple{
\centering
\dymForestAddEdge{}  : $\O{\lg n}$\\
\ETmovetofront{}  : $\O{\lg n}$
}
\end{frame}

\begin{frame}{Implementação da interface de floresta dinâmica}
\begin{algorithm}[H]
\caption{\dymForestDelEdge($F$, $u$, $v$)}
\label{Algo:dymForestDelEdge}
\begin{algorithmic}[1]
\State $uv$ $\gets$ $F[u,v]$\label{Algo:dymForestDelEdge:1}
\State $vu$ $\gets$ $F[v,u]$\label{Algo:dymForestDelEdge:2}
\State $A$, $uv$, $B$ $\gets$ \treapSplit($uv$)\label{Algo:dymForestDelEdge:3}
\State \treapJoin($B$, $A$)\label{Algo:dymForestDelEdge:4}
\State \treapSplit($vu$)\label{Algo:dymForestDelEdge:5}
\State $F[u,v]$ $\gets$ $\Nil{}$\label{Algo:dymForestDelEdge:6}
\State $F[v,u]$ $\gets$ $\Nil{}$\label{Algo:dymForestDelEdge:7}
\end{algorithmic}
\end{algorithm}
\boxpurple{
\centering
\dymForestDelEdge{}  : $\O{\lg n}$
}
\end{frame}


% --------------------------------------------------- %
%      Conexidade em grafos dinâmicos              %
% --------------------------------------------------- %
\section{Conexidade em grafos dinâmicos}
\subsection{Definição}
\begin{frame}{Conexidade em grafos dinâmicos}
\begin{exampleblock}{Conexidade em grafos dinâmicos}
\begin{itemize}
\item \dymGraphCreate($n$): cria um grafo dinâmico com $n$ vértices isolados;
\item \dymGraphAddEdge($G$, $u$, $v$): adiciona a aresta $uv$ ao grafo dinâmico $G$;
\item \dymGraphDelEdge($G$, $u$, $v$): remove a aresta $uv$ de $G$; e
\item \dymGraphQuery($G$, $u$, $v$): retorna verdadeiro se $u$ e $v$ estão na mesma componente conexa de $G$ e falso, caso contrário.
\end{itemize}
\end{exampleblock}
Para solucionar esse problema, vamos apresentar a estrutura de dados proposta por Holm, de Lichtenberg e Thorup.

\end{frame}

\begin{frame}{Ideia inicial}
\begin{block}{Manteremos}
\begin{itemize}
    \item floresta maximal dinâmica~$F$ de~$G$; e
    \item um grafo~$R$ = $G-F$
\end{itemize}
\end{block}

\begin{exampleblock}{Lista de adjacências}
\begin{itemize}
    \item \graphCreate($n$): devolve a representação por listas de adjacências de um grafo com~$n$ vértices isolados.
    \item \graphAdd($G$, $u$, $v$): adiciona $u$ na lista de adjacências de $v$ em $G$ e vice-versa.
    \item \graphDel($G$, $u$, $v$): remove $u$ da lista de adjacências de $v$ em $G$ e vice-versa.
\end{itemize}
\end{exampleblock}
\boxpurple{
\centering
~~~~~~\graphCreate{} : $\O{n}$\\
\graphAdd{}  : $\O{1}$\\
\graphDel{}  : $\O{1}$\\
}
\end{frame}


\begin{frame}{Ideia inicial}
\begin{algorithm}[H]
\caption{\dymGraphQuery($G$, $u$, $v$)}
\label{Algo:dymGraphQuery}
\begin{algorithmic}[1]
\State \Return \dymForestQuery($G$.$F$, $u$, $v$)
\end{algorithmic}
\end{algorithm}

\begin{algorithm}[H]
\caption{\dymGraphAddEdge($G$, $u$, $v$)}
\label{Algo:dymGraphAddEdge}
\begin{algorithmic}[1]
\If {\dymForestQuery($G.F$, $u$, $v$)}
\State \graphAdd($G$.$R$, $u$, $v$)
\Else 
\State \dymForestAddEdge($G.F$, $u$, $v$)
\EndIf
\end{algorithmic}
\end{algorithm}


\boxpurple{
\dymGraphQuery{} : $\O{\lg n}$\\
\dymGraphAddEdge{}  : $\O{\lg^2 n}$ amortizado\TODO{!!}
}
\end{frame}



\begin{frame}{Remoção de arestas}
\begin{block}{Busca por substituição de uma aresta}
\begin{itemize}
    \item Cada aresta possui um \defi{nível}, que é um inteiro entre~$1$ e $\lceil \log n \rceil$;
    \item Arestas serão inseridas no nível~$\lceil \log n \rceil$;
    \item O nível de uma aresta pode diminuir, mas nunca aumentar. 
\end{itemize}
\end{block}
\begin{block}{Estrutura}
$G_{\leqslant i}$: grafo com arestas de nível $\leqslant i$. Para cada camada $i$, manteremos:
\begin{itemize}
    \item $F_{\leqslant i}$: floresta maximal  de~$G_{\leqslant i}$; e
    \item $R_i$: arestas de nível~$i \notin F_{\leqslant i}$.
\end{itemize}
\end{block}
\begin{block}{Invariantes}
\begin{itemize}
    \item $F_{\leqslant i} \subseteq F_{\leqslant i+1}$, para cada $1\leqslant i \leqslant \lceil \log n \rceil-1$;
    \item $F_{\leqslant i}$ é uma floresta maximal de~$G_{\leqslant i}$; e
    \item Cada componente de $F_{\leqslant i}$ possui menos do que $2^i$ arestas.
\end{itemize}
\end{block}
\end{frame}

\begin{frame}{Implementações}
\begin{exampleblock}{Adaptações}
\centering
$G.F\rightarrow G.F_{\leqslant \lceil \log n \rceil}$\\
$G.R\rightarrow G.R_{\lceil \log n \rceil}$
\end{exampleblock}
\begin{algorithm}[H]
\caption{\dymGraphDelEdge($G$, $u$, $v$)}
\label{Algo:dymGraphDelEdge}
\begin{algorithmic}[1]
\State $i$ $\gets$ \nivel[$u,v$]
\State \nivel[$u,v$] $\gets$ $\Nil$
\If {$uv$ $\in G.F_{\leqslant\lceil \lg n \rceil}$}\label{Algo:dymGraphDelEdge:linha:if}
\For {$j$ $\gets$ $i$ até $\lceil \lg n \rceil$}\label{linha2}
\State \dymForestDelEdge($G$.$F_{\leqslant j}$, $u$, $v$)
\EndFor
\State \dymGraphReplace($G$, $u$, $v$, $i$)
\Else
  \State \graphDel($G$.$R_i$, $u$, $v$)\label{Algo:dymGraphDelEdge:linha:removeLA}
\EndIf
\end{algorithmic}
\end{algorithm}
\boxpurple{
\centering
\dymGraphDelEdge{}  : $\O{\lg^2 n}$ amortizado.
}
\end{frame}

\begin{frame}{Estrutura de níveis}
\begin{figure}[htb]
\scalebox{.5}{
\begin{tikzpicture}[line cap=round,line join=round,x=1cm,y=1cm]
\clip(-1.5,-.5) rectangle (7.5,5);
\draw (3.3,4.7) node {{\Huge ANTES}};
\draw (-.5,2) node {{\Huge $i$}};
\draw [line width=2pt,color=ccqqqq] (3,0)-- (3.677138935355088,0.9555967043150745);
\draw [line width=2pt] (3,0)-- (1,0);
\draw [line width=2pt] (1,0)-- (1,4);
\draw [line width=2pt] (1,0)-- (0,2);
\draw [line width=2pt,color=ccqqqq,dash pattern=on 1pt off 3pt] (3,0)-- (3,4);
\draw [line width=2pt,color=ccqqqq] (3,0)-- (5,0);
\draw [line width=2pt,color=ccqqqq] (5,0)-- (5,4);
\draw [line width=2pt,color=ccqqqq] (1,4)-- (3,4);
\draw [line width=2pt] (3,4)-- (5,0);
\draw [line width=2pt] (3.677138935355088,0.9555967043150745)-- (5,0);
\draw [line width=2pt] (5,4)-- (3,4);
\draw [line width=2pt,color=ccqqqq] (1,4)-- (0,2);
\draw [line width=2pt,color=ccqqqq] (1,0)-- (3,4);
\draw [line width=2pt] (3,4)-- (3.677138935355088,0.9555967043150745);
\draw [line width=2pt,color=ccqqqq] (5,4)-- (7,0);
\draw [line width=2pt] (7,0)-- (5,0);
\begin{scriptsize}
\draw [fill=black] (3,0) circle (2.5pt);
\draw [fill=black] (3.677138935355088,0.9555967043150745) circle (2.5pt);
\draw [fill=black] (1,0) circle (2.5pt);
\draw [fill=black] (1,4) circle (2.5pt);
\draw [fill=black] (0,2) circle (2.5pt);
\draw [fill=black] (3,4) circle (2.5pt);
\draw [fill=black] (5,0) circle (2.5pt);
\draw [fill=black] (5,4) circle (2.5pt);
\draw [fill=black] (7,0) circle (2.5pt);
\end{scriptsize}
\end{tikzpicture}

\begin{tikzpicture}[line cap=round,line join=round,x=1cm,y=1cm]
\clip(0,-.5) rectangle (5,4.5);
\end{tikzpicture}
\begin{tikzpicture}[line cap=round,line join=round,x=1cm,y=1cm]
\clip(-.5,-.5) rectangle (7.5,5);
\draw (3.3,4.7) node {{\Huge DEPOIS}};
\draw [line width=2pt,color=ccqqqq] (3,0)-- (3.677138935355088,0.9555967043150745);
\draw [line width=2pt,color=ccqqqq] (3,0)-- (1,0);
\draw [line width=2pt,color=ccqqqq] (3,0)-- (5,0);
\draw [line width=2pt,color=ccqqqq] (5,0)-- (5,4);
\draw [line width=2pt,color=ccqqqq] (1,4)-- (3,4);
\draw [line width=2pt] (3,4)-- (5,0);
\draw [line width=2pt] (3.677138935355088,0.9555967043150745)-- (5,0);
\draw [line width=2pt] (5,4)-- (3,4);
\draw [line width=2pt,color=ccqqqq] (1,4)-- (0,2);
\draw [line width=2pt,color=ccqqqq] (1,0)-- (3,4);
\draw [line width=2pt] (3,4)-- (3.677138935355088,0.9555967043150745);
\draw [line width=2pt,color=ccqqqq] (5,4)-- (7,0);
\draw [line width=2pt] (7,0)-- (5,0);
\begin{scriptsize}
\draw [fill=black] (3,0) circle (2.5pt);
\draw [fill=black] (3.677138935355088,0.9555967043150745) circle (2.5pt);
\draw [fill=black] (1,0) circle (2.5pt);
\draw [fill=black] (1,4) circle (2.5pt);
\draw [fill=black] (0,2) circle (2.5pt);
\draw [fill=black] (3,4) circle (2.5pt);
\draw [fill=black] (5,0) circle (2.5pt);
\draw [fill=black] (5,4) circle (2.5pt);
\draw [fill=black] (7,0) circle (2.5pt);
\end{scriptsize}
\end{tikzpicture}
}
\end{figure}
\begin{figure}[htb]
\scalebox{.5}{
\begin{tikzpicture}[line cap=round,line join=round,x=1cm,y=1cm]
\clip(-2,-.5) rectangle (7.5,4.5);
\draw (-1,2) node {{\Huge $i-1$}};
\begin{scriptsize}
\draw [fill=black] (3,0) circle (2.5pt);
\draw [fill=black] (3.677138935355088,0.9555967043150745) circle (2.5pt);
\draw [fill=black] (1,0) circle (2.5pt);
\draw [fill=black] (1,4) circle (2.5pt);
\draw [fill=black] (0,2) circle (2.5pt);
\draw [fill=black] (3,4) circle (2.5pt);
\draw [fill=black] (5,0) circle (2.5pt);
\draw [fill=black] (5,4) circle (2.5pt);
\draw [fill=black] (7,0) circle (2.5pt);
\end{scriptsize}
\end{tikzpicture}

\begin{tikzpicture}[line cap=round,line join=round,x=1cm,y=1cm]
\clip(0,-.5) rectangle (5,4.5);
\end{tikzpicture}
\begin{tikzpicture}[line cap=round,line join=round,x=1cm,y=1cm]
\clip(-.5,-.5) rectangle (7.5,4.5);
\draw [line width=2pt] (1,0)-- (1,4);
\draw [line width=2pt] (1,0)-- (0,2);
\draw [line width=2pt,color=ccqqqq] (1,4)-- (3,4);
\draw [line width=2pt,color=ccqqqq] (1,4)-- (0,2);
\draw [line width=2pt,color=ccqqqq] (1,0)-- (3,4);
\begin{scriptsize}
\draw [fill=black] (3,0) circle (2.5pt);
\draw [fill=black] (3.677138935355088,0.9555967043150745) circle (2.5pt);
\draw [fill=black] (1,0) circle (2.5pt);
\draw [fill=black] (1,4) circle (2.5pt);
\draw [fill=black] (0,2) circle (2.5pt);
\draw [fill=black] (3,4) circle (2.5pt);
\draw [fill=black] (5,0) circle (2.5pt);
\draw [fill=black] (5,4) circle (2.5pt);
\draw [fill=black] (7,0) circle (2.5pt);
\end{scriptsize}
\end{tikzpicture}}
\end{figure}
\end{frame}




\begin{frame}{A rotina \dymGraphReplace{}}
\begin{algorithm}[H]
\caption{\dymGraphReplace($G$,$u$,$v$,$niv$)}
\label{Algo:dymGraphReplace}
\begin{algorithmic}[1]
\For {$i$ $\gets$ $niv$ até $\lceil \lg n \rceil$}\label{Algo:dymGraphReplace:linha:primeira}
\State $T_v$ $\gets$  \treapGetRoot($F_{\leqslant i}[v,v]$)
\State $T_u$ $\gets$  \treapGetRoot($F_{\leqslant i}[u,u]$)
\If {\treapGetSize($T_v$) < \treapGetSize($T_u$)}\Comment{Garantimos que $|T_v|\geqslant |T_u|$}
\State $u$ $\leftrightarrow$ $v$
\State $T_u \leftrightarrow T_v$
\EndIf
\For {$xy$ em $T_u$ com \nivel[x, y] = $i$}\label{Algo:dymGraphReplace:linha:moveTu}\Comment{Move $T_u$ para o nível $i-1$}
\State \nivel$[x,y]$ $\gets$ $i-1$ \label{Algo:dymGraphReplace:linha:moveTu2}
\State \dymForestAddEdge($G$.$F_{\leqslant i-1}$, $x$, $y$) \label{Algo:dymGraphReplace:linha:moveTu3}
\EndFor
\For {$xy$ em $G$.$R_i$ com $x$ em $T_u$}\label{Algo:dymGraphReplace:linha:achaSub}\Comment{Procura substituta para $uv$}
\State \graphDel($G$.$R_i$, $x$, $y$)
	\If {\textbf{não} \dymForestQuery($F_{\leqslant i}$, $x$, $y$}\label{Algo:dymGraphReplace:linhayinTv}
\For {$j \gets i$ até $\lceil \lg n \rceil$}\label{Algo:dymGraphReplace:linha:inseresub}
\State \dymForestAddEdge($G$.$F_{\leqslant j}$, $x$, $y$)
\EndFor
\State \Return
\Else
\State \nivel$[x,y]$ $\gets$ $i-1$ \label{Algo:dymGraphReplace:linha:rebaixar}
\State \graphAdd($G$.$R_{i-1}$, $x$, $y$) \label{Algo:dymGraphReplace:linha:rebaixar2}
\EndIf
\EndFor
\EndFor\label{Algo:dymGraphReplace:linha:ultima}
\end{algorithmic}
\end{algorithm}
\end{frame}

\begin{frame}{Implementação}
\begin{block}{}
\begin{itemize}
\item Implementamos HDT em Python3;
\item O código dessa implementação está livremente disponível;
\item O código \LaTeX do dissertão também está disponível.
\end{itemize}
\end{block}
\end{frame}

% --------------------------------------------------------------------- %
% Floresta maximal de peso mínimo em grafos planos ponderados dinâmicos %
% --------------------------------------------------------------------- %
\section[MSF]{Floresta maximal de peso mínimo em grafos planos ponderados dinâmicos}

\subsection{Definição do problema}
\begin{frame}{Grafo plano}
\begin{block}{Grafo plano}
Um \defi{grafo plano} é um grafo $G = (V, E)$ com as seguintes propriedades:
\begin{enumerate}
\item $V\subset \R^2$;
\item Toda aresta é um arco entre dois vértices;
\item O interior de uma aresta não contém vértices nem intersecta outras arestas.
\end{enumerate}
\end{block}
\begin{minipage}[H]{0.4\textwidth}
\begin{tabular}{| c | c |} 
 \hline
 aresta & peso\\
 \hline
 $a$ & 2 \\ 
 \hline
$b$ & 7 \\
 \hline
$c$ & 3 \\
 \hline
$d$ & 1 \\
 \hline
$f$ & 2 \\
 \hline
$g$ & 4 \\
 \hline
\end{tabular}
\end{minipage}
\begin{minipage}[H]{0.4\textwidth}
\begin{figure}[H]
\scalebox{1}{
\begin{tikzpicture}[line cap=round,line join=round,x=1cm,y=1cm]
\clip(-1,-.5) rectangle (6,3);


\draw [line width=1pt] (0,0) to[in=-135,out=135,looseness=1]  (0,2); % v -- u
\draw [line width=1pt] (0,0) to[in=-45,out=45,looseness=1]  (0,2);   % v -- u
\draw [line width=1pt] (2,1) to  (0,0); % 4 -- 6
\draw [line width=1pt] (2,1) to  (0,2); % 4 -- 6
\draw [line width=1pt] (2,1) to  (3,1); % 4 -- 6
\draw [line width=1pt] (3,1.05) to[out=45,in=-45,looseness=50] (3,.95); % 4 -- 4


%\draw [line width=1pt] (1,2.5) to[out=-45,in=45,looseness=1] (1,1); % F0 -- F1
%\draw [line width=1pt] (1,2.5) to[out=20,in=-70,looseness=10] (1,1); % F0 -- F1
%\draw [line width=1pt] (1,2.5) to[out=170,in=180,looseness=2.5] (0,1); % F0 -- F2
%\draw [line width=1pt] (1,2.5) to[out=0,in=90,looseness=1] (2.5,1); % F0 -- F3
%\draw [line width=1pt] (1,1) to (0,1); % F1 -- F2




\begin{scriptsize}
\draw [fill=black] (0,2) circle (1.5pt);
\draw (0,2) node[anchor=south east] {$u$};


\draw [fill=black] (0,0) circle (1.5pt);
\draw (0,0) node[anchor=north east] {$v$};
\draw [fill=black] (2,1) circle (1.5pt);
\draw (1.95,1) node[anchor=south] {$z$};
\draw [fill=black] (3,1) circle (1.5pt);
\draw (2.95,1) node[anchor=south] {$y$};


%\draw [fill=black] (1,2.5) circle (1.5pt);
\draw (1,2.5) node {$F_0$};
%\draw [fill=black] (1,1) circle (1.5pt);
\draw (1,1) node {$F_1$};
%\draw [fill=black] (0,1) circle (1.5pt);
\draw (0,1) node {$F_2$};
%\draw [fill=black] (2.5,1) circle (1.5pt);
\draw (3.5,1) node {$F_3$};

%\draw (-0.3,1) node {$2$};
\draw (-0.53,1) node {$a$};

\draw (0.55,1.) node {$b$};
%\draw (0.53,1) node {$7$};

\draw (1.15,1.7) node[anchor=north] {$c$}; % u -- w
%\draw (1.05,1.5) node[anchor=north] {$3$}; % u -- w

\draw (1.2,.65) node[anchor=north] {$d$}; % v -- w
%\draw (1.35,.65) node[anchor=north] {$1$}; % v -- w

\draw (2.5,1) node[anchor=north] {$g$}; % v -- w
%\draw (2.5,.95) node[anchor=north] {$4$}; % v -- w


\draw (4.2,1) node {$f$}; % w -- w
%\draw (4.15,1) node {$2$}; % w -- w

\end{scriptsize}
\end{tikzpicture}

	}
\end{figure}
\end{minipage}

\end{frame}




\begin{frame}{Grafo dual}
\begin{block}{Grafo dual}
Dado um grafo plano~$G$, o grafo \defi{dual} de~$G$ é o grafo $G^\star = (F,E^\star)$, onde
\begin{itemize}
\item $F$ é o conjunto de faces de~$G$;
\item $E^\star$ é o conjunto de arestas duais de~$G$.
\end{itemize}
\end{block}
\begin{figure}[H]
\scalebox{1.2}{
\centering
\begin{tikzpicture}[line cap=round,line join=round,x=1cm,y=1cm]
\clip(-1,-1) rectangle (6,3);


\draw [line width=1pt] (0,0) to[in=-135,out=135,looseness=1]  (0,2); % v -- u
\draw [line width=1pt] (0,0) to[in=-45,out=45,looseness=1]  (0,2);   % v -- u
\draw [line width=1pt] (2,1) to  (0,0); % 4 -- 6
\draw [line width=1pt] (2,1) to  (0,2); % 4 -- 6
\draw [line width=1pt] (2,1) to  (3,1); % 4 -- 6
\draw [line width=1pt] (3,1.05) to[out=45,in=-45,looseness=50] (3,.95); % 4 -- 4


\draw [line width=1pt] (1,2.5) to[out=-45,in=45,looseness=1] (1,1); % c: F0 -- F1
\draw [line width=1pt] (1,2.5) to[out=20,in=160,looseness=1] (4.2,2.5); % d: F0 -- F1
\draw [line width=1pt] (4.2,2.5) to[out=-20,in=20,looseness=1] (4.2,-.5); % d: F0 -- F1
\draw [line width=1pt] (1,1) to[out=-70,in=200,looseness=1] (4.2,-.5); % d: F0 -- F1


% laço F0 para F0
\draw [line width=1pt] (1,2.5) to[out=-30,in=90,looseness=1] (2.5,1); % F0 -- F1
\draw [line width=1pt] (2.5,1) to[out=-90,in=-90,looseness=2] (4.5,1); % F0 -- F1
\draw [line width=1pt] (1,2.5) to[out=0,in=90,looseness=1] (4.5,1); % F0 -- F1


\draw [line width=1pt] (1,2.5) to[out=170,in=180,looseness=2.5] (0,1); % F0 -- F2
\draw [line width=1pt] (1,2.5) to[out=0,in=90,looseness=1] (3.5,1); % F0 -- F3
\draw [line width=1pt] (1,1) to (0,1); % F1 -- F2




\begin{scriptsize}
\draw [fill=black] (0,2) circle (1.5pt);
\draw (0,2) node[anchor=south east] {$u$};


\draw [fill=black] (0,0) circle (1.5pt);
\draw (0,0) node[anchor=north east] {$v$};
\draw [fill=black] (2,1) circle (1.5pt);
\draw (1.95,1) node[anchor=south] {$z$};
\draw [fill=black] (3,1) circle (1.5pt);
\draw (2.95,1) node[anchor=south] {$y$};


\draw [fill=black] (1,2.5) circle (1.5pt);
\draw (1,2.5) node[anchor=south east] {$F_0$};
\draw [fill=black] (1,1) circle (1.5pt);
\draw (1,1) node[anchor=west] {$F_1$};
\draw [fill=black] (0,1) circle (1.5pt);
\draw (0,1) node[anchor=south] {$F_2$};
\draw [fill=black] (3.5,1) circle (1.5pt);
\draw (3.5,1) node[anchor=west] {$F_3$};

\draw (-0.5,1.05) node[anchor=north] {$a$};
%\draw (-0.5,1.1) node[anchor=north] {$2$};

\draw (0.5,1.35) node[anchor=north] {$b$};
%\draw (0.5,1.05) node[anchor=north] {$7$};

\draw (1.15,1.7) node[anchor=north] {$c$}; % u -- w
%\draw (1.4,1.7) node[anchor=north] {$3$}; % u -- w

\draw (1.1,.6) node[anchor=north] {$d$}; % v -- w
%\draw (1.2,.65) node[anchor=north] {$1$}; % v -- w

\draw (2.3,1) node[anchor=north] {$g$}; % v -- w
%\draw (2.3,.95) node[anchor=north] {$4$}; % v -- w

\draw (4.15,1) node {$f$}; % w -- w
%\draw (4.15,1) node {$2$}; % w -- w

\end{scriptsize}
\end{tikzpicture}

}
\label{fig:MSF-basico-1}
\end{figure}
\end{frame}

\begin{frame}{Descrição combinatória plana}
%\begin{block}{Definição}
%Para cada vértice $v$ de um grafo plano~$G$, existe uma ordem cíclica~$D(v)$ das aresta incidentes a~$v$ obtida percorrendo as arestas incidentes a~$v$ em sentido anti-horário até retornar à aresta inicial do percurso.
%Uma \defi{descrição combinatória plana} de~$G$ é uma lista de todas essas ordens cíclicas.
%\end{block}
\begin{minipage}[H]{0.1\textwidth}
\begin{align*}
D(u)&=\langle (a,v), (b,v), (c,z)\rangle\\
D(v)&=\langle (a,u), (d,z), (b,u)  \rangle\\
D(z)&=\langle (c,u), (d,v), (g,y) \rangle \\
D(y)&=\langle (g,z), (f,y), (f,y) \rangle
\end{align*}
\begin{align*}
D(F_0)&=\langle (a^\star, F_2), (c^\star, F_1), \\
      &~~~~~~~(g^\star, F_0), (y^\star, F_3), \\
      &~~~~~~~(g^\star, F_0), (d^\star, F_1)\rangle\\
D(F_1)&=\langle (b^\star, F_2), (d^\star, F_0), (c^\star, F_0)  \rangle\\
D(F_2)&=\langle (a^\star, F_0), (b^\star, F_1)\rangle \\
D(F_3)&=\langle (f^\star, F_0)\rangle
\end{align*}
\end{minipage}
\hspace{-1cm}
\begin{minipage}[H]{0.4\textwidth}
\begin{figure}[H]
\scalebox{1.2}{
\centering
\begin{tikzpicture}[line cap=round,line join=round,x=1cm,y=1cm]
\clip(-1,-1) rectangle (6,3);


\draw [line width=1pt] (0,0) to[in=-135,out=135,looseness=1]  (0,2); % v -- u
\draw [line width=1pt] (0,0) to[in=-45,out=45,looseness=1]  (0,2);   % v -- u
\draw [line width=1pt] (2,1) to  (0,0); % 4 -- 6
\draw [line width=1pt] (2,1) to  (0,2); % 4 -- 6
\draw [line width=1pt] (2,1) to  (3,1); % 4 -- 6
\draw [line width=1pt] (3,1.05) to[out=45,in=-45,looseness=50] (3,.95); % 4 -- 4


\draw [line width=1pt] (1,2.5) to[out=-45,in=45,looseness=1] (1,1); % c: F0 -- F1
\draw [line width=1pt] (1,2.5) to[out=20,in=160,looseness=1] (4.2,2.5); % d: F0 -- F1
\draw [line width=1pt] (4.2,2.5) to[out=-20,in=20,looseness=1] (4.2,-.5); % d: F0 -- F1
\draw [line width=1pt] (1,1) to[out=-70,in=200,looseness=1] (4.2,-.5); % d: F0 -- F1


% laço F0 para F0
\draw [line width=1pt] (1,2.5) to[out=-30,in=90,looseness=1] (2.5,1); % F0 -- F1
\draw [line width=1pt] (2.5,1) to[out=-90,in=-90,looseness=2] (4.5,1); % F0 -- F1
\draw [line width=1pt] (1,2.5) to[out=0,in=90,looseness=1] (4.5,1); % F0 -- F1


\draw [line width=1pt] (1,2.5) to[out=170,in=180,looseness=2.5] (0,1); % F0 -- F2
\draw [line width=1pt] (1,2.5) to[out=0,in=90,looseness=1] (3.5,1); % F0 -- F3
\draw [line width=1pt] (1,1) to (0,1); % F1 -- F2




\begin{scriptsize}
\draw [fill=black] (0,2) circle (1.5pt);
\draw (0,2) node[anchor=south east] {$u$};


\draw [fill=black] (0,0) circle (1.5pt);
\draw (0,0) node[anchor=north east] {$v$};
\draw [fill=black] (2,1) circle (1.5pt);
\draw (1.95,1) node[anchor=south] {$z$};
\draw [fill=black] (3,1) circle (1.5pt);
\draw (2.95,1) node[anchor=south] {$y$};


\draw [fill=black] (1,2.5) circle (1.5pt);
\draw (1,2.5) node[anchor=south east] {$F_0$};
\draw [fill=black] (1,1) circle (1.5pt);
\draw (1,1) node[anchor=west] {$F_1$};
\draw [fill=black] (0,1) circle (1.5pt);
\draw (0,1) node[anchor=south] {$F_2$};
\draw [fill=black] (3.5,1) circle (1.5pt);
\draw (3.5,1) node[anchor=west] {$F_3$};

\draw (-0.5,1.05) node[anchor=north] {$a$};
%\draw (-0.5,1.1) node[anchor=north] {$2$};

\draw (0.5,1.35) node[anchor=north] {$b$};
%\draw (0.5,1.05) node[anchor=north] {$7$};

\draw (1.15,1.7) node[anchor=north] {$c$}; % u -- w
%\draw (1.4,1.7) node[anchor=north] {$3$}; % u -- w

\draw (1.1,.6) node[anchor=north] {$d$}; % v -- w
%\draw (1.2,.65) node[anchor=north] {$1$}; % v -- w

\draw (2.3,1) node[anchor=north] {$g$}; % v -- w
%\draw (2.3,.95) node[anchor=north] {$4$}; % v -- w

\draw (4.15,1) node {$f$}; % w -- w
%\draw (4.15,1) node {$2$}; % w -- w

\end{scriptsize}
\end{tikzpicture}

}
\label{fig:MSF-basico-1}
\end{figure}
\end{minipage}
\end{frame}

\begin{frame}{Floresta maximal de peso mínimo}
\begin{exampleblock}{O problema de floresta maximal de peso mínimo em grafos planos}
\begin{itemize}
\item \MSFCreate($n$): Cria e devolve um grafo plano ponderado~$G$ com $n$ vértices isolados.
\item \MSFaddEdge($G$, $e$, $u$, $e_u$, $v$, $e_v$, $w$): Insere em~$G$ uma nova aresta~$e$ com peso~$w$ ligando os vértices~$u$ e~$v$. A nova aresta~$e$ é sucessora das arestas~$e_u$ e~$e_v$ nas ordens cíclicas de~$u$ e~$v$, respectivamente.
\item \MSFdelEdge($G$, $e$): Remove a aresta~$e$ de~$G$.
\item \MSFupdate($G$, $e$, $w$): Altera o peso da aresta~$e$ de~$G$ para o valor~$w$.
\item \MSFweight($G$): Devolve o peso de uma MSF de $G$.
\end{itemize}
\end{exampleblock}

\begin{exampleblock}{Árvores dinâmicas planas}
Apresentaremos a solução proposta por Eppstein, Italiano, Tamassia, Tarjan, Westbrook e Yung para esse problema.
O nome da estrutura de dados introduzida por esses autores é \textit{edge-ordered dynamic tree} que traduzimos para \defi{árvores dinâmicas planas}.
\end{exampleblock}

\end{frame}

\begin{frame}{Árvore maximal e sua dual}
\begin{teorema}
\label{teo:MSFdual}
Seja~$T$ uma árvore geradora de um grafo plano conexo~$G$. O conjunto
$$
T^\star = \{e^\star:e\notin T\}
$$
é uma árvore geradora de~$G^\star$.
Além disso, se~$G$ for ponderado e adotarmos $w(e^\star) = w(e)$, então~$T$ será de peso mínimo em~$G$ se e somente se~$T^\star$ for de peso máximo em~$G^\star$.
\end{teorema}

\begin{figure}[H]
\scalebox{.6}{
\begin{tikzpicture}[line cap=round,line join=round,x=1cm,y=1cm]
\clip(-1,-1) rectangle (6,3);


\draw [line width=1pt] (0,0) to[in=-135,out=135,looseness=1]  (0,2); % v -- u
\draw [line width=1pt] (0,0) to[in=-45,out=45,looseness=1]  (0,2);   % v -- u
\draw [line width=1pt] (2,1) to  (0,0); % 4 -- 6
\draw [line width=1pt] (2,1) to  (0,2); % 4 -- 6
\draw [line width=1pt] (2,1) to  (3,1); % 4 -- 6
\draw [line width=1pt] (3,1.05) to[out=45,in=-45,looseness=50] (3,.95); % 4 -- 4


\draw [line width=1pt] (1,2.5) to[out=-45,in=45,looseness=1] (1,1); % c: F0 -- F1
\draw [line width=1pt] (1,2.5) to[out=20,in=160,looseness=1] (4.2,2.5); % d: F0 -- F1
\draw [line width=1pt] (4.2,2.5) to[out=-20,in=20,looseness=1] (4.2,-.5); % d: F0 -- F1
\draw [line width=1pt] (1,1) to[out=-70,in=200,looseness=1] (4.2,-.5); % d: F0 -- F1


% laço F0 para F0
\draw [line width=1pt] (1,2.5) to[out=-30,in=90,looseness=1] (2.5,1); % F0 -- F1
\draw [line width=1pt] (2.5,1) to[out=-90,in=-90,looseness=2] (4.5,1); % F0 -- F1
\draw [line width=1pt] (1,2.5) to[out=0,in=90,looseness=1] (4.5,1); % F0 -- F1


\draw [line width=1pt] (1,2.5) to[out=170,in=180,looseness=2.5] (0,1); % F0 -- F2
\draw [line width=1pt] (1,2.5) to[out=0,in=90,looseness=1] (3.5,1); % F0 -- F3
\draw [line width=1pt] (1,1) to (0,1); % F1 -- F2




\begin{scriptsize}
\draw [fill=black] (0,2) circle (1.5pt);
\draw (0,2) node[anchor=south east] {$u$};


\draw [fill=black] (0,0) circle (1.5pt);
\draw (0,0) node[anchor=north east] {$v$};
\draw [fill=black] (2,1) circle (1.5pt);
\draw (1.95,1) node[anchor=south] {$z$};
\draw [fill=black] (3,1) circle (1.5pt);
\draw (2.95,1) node[anchor=south] {$y$};


\draw [fill=black] (1,2.5) circle (1.5pt);
\draw (1,2.5) node[anchor=south east] {$F_0$};
\draw [fill=black] (1,1) circle (1.5pt);
\draw (1,1) node[anchor=west] {$F_1$};
\draw [fill=black] (0,1) circle (1.5pt);
\draw (0,1) node[anchor=south] {$F_2$};
\draw [fill=black] (3.5,1) circle (1.5pt);
\draw (3.5,1) node[anchor=west] {$F_3$};

\draw (-0.5,1.05) node[anchor=north] {$a$};
%\draw (-0.5,1.1) node[anchor=north] {$2$};

\draw (0.5,1.35) node[anchor=north] {$b$};
%\draw (0.5,1.05) node[anchor=north] {$7$};

\draw (1.15,1.7) node[anchor=north] {$c$}; % u -- w
%\draw (1.4,1.7) node[anchor=north] {$3$}; % u -- w

\draw (1.1,.6) node[anchor=north] {$d$}; % v -- w
%\draw (1.2,.65) node[anchor=north] {$1$}; % v -- w

\draw (2.3,1) node[anchor=north] {$g$}; % v -- w
%\draw (2.3,.95) node[anchor=north] {$4$}; % v -- w

\draw (4.15,1) node {$f$}; % w -- w
%\draw (4.15,1) node {$2$}; % w -- w

\end{scriptsize}
\end{tikzpicture}

}
\scalebox{1}{
\begin{tikzpicture}[line cap=round,line join=round,x=1cm,y=1cm]
\clip(-.6,-.5) rectangle (5,3);


\draw [line width=2pt,color=cyan] (0,0) to[in=-135,out=135,looseness=1]  (0,2); % v -- u
\draw [line width=.5pt] (0,0) to[in=-135,out=135,looseness=1]  (0,2); % v -- u
%\draw [line width=1pt] (0,0) to[in=-45,out=45,looseness=1]  (0,2);   % v -- u
\draw [line width=2pt,color=cyan] (2,1) to  (0,0); % 4 -- 6
\draw [line width=.5pt] (2,1) to  (0,0); % 4 -- 6
\draw [line width=2pt,color=cyan] (2,1) to  (3,1); % 4 -- 6
\draw [line width=.5pt] (2,1) to  (3,1); % 4 -- 6
%\draw [line width=1pt] (2,1) to  (0,2); % 4 -- 6
%\draw [line width=1pt] (2,1.05) to[out=45,in=-45,looseness=50] (2,.95); % 4 -- 4

\draw [line width=2pt,color=red] (1,2.5) to[out=-45,in=45,looseness=1] (1,1); % F0 -- F1
\draw [line width=.5pt] (1,2.5) to[out=-45,in=45,looseness=1] (1,1); % F0 -- F1
%\draw [line width=1pt] (1,2.5) to[out=20,in=-70,looseness=10] (1,1); % F0 -- F1
%\draw [line width=1pt] (1,2.5) to[out=170,in=180,looseness=2.5] (0,1); % F0 -- F2
\draw [line width=2pt,color=red] (1,2.5) to[out=0,in=90,looseness=1] (3.5,1); % F0 -- F3
\draw [line width=.5pt] (1,2.5) to[out=0,in=90,looseness=1] (3.5,1); % F0 -- F3
\draw [line width=2pt,color=red] (1,1) to (0,1); % F1 -- F2
\draw [line width=.5pt] (1,1) to (0,1); % F1 -- F2

\begin{scriptsize}
\draw [fill=black] (0,2) circle (1.5pt);
\draw (0,2) node[anchor=south east] {$u$};


\draw [fill=black] (0,0) circle (1.5pt);
\draw (0,0) node[anchor=north east] {$v$};
\draw [fill=black] (2,1) circle (1.5pt);
\draw (2,1) node[anchor=south] {$z$};
\draw [fill=black] (3,1) circle (1.5pt);
\draw (3,1) node[anchor=south] {$y$};


\draw [fill=black] (1,2.5) circle (1.5pt);
\draw (1,2.5) node[anchor=south east] {$F_0$};
\draw [fill=black] (1,1) circle (1.5pt);
\draw (.9,1) node[anchor=south] {$F_1$};
\draw [fill=black] (0,1) circle (1.5pt);
\draw (0,1) node[anchor=south] {$F_2$};
\draw [fill=black] (3.5,1) circle (1.5pt);
\draw (3.5,1) node[anchor=west] {$F_3$};


\draw (-0.3,1.05) node[anchor=north] {$a$};
\draw (-0.5,1.1) node[anchor=north] {$2$};

\draw (0.55,1.35) node[anchor=north] {$b^\star$};
\draw (0.5,1.05) node[anchor=north] {$7$};

\draw (1.15,1.9) node[anchor=north] {$c^\star$}; % u -- w
\draw (1.45,1.85) node[anchor=north] {$3$}; % u -- w

\draw (1.2,1) node[anchor=north] {$d$}; % v -- w
\draw (1.35,.65) node[anchor=north] {$1$}; % v -- w

\draw (2.5,1.4) node[anchor=north] {$g$}; % v -- w
\draw (2.5,1) node[anchor=north] {$4$}; % v -- w

\draw (2.55,2.7) node[anchor=north] {$f^\star$}; % w -- w
\draw (2.4,2.3) node[anchor=north] {$2$}; % w -- w

\end{scriptsize}
\end{tikzpicture}

}
\end{figure}
\end{frame}



\begin{frame}{Árvores modificadas}
\begin{minipage}[H]{0.3\textwidth}
\centering
\begin{tabular}{| c  c |} 
 \hline
 vértices & pesos\\
 \hline
 $\hat a$, $\hat a_1$, $\hat a_3$ & 2 \\ 
 \hline
 $\hat b$, $\hat b_0$, $\hat b_2$ & 7 \\
 \hline
 $\hat c$, $\hat c_0$, $\hat c_2$ & 3 \\
 \hline
 $\hat d$, $\hat d_0$, $\hat d_2$ & 2 \\
 \hline
 $\hat f$, $\hat f_1$, $\hat f_3$ & 1 \\
 \hline
 $\hat g$, $\hat g_1$, $\hat g_3$ & 4 \\
 \hline
 $\hat u$, $\hat v$, $\hat y$, $\hat z$ & $-\infty$ \\
 \hline
 $\hat F_0$, $\hat F_1$, $\hat F_2$, $\hat F_3$ & $\infty$ \\
 \hline
\end{tabular}
\end{minipage}
\begin{minipage}[H]{0.5\textwidth}
    \centering
\scalebox{.7}{
\begin{tikzpicture}[line cap=round,line join=round,x=1cm,y=1cm]
\clip(-2,-1) rectangle (9,6);

\draw [line width=2pt,color=cyan] (0,0) to[in=-135,out=135,looseness=1]  (0,4); % v -- u
\draw [line width=.5pt] (0,0) to[in=-135,out=135,looseness=1]  (0,4); % v -- u
%\draw [line width=1pt] (0,0) to[in=-45,out=45,looseness=1]  (0,4);   % v -- u

\draw [line width=2pt,color=cyan] (4,2) to  (0,0); % 4 -- 6
\draw [line width=.5pt] (4,2) to  (0,0); % 4 -- 6
%\draw [line width=1pt] (4,2) to  (0,4); % 4 -- 6
%\draw [line width=1pt] (4,2.05) to[out=45,in=-45,looseness=100] (4,1.95); % w -- w

\draw [line width=2pt,color=red] (2,5) to[out=-45,in=45,looseness=1] (2,2); % F0 -- F1
\draw [line width=.5pt] (2,5) to[out=-45,in=45,looseness=1] (2,2); % F0 -- F1
%\draw [line width=1pt] (2,5) to[out=20,in=-70,looseness=7.5] (2,2); % F0 -- F1
%\draw [line width=1pt] (2,5) to[out=170,in=180,looseness=2] (0,2); % F0 -- F2
\draw [line width=2pt,color=red] (2,5) to[out=0,in=90,looseness=1] (7,2); % F0 -- F3
\draw [line width=.5pt] (2,5) to[out=0,in=90,looseness=1] (7,2); % F0 -- F3

\draw [line width=2pt,color=red] (2,2) to (0,2); % F1 -- F2
\draw [line width=.5pt] (2,2) to (0,2); % F1 -- F2

\draw [line width=2pt,color=red] (2,5) to[out=0,in=90,looseness=1] (5,2.5); % F0 -- F0
\draw [line width=.5pt] (2,5) to[out=0,in=90,looseness=1] (5,2.5); % F0 -- F0


%F0 -- e1
\draw [line width=2pt,color=red] (2,5) to[out=0,in=90,looseness=1] (8.3,2); % F0 -- F0
\draw [line width=2pt,color=red] (8.3,2) to[out=-90,in=-90,looseness=2] (5, 1.5); % F0 -- F0
\draw [line width=.5pt] (2,5) to[out=0,in=90,looseness=1] (8.3,2); % F0 -- F0
\draw [line width=.5pt] (8.3,2) to[out=-90,in=-90,looseness=2] (5, 1.5); % F0 -- F0

\begin{scriptsize}

% Aresta a
\draw [line width=2pt,color=red] (0,2) to  (-.6,2); 
\draw [line width=.5pt] (0,2) to  (-.6,2); 
\draw [line width=2pt,color=red] (-1.1,2) to[out=135,in=180,looseness=1.5]  (2,5);   % v -- u
\draw [line width=.5pt] (-1.1,2) to[out=135,in=180,looseness=1.5]  (2,5);   % v -- u
\draw (-1,1.7) node[anchor=north] {{\normalsize $a_0$}};
\draw [fill=black] (-.83,1.7) circle (2pt);
\draw (-.5,1.9) node[anchor=north] {{\normalsize $a_1$}};
\draw [fill=black] (-.6,2) circle (2pt);
\draw (-1.05,2.6) node[anchor=north] {{\normalsize $a_2$}};
\draw [fill=black] (-.83,2.3) circle (2pt);
\draw (-1.25,2.03) node[anchor=north] {{\normalsize $a_3$}};
\draw [fill=black] (-1.1,2) circle (2pt);



% Aresta b
\draw [line width=2pt,color=cyan] (0,0) to[in=-90,out=45,looseness=1]  (.8,1.7);   % v -- u
\draw [line width=.5pt] (0,0) to[in=-90,out=45,looseness=1]  (.8,1.7);   % v -- u

\draw [line width=2pt,color=cyan] (0,4) to[out=-45,in=90,looseness=1]  (.8,2.3);   % v -- u
\draw [line width=.5pt] (0,4) to[out=-45,in=90,looseness=1]  (.8,2.3);   % v -- u

\draw (.6,2.7) node[anchor=north] {{\normalsize $b_0$}};
\draw [fill=black] (.8,2.3) circle (2pt);
\draw (.5,1.95) node[anchor=north] {{\normalsize $b_1$}};
\draw [fill=black] (.5,2) circle (2pt);
\draw (1.05,1.8) node[anchor=north] {{\normalsize $b_2$}};
\draw [fill=black] (.8,1.7) circle (2pt);
\draw (1.15,2.4) node[anchor=north] {{\normalsize $b_3$}};
\draw [fill=black] (1,2) circle (2pt);



% Aresta c

\draw [line width=2pt,color=cyan] (0,4) to  (2.2,2.9);   
\draw [line width=.5pt] (0,4) to  (2.2,2.9);   
\draw [line width=2pt,color=cyan] (2.7,2.65) to  (4,2);   
\draw [line width=.5pt] (2.7,2.65) to  (4,2);   

\draw (2.95,2.9) node[anchor=north] {{\normalsize $c_0$}}; % u -- w
\draw [fill=black] (2.7,2.65) circle (2pt);
\draw (2.4,3.3) node[anchor=north] {{\normalsize $c_1$}}; % u -- w
\draw [fill=black] (2.6,3.1) circle (2pt);
\draw (2,3) node[anchor=north] {{\normalsize $c_2$}}; % u -- w
\draw [fill=black] (2.2,2.9) circle (2pt);
\draw (2.5,2.5) node[anchor=north] {{\normalsize $c_3$}}; % u -- w
\draw [fill=black] (2.35,2.5) circle (2pt);



% Aresta d
\draw [line width=2pt,color=red] (2,2) to  (2.2,1.5);   
\draw [line width=.5pt] (2,2) to  (2.2,1.5);   

\draw [line width=2pt,color=red] (2.5,.9) to[out=-45,in=-90,looseness=1.5]  (8.5,2);   
\draw [line width=.5pt] (2.5,.9) to[out=-45,in=-90,looseness=1.5]  (8.5,2);   
\draw [line width=2pt,color=red] (2,5) to[out=10,in=90,looseness=1]  (8.5,2);   
\draw [line width=.5pt] (2,5) to[out=10,in=90,looseness=1]  (8.5,2);   



\draw (2.05,1.175) node[anchor=east] {{\normalsize $d_0$}}; % v -- w
\draw [fill=black] (2.1,1.05) circle (2pt);

\draw (2.35,.95) node[anchor=north] {{\normalsize $d_1$}}; % v -- w
\draw [fill=black] (2.5,.9) circle (2pt);

\draw (2.6,1.7) node[anchor=north] {{\normalsize $d_2$}}; % v -- w
\draw [fill=black] (2.65,1.275) circle (2pt);

\draw (1.95,1.8) node[anchor=north] {{\normalsize $d_3$}}; % v -- w
\draw [fill=black] (2.2,1.5) circle (2pt);

% Aresta e
\draw [line width=2pt,color=cyan] (4,2) to  (6,2);   
\draw [line width=.5pt] (4,2) to  (6,2);   

\draw (4.7,2.1) node[anchor=south] {{\normalsize $g_0$}}; % v -- w
\draw [fill=black] (4.7,2) circle (2pt);
\draw (5,1.4) node[anchor=east] {{\normalsize $g_1$}}; % v -- w
\draw [fill=black] (5,1.5) circle (2pt);
\draw (5.5,2) node[anchor=north] {{\normalsize $g_2$}}; % v -- w
\draw [fill=black] (5.3,2) circle (2pt);
\draw (5.1,2.6) node[anchor=west] {{\normalsize $g_3$}}; % v -- w
\draw [fill=black] (5,2.5) circle (2pt);



% Aresta f

\draw [line width=2pt,color=cyan] (6,2) to  (6.5,2.5);   
\draw [line width=.5pt] (6,2) to  (6.5,2.5);   
\draw [line width=2pt,color=cyan] (6,2) to[out=-45,in=-45,looseness=3]  (7.4,2.8);   
\draw [line width=.5pt] (6,2) to[out=-45,in=-45,looseness=3]  (7.4,2.8);   

\draw (6.5,3) node[anchor=north] {{\normalsize $f_0$}}; % w -- w
\draw [fill=black] (6.5,2.5) circle (2pt);
\draw (7.15,2.7) node[anchor=north] {{\normalsize $f_1$}}; % w -- w
\draw [fill=black] (6.95,2.5) circle (2pt);
\draw (7.4,3.3) node[anchor=north] {{\normalsize $f_2$}}; % w -- w
\draw [fill=black] (7.4,2.8) circle (2pt);
\draw (7,3.2) node[anchor=north] {{\normalsize $f_3$}}; % w -- w
\draw [fill=black] (6.8,3) circle (2pt);



\draw [fill=black] (0,4) circle (2pt);
\draw (0,4) node[anchor=south east] {{\Large $u$}};

\draw [fill=black] (0,0) circle (2pt);
\draw (0,0) node[anchor=north east] {{\Large $v$}};
\draw [fill=black] (4,2) circle (2pt);
\draw (4,2.1) node[anchor=south] {{\Large $z$}};
\draw [fill=black] (6,2) circle (2pt);
\draw (5.9,2) node[anchor=south] {{\Large $y$}};

\draw [fill=black] (2,5) circle (2pt);
\draw (2,5) node[anchor=south east] {{\Large $F_0$}};
\draw [fill=black] (2,2) circle (2pt);
\draw (1.8,2) node[anchor=south] {{\Large $F_1$}};
\draw [fill=black] (0,2) circle (2pt);
\draw (0,2.1) node[anchor=south] {{\Large $F_2$}};
\draw [fill=black] (7,2) circle (2pt);
\draw (7,1.7) node[anchor=west] {{\Large $F_3$}};


\end{scriptsize}
\end{tikzpicture}

}
\end{minipage}
\end{frame}

\begin{frame}{Exemplo de ADP}
\begin{figure}[htb]
\scalebox{1}{
\centering
\begin{tikzpicture}[line cap=round,line join=round,x=1cm,y=1cm]
\clip(-3,-.3) rectangle (10,5.6);

\begin{scriptsize}

%
% Node v
%
\draw (0,0) node {$\hat v$};
\draw [line width=1pt,color=cyan] (0,.5) to  (.35,.35); 
\draw [line width=1pt,color=cyan] (.35,.35) to  (.5,0); 

\draw [line width=1pt,color=cyan] (0,.5) to  (-1,1.8); 
\draw [line width=1pt,color=cyan] (.35,.35) to  (1.05,1.75); 
\draw [line width=1pt,color=cyan] (.5,0) to  (2.5,1); 


\draw (0,.5)[anchor=east] node {\tiny $a_v$};
\draw (.35,.35)[anchor=west] node {\tiny $b_v$};
\draw (.5,0)[anchor=north west] node {\tiny $d_v$};
\draw [fill=black] (0,.5) circle (1.5pt);
\draw [fill=black] (.35,.35) circle (1.5pt);
\draw [fill=black] (.5,0) circle (1.5pt);

\draw [line width=1pt,color=cyan] (-1,1.8) to  (-1,2.2); 
\draw (-.6,1.9) node[anchor=north east] {\tiny $a_0$};
\draw [fill=white] (-1,1.8) circle (1.5pt);

% ^b
\draw [line width=1pt,color=cyan] (2.5,1) to (2.9,1.15); 
\draw (1,1.65)[anchor=west] node {\tiny $b_2$};
\draw [fill=white] (1.05,1.75) circle (1.5pt);
\draw (2.5,1)[anchor=north] node {\tiny $d_0$};
\draw [fill=white] (2.5,1) circle (1.5pt);

%
% Node u
%
\draw (0,4) node {$\hat u$};

\draw [line width=1pt,color=cyan] (0,3.5) to  (.3,3.7); 
\draw [line width=1pt,color=cyan] (.5,4) to  (.3,3.7); 

\draw [line width=1pt,color=cyan] (0,3.5) to  (-1,2.2); 
\draw [line width=1pt,color=cyan] (.3,3.7) to  (1.05,2.25); 
\draw [line width=1pt,color=cyan] (.5,4) to  (2.5,3); 

\draw (-1.1,2.35) node {\tiny $a_2$};
\draw [fill=white] (-1,2.2) circle (1.5pt);
\draw (1.2,2.33) node {\tiny $b_0$};
\draw [fill=white] (1.05,2.25) circle (1.5pt);

\draw (2.4,2.9) node {\tiny $c_2$};
\draw [fill=white] (2.5,3) circle (1.5pt);

% u node path
\draw (-.2,3.5) node {\tiny $a_u$};
\draw [fill=black] (0,3.5) circle (1.5pt);
\draw (.5,4.15) node {\tiny $c_u$};
\draw [fill=black] (.5,4) circle (1.5pt);
\draw (.19,3.82) node {\tiny $b_u$};
\draw [fill=black] (.3,3.7) circle (1.5pt);

%
% Node z
%
\draw (3.95,2.1) node {$\hat z$};
\draw [line width=1pt,color=cyan] (4.3,2) to (6.5,2); 
\draw [line width=1pt,color=cyan] (3.7,2.3) to (3,2.7); 
\draw [line width=1pt,color=cyan] (3.7,1.7) to (2.9,1.15); 

% node path
\draw [line width=1pt,color=cyan] (3.7,2.3) to  (3.7,1.7); 
\draw [line width=1pt,color=cyan] (3.7,1.7) to  (4.3,2); 


\draw (3.75,1.55) node {\tiny $d_z$};
\draw [fill=black] (3.7,1.7)circle (1.5pt);
\draw (3.7,2.43) node {\tiny $c_z$};
\draw [fill=black] (3.7,2.3)circle (1.5pt);
\draw (4.3,2.2) node {\tiny $g_z$};
\draw [fill=black] (4.3,2)circle (1.5pt);

\draw (3.2,2.7) node {\tiny $c_0$};
\draw [fill=white] (3,2.7)circle (1.5pt);
\draw (3.05,1.25)[anchor=north] node {\tiny $d_2$};
\draw [fill=white] (2.9,1.15)circle (1.5pt);

%eF0 -- e3
\draw [line width=1pt,color=red] (3.3,4.7) to  (5.25,2.25);


%
% Node F_0
%
\draw (3,5.2) node {$\hat F_0$};
\draw [line width=1pt,color=red] (2.7,4.7) to  (3.8,4.7); 
\draw [line width=1pt,color=red] (3.8,4.7) to  (4,5); 
\draw [line width=1pt,color=red] (4,5) to  (2.5,5); 

\draw [line width=1pt,color=red] (2.5,5) to[in=180,out=180,looseness=2]  (-1.2,2); 
\draw [line width=1pt,color=red] (4,5) to[in=-20,out=10,looseness=3]  (5.25,1.75); 
\draw [line width=1pt,color=red] (3.5,5) to[in=-30,out=20,looseness=4.5]  (2.8,.85); 

%^c
\draw [line width=1pt,color=red] (2.6,2.6) to  (2.9,3.1);
\draw [line width=1pt,color=red] (2.7,4.7) to  (2.9,3.1);

\draw (-1.3,1.9) node {\tiny $a_3$};
\draw [fill=white] (-1.2,2)circle (1.5pt);
\draw (3.07,3.1) node {\tiny $c_1$};
\draw [fill=white] (2.9,3.1)circle (1.5pt);
\draw (2.9,.7) node {\tiny $d_1$};
\draw [fill=white] (2.8,.85)circle (1.5pt);


%fF3 -- 
\draw [line width=1pt,color=red] (6.5,3.9) to  (6.5,3);

% ^f
\draw [line width=1pt,color=cyan] (6.8,2.5) to[in=-60,out=60,looseness=1.5] (6.65,3.8);
\draw [line width=1pt,color=cyan] (6.2,2.5) to[in=-120,out=120,looseness=1.5] (6.35,3.8);

\draw [line width=1pt,color=red] (3.8,4.7) to  (6.5,3.95);
\draw (6.65,4.05) node {\tiny $f_0$};
\draw [fill=white] (6.5,3.95)circle (1.5pt);
\draw (6.65,3.55) node {\tiny $f_2$};
\draw [fill=white] (6.5,3.65)circle (1.5pt);
\draw (6.8,3.8) node {\tiny $f_3$};
\draw [fill=white] (6.65,3.8)circle (1.5pt);
\draw (6.2,3.8) node {\tiny $f_1$};
\draw [fill=white] (6.35,3.8)circle (1.5pt);


\draw (2.5,5)[anchor=south] node {\tiny $a_{F_0}$};
\draw [fill=black] (2.5,5)circle (1.5pt);
\draw (2.5,4.6) node {\tiny $c_{F_0}$};
\draw [fill=black] (2.7,4.7)circle (1.5pt);
\draw (3.5,5)[anchor=south] node {\tiny $d_{F_0}$};
\draw [fill=black] (3.5,5)circle (1.5pt);
\draw (4,5)[anchor=north west] node {\tiny $g_{F_0}$};
\draw [fill=black] (4,5)circle (1.5pt);
\draw (3.8,4.7)[anchor=north] node {\tiny $f_{F_0}$};
\draw [fill=black] (3.8,4.7)circle (1.5pt);
\draw (3.3,4.7)[anchor=north] node {\tiny $g_{F_0}$};
\draw [fill=black] (3.3,4.7)circle (1.5pt);

%
% Node y
%
\draw (6.5,2.35) node {$\hat y$};

%node path
\draw [line width=1pt,color=cyan] (6.5,2) to  (6.2,2.5); 
\draw [line width=1pt,color=cyan] (6.5,2) to  (6.8,2.5); 


\draw (6.5,1.8) node {\tiny $g_y$};
\draw (6,2.5) node {\tiny $f_y$};
\draw (7,2.5) node {\tiny $f_y$};
\draw [fill=black] (6.5,2)circle (1.5pt);
\draw [fill=black] (6.2,2.5)circle (1.5pt);
\draw [fill=black] (6.8,2.5)circle (1.5pt);


%
% Node F_1
%
\draw (2.1,2) node {$\hat F_1$};
%node path
\draw [line width=1pt,color=red] (2.3,2.3) to  (2.3,1.7); 
\draw [line width=1pt,color=red] (1.7,2) to  (2.3,1.7); 

\draw [line width=1pt,color=red] (2.6,2.6) to  (2.3,2.3); 
\draw [line width=1pt,color=red] (2.6,1.3) to  (2.3,1.7); 
\draw [line width=1pt,color=red] (1.7,2) to  (1.3,2); 

% F_1 to F_2
\draw [line width=1pt,color=red] (.8,2) to  (1.3,2); 

\draw (1.8,2.13) node {\tiny $b_{F_1}$};
\draw [fill=black] (1.7,2)circle (1.5pt);
\draw (2.3,1.7)[anchor=north] node {\tiny $d_{F_1}$};
\draw [fill=black] (2.3,1.7)circle (1.5pt);
\draw (2.25,2.2)[anchor=west] node {\tiny $c_{F_1}$};
\draw [fill=black] (2.3,2.3)circle (1.5pt);
\draw (1.4,1.85) node {\tiny $b_3$};
\draw [fill=white] (1.3,2)circle (1.5pt);

\draw (2.8,1.4) node {\tiny $d_3$};
\draw [fill=white] (2.6,1.3)circle (1.5pt);
\draw (2.75,2.55) node {\tiny $c_3$};
\draw [fill=white] (2.6,2.6)circle (1.5pt);

%
% Node F_2
%
\draw (0,2.2) node {$\hat F_2$};

\draw [line width=1pt,color=red] (-.8,2) to  (-.3,2); 
\draw [line width=1pt,color=red] (.8,2) to  (.3,2);   
\draw [line width=1pt,color=red] (-.3,2) to (.3,2);   

\draw [fill=black] (-.2,2) circle (1.5pt);
\draw (-.2,1.8) node {\tiny $a_{F_2}$};
\draw [fill=black] (.2,2) circle (1.5pt);
\draw (.3,1.8) node {\tiny $b_{F_2}$};
\draw (-.75,2.1) node {\tiny $a_1$};
\draw [fill=white] (-.8,2) circle (1.5pt);
\draw (.8,2.16) node {\tiny $b_1$};
\draw [fill=white] (.8,2) circle (1.5pt);

%
% Node F_3
%
\draw (6.7,3.2) node {$\hat F_3$};
\draw (6.3,3) node {\tiny $f_{F_3}$};
\draw [fill=black] (6.5,3)circle (1.5pt);

\draw (5,2)[anchor=south] node {\tiny $g_0$};
\draw [fill=white] (5,2)circle (1.5pt);
\draw (5.5,2)[anchor=south] node {\tiny $g_2$};
\draw [fill=white] (5.5,2)circle (1.5pt);
\draw (5.3,2.25)[anchor=south] node {\tiny $g_3$};
\draw [fill=white] (5.25,2.25)circle (1.5pt);
\draw (5.25,1.75)[anchor=north] node {\tiny $g_1$};
\draw [fill=white] (5.25,1.75)circle (1.5pt);

\end{scriptsize}
\end{tikzpicture}

}
\label{fig:MSF-figura-4}
\end{figure}
\end{frame}
\begin{frame}{Exemplo de ADP}
\begin{block}{Óctupla de $e$}
Há $8$ nós de link cut tree para cada aresta $e$ de~$G$.
Chamaremos esses~$8$ vértices de \defi{óctupla de $e$}.
\end{block}
\begin{figure}[htb]
\scalebox{1}{
\centering
\begin{tikzpicture}[line cap=round,line join=round,x=1cm,y=1cm]
\clip(-.3,0) rectangle (8,5.6);

\begin{scriptsize}


\draw [line width=1pt,color=cyan] (.35,.35) to  (1.05,1.75); 
\draw [line width=1pt,color=cyan] (.3,3.7) to  (1.05,2.25); 
\draw [line width=1pt,color=cyan] (4.3,2) to (6.5,2); 
\draw [line width=1pt,color=red] (3.3,4.7) to  (5.25,2.25);
\draw [line width=1pt,color=red] (4,5) to[in=-20,out=10,looseness=3]  (5.25,1.75); 
\draw [line width=1pt,color=red] (1.7,2) to  (.2,2); 
\draw (5,2)[anchor=south] node {\normalsize $g_0$};
\draw [fill=white] (5,2)circle (1.5pt);
\draw (5.7,2)[anchor=south] node {\normalsize $g_2$};
\draw [fill=white] (5.5,2)circle (1.5pt);
\draw (5.4,2.25)[anchor=south] node {\normalsize $g_3$};
\draw [fill=white] (5.25,2.25)circle (1.5pt);
\draw (5.25,1.75)[anchor=north] node {\normalsize $g_1$};
\draw [fill=white] (5.25,1.75)circle (1.5pt);
\draw (4.3,2.2) node {\normalsize $g_z$};
\draw [fill=black] (4.3,2)circle (1.5pt);
\draw (4,5)[anchor=east] node {\normalsize $g_{F_0}$};
\draw [fill=black] (4,5)circle (1.5pt);
\draw (3.3,4.7)[anchor=east] node {\normalsize $g_{F_0}$};
\draw [fill=black] (3.3,4.7)circle (1.5pt);
\draw (6.5,1.8) node {\normalsize $g_y$};
\draw [fill=black] (6.5,2)circle (1.5pt);

\draw (1.25,2.33) node {\normalsize $b_0$};
\draw [fill=white] (1.05,2.25) circle (1.5pt);
\draw (.8,2.16) node {\normalsize $b_1$};
\draw [fill=white] (.8,2) circle (1.5pt);
\draw (1,1.65)[anchor=west] node {\normalsize $b_2$};
\draw [fill=white] (1.05,1.75) circle (1.5pt);
\draw (1.45,1.8) node {\normalsize $b_3$};
\draw [fill=white] (1.3,2)circle (1.5pt);
\draw (.35,.35)[anchor=west] node {\normalsize $b_v$};
\draw [fill=black] (.35,.35) circle (1.5pt);
\draw (.3,3.7) node[anchor=west] {\normalsize $b_u$};
\draw [fill=black] (.3,3.7) circle (1.5pt);
\draw (.2,2) node[anchor=east] {\normalsize $b_{F_2}$};
\draw [fill=black] (.2,2) circle (1.5pt);
\draw (1.7,2) node[anchor=west] {\normalsize $b_{F_1}$};
\draw [fill=black] (1.7,2)circle (1.5pt);
\end{scriptsize}
\end{tikzpicture}

}
\end{figure}
\end{frame}

%\subsection{Árvores dinâmicas planas}
%\begin{frame}{Árvores Dinâmicas Planas}
%\begin{exampleblock}{Biblioteca ADP}
%\begin{itemize}
%\item \LCOMakeOcto($e$, $w$): Recebe um identificador~$e$ e um peso~$w$ e cria e retorna uma óctupla de nós de ADP associados a~$e$ com peso~$w$.
%\item \LCODestroyOcto($H$, $e$): Recebe uma tabela de símbolos~$H$ e um identificador~$e$ e desaloca a óctupla associada a~$e$ da memória.
%
%\item \LCOConnected($p$, $q$): Retorna verdadeiro se os nós~$p$ e~$q$ estiverem na mesma ADP e falso caso contrário.
%\item \LCOFindNode($p$): Recebe um nó $p$ e retorna o vértice da~$\hat F$ que contêm $p$ em sua ordem cíclica.
%
%\item \LCOAddCost($p$, $w$): Atribui o peso~$w$ ao vértice que contém $p$ em sua ordem cíclica.
%\item \LCOMax($p$, $q$): Retorna o nó de peso máximo no percurso entre os nós~$p$ e~$q$.
%Essa operação assume que $p$ e~$q$ são nós da mesma ADP.
%\item \LCOMin($p$, $q$): Retorna o nó de peso mínimo no percurso entre os nós~$p$ e~$q$.
%Essa operação assume que~$p$ e~$q$ são nós da mesma ADP.
%\item \LCOCycle($p$): Permuta ciclicamente $D(u)$ de forma que o nó~$p$ seja o último na ordem.
%Se a ordem inicial tem forma $\alpha p \beta$, então a ordem resultante é $\beta\alpha p$.
%\item \treapPredecessor($p$): Retorna o predecessor do nó~$p$ na ordem cíclica.
%\end{itemize}
%\end{exampleblock}
%\end{frame}

\subsection{Resolvendo MSF com ADPs}
\begin{frame}{Criação de grafo plano ponderado dinâmico e obtenção de peso}
\begin{algorithm}[H]
\caption{\MSFCreate($n$)}
\label{Algo:MSFCreate}
\begin{algorithmic}[1]
\State $G.H$ $\gets$ \hashCreate($n$)
\State $G.p$ $\gets$ $0$
\State \Return $G$
\end{algorithmic}
\end{algorithm}
\begin{algorithm}[H]
\caption{\MSFweight($G$)}
\label{Algo:MSFweight}
\begin{algorithmic}[1]
\State \Return $G$.$p$
\end{algorithmic}
\end{algorithm}
\boxpurple{
	\centering
	\MSFCreate:~$\O{n}$.\\
	\MSFweight:~$\O{1}$.\\
	\MSFaddEdge:~$\O{\lg m}$\\
	\MSFdelEdge:~$\O{\lg m}$.\\
	\MSFupdate:~$\O{\lg m}$.
}
\end{frame}
\begin{frame}{Execução de Mudança de peso \MSFupdate($G$, $a$, $5$)}
\begin{minipage}[H]{0.3\textwidth}
\begin{tabular}{| c | c |} 
 \hline
 aresta & peso\\
 \hline
 $a_i$ & 2 \\ 
 \hline
$b_i$ & 7 \\
 \hline
$c_i$ & 3 \\
 \hline
$d_i$ & 1 \\
 \hline
$f_i$ & 2 \\
 \hline
$g_i$ & 4 \\
 \hline
\end{tabular}
\end{minipage}
\begin{minipage}[H]{0.4\textwidth}
\begin{figure}[!h]
\scalebox{1.045}{
\input{fig/MSF-antes-mudança-peso}
}
\end{figure}
\end{minipage}
\end{frame}

\begin{frame}{Execução de Mudança de peso \MSFupdate($G$, $a$, $5$)}
\begin{figure}[!h]
\scalebox{1.4}{
\begin{tikzpicture}[line cap=round,line join=round,x=1cm,y=1cm]
\clip(-2,-1) rectangle (9,6);

\draw [line width=2pt,color=red] (2,5) to[out=-45,in=45,looseness=1] (2,2); % F0 -- F1
\draw [line width=.5pt] (2,5) to[out=-45,in=45,looseness=1] (2,2); % F0 -- F1

\draw [line width=2pt,color=red] (2,2) to (0,2); % F1 -- F2
\draw [line width=.5pt] (2,2) to (0,2); % F1 -- F2


\begin{scriptsize}

% Aresta a
\draw [line width=2pt,color=red] (0,2) to  (-.6,2); 
\draw [line width=.5pt] (0,2) to  (-.6,2); 
\draw [line width=2pt,color=red] (-1.1,2) to[out=135,in=180,looseness=1.5]  (2,5);   % v -- u
\draw [line width=.5pt] (-1.1,2) to[out=135,in=180,looseness=1.5]  (2,5);   % v -- u
\draw (-.6,2) node[anchor=south] {{\normalsize $a_1$}};
\draw [fill=black] (-.6,2) circle (2pt);
\draw (-.6,1.9) node[anchor=north] {{\normalsize $5$}};

\draw (-1.05,2.03) node[anchor=south] {{\normalsize $a_3$}};
\draw [fill=black] (-1.1,2) circle (2pt);
\draw (-1.05,1.9) node[anchor=north] {{\normalsize $5$}};



% Aresta b

\draw (.6,2) node[anchor=south] {{\normalsize $b_1$}};
\draw [fill=black] (.5,2) circle (2pt);
\draw (.5,1.9) node[anchor=north] {{\normalsize $7$}};

\draw (1.1,2) node[anchor=south] {{\normalsize $b_3$}};
\draw [fill=black] (1,2) circle (2pt);
\draw (1,1.9) node[anchor=north] {{\normalsize $7$}};



% Aresta c

\draw (2.4,3.3) node[anchor=north] {{\normalsize $c_1$}}; % u -- w
\draw [fill=black] (2.6,3.1) circle (2pt);
\draw (2.8,3.3) node[anchor=north] {{\normalsize $3$}}; % u -- w

\draw (2.15,2.8) node[anchor=north] {{\normalsize $c_3$}}; % u -- w
\draw [fill=black] (2.35,2.5) circle (2pt);
\draw (2.55,2.7) node[anchor=north] {{\normalsize $3$}}; % u -- w



% Aresta d
\draw [line width=2pt,color=red] (2,2) to  (2.2,1.5);   
\draw [line width=.5pt] (2,2) to  (2.2,1.5);   




%\draw (1.95,1.8) node[anchor=north] {{\normalsize $d_3$}}; % v -- w
\draw [fill=black] (2.2,1.5) circle (2pt);



\draw [fill=black] (2,5) circle (2pt);
\draw (2,5) node[anchor=south east] {{\Large $F_0$}};
\draw (2.5,5) node[anchor=south east] {{\Large $\infty$}};
\draw [fill=black] (2,2) circle (2pt);
\draw (1.8,2) node[anchor=south] {{\Large $F_1$}};
\draw [fill=black] (0,2) circle (2pt);
\draw (1.8,1.9) node[anchor=north] {{\Large $\infty$}};
\draw (0,2) node[anchor=south] {{\Large $F_2$}};
\draw (0,1.9) node[anchor=north] {{\Large $\infty$}};


\end{scriptsize}
\end{tikzpicture}

}
\end{figure}
\end{frame}


\begin{frame}{Execução de Mudança de peso \MSFupdate($G$, $a$, $5$)}
\begin{figure}[!h]
\scalebox{1}{
\begin{tikzpicture}[line cap=round,line join=round,x=1cm,y=1cm]
\clip(-3,-.3) rectangle (10,5.6);

\begin{scriptsize}

%
% Node v
%
\draw (0,0) node {$\hat v$};
\draw [line width=1pt,color=cyan] (0,.5) to  (.35,.35); 
\draw [line width=1pt,color=cyan] (.35,.35) to  (.5,0); 

\draw [line width=1pt,color=cyan] (0,.5) to  (-1,1.8); 
\draw [line width=1pt,color=cyan] (.35,.35) to  (1.05,1.75); 
\draw [line width=1pt,color=cyan] (.5,0) to  (2.5,1); 


%\draw (0,.5)[anchor=east] node {\tiny $a_v$};
%%\draw (.35,.35)[anchor=west] node {\tiny $b_v$};
%%\draw (.5,0)[anchor=north west] node {\tiny $d_v$};
\draw [fill=black] (0,.5) circle (1.5pt);
\draw [fill=black] (.35,.35) circle (1.5pt);
\draw [fill=black] (.5,0) circle (1.5pt);

%\draw [line width=1pt,color=cyan] (-1,1.8) to  (-1,2.2); 
\draw (-.6,1.9) node[anchor=north east] {\tiny $a_0$};
\draw [fill=white] (-1,1.8) circle (1.5pt);

% ^b
\draw [line width=1pt,color=cyan] (2.5,1) to (2.9,1.15); 
%\draw (1,1.65)[anchor=west] node {\tiny $b_2$};
\draw [fill=white] (1.05,1.75) circle (1.5pt);
%\draw (2.5,1)[anchor=north] node {\tiny $d_0$};
\draw [fill=white] (2.5,1) circle (1.5pt);

%
% Node u
%
\draw (0,4) node {$\hat u$};

\draw [line width=1pt,color=cyan] (0,3.5) to  (.3,3.7); 
\draw [line width=1pt,color=cyan] (.5,4) to  (.3,3.7); 

\draw [line width=1pt,color=cyan] (0,3.5) to  (-1,2.2); 
\draw [line width=1pt,color=cyan] (.3,3.7) to  (1.05,2.25); 
\draw [line width=1pt,color=cyan] (.5,4) to  (2.5,3); 

\draw (-1.1,2.35) node {\tiny $a_2$};
\draw [fill=white] (-1,2.2) circle (1.5pt);
%\draw (1.2,2.33) node {\tiny $b_0$};
\draw [fill=white] (1.05,2.25) circle (1.5pt);

\draw (2.4,2.9) node {\tiny $c_2$};
\draw [fill=white] (2.5,3) circle (1.5pt);

% u node path
%\draw (-.2,3.5) node {\tiny $a_u$};
\draw [fill=black] (0,3.5) circle (1.5pt);
%\draw (.5,4.15) node {\tiny $c_u$};
\draw [fill=black] (.5,4) circle (1.5pt);
%%\draw (.19,3.82) node {\tiny $b_u$};
\draw [fill=black] (.3,3.7) circle (1.5pt);

%
% Node z
%
\draw (3.95,2.1) node {$\hat z$};
\draw [line width=1pt,color=cyan] (4.3,2) to (6.5,2); 
\draw [line width=1pt,color=cyan] (3.7,2.3) to (3,2.7); 
\draw [line width=1pt,color=cyan] (3.7,1.7) to (2.9,1.15); 

% node path
\draw [line width=1pt,color=cyan] (3.7,2.3) to  (3.7,1.7); 
\draw [line width=1pt,color=cyan] (3.7,1.7) to  (4.3,2); 


%\draw (3.75,1.55) node {\tiny $d_z$};
\draw [fill=black] (3.7,1.7)circle (1.5pt);
%\draw (3.7,2.43) node {\tiny $c_z$};
\draw [fill=black] (3.7,2.3)circle (1.5pt);
%\draw (4.3,2.2) node {\tiny $e_z$};
\draw [fill=black] (4.3,2)circle (1.5pt);

\draw (3.2,2.7) node {\tiny $c_0$};
\draw [fill=white] (3,2.7)circle (1.5pt);
%\draw (3.05,1.25)[anchor=north] node {\tiny $d_2$};
\draw [fill=white] (2.9,1.15)circle (1.5pt);

%eF0 -- e3
\draw [line width=1pt,color=red] (3.3,4.7) to  (5.25,2.25);


%
% Node F_0
%
\draw (3,5.2) node {$\hat F_0$};
\draw [line width=1pt,color=red] (2.7,4.7) to  (3.8,4.7); 
\draw [line width=1pt,color=red] (3.8,4.7) to  (4,5); 
\draw [line width=1pt,color=red] (4,5) to  (2.5,5); 

\draw [line width=1pt,color=red] (2.5,5) to[in=180,out=180,looseness=2]  (-1.2,2); 
\draw [line width=1pt,color=red] (4,5) to[in=-20,out=10,looseness=3]  (5.25,1.75); 
\draw [line width=1pt,color=red] (3.5,5) to[in=-30,out=20,looseness=4.5]  (2.8,.85); 

%^c
%\draw [line width=1pt,color=red] (2.6,2.6) to  (2.9,3.1);
\draw [line width=1pt,color=red] (2.7,4.7) to  (2.9,3.1);

\draw (-1.3,1.9) node {\tiny $a_3$};
\draw [fill=white] (-1.2,2)circle (1.5pt);
\draw (3.07,3.1) node {\tiny $c_1$};
\draw [fill=white] (2.9,3.1)circle (1.5pt);
%\draw (2.9,.7) node {\tiny $d_1$};
\draw [fill=white] (2.8,.85)circle (1.5pt);


%fF3 -- 
\draw [line width=1pt,color=red] (6.5,3.9) to  (6.5,3);

% ^f
\draw [line width=1pt,color=cyan] (6.8,2.5) to[in=-60,out=60,looseness=1.5] (6.65,3.8);
\draw [line width=1pt,color=cyan] (6.2,2.5) to[in=-120,out=120,looseness=1.5] (6.35,3.8);

\draw [line width=1pt,color=red] (3.8,4.7) to  (6.5,3.95);
%\draw (6.65,4.05) node {\tiny $f_0$};
\draw [fill=white] (6.5,3.95)circle (1.5pt);
%\draw (6.65,3.55) node {\tiny $f_2$};
\draw [fill=white] (6.5,3.65)circle (1.5pt);
%\draw (6.8,3.8) node {\tiny $f_3$};
\draw [fill=white] (6.65,3.8)circle (1.5pt);
%\draw (6.2,3.8) node {\tiny $f_1$};
\draw [fill=white] (6.35,3.8)circle (1.5pt);


%\draw (2.5,5)[anchor=south] node {\tiny $a_{F_0}$};
\draw [fill=black] (2.5,5)circle (1.5pt);
%\draw (2.5,4.6) node {\tiny $c_{F_0}$};
\draw [fill=black] (2.7,4.7)circle (1.5pt);
%\draw (3.5,5)[anchor=south] node {\tiny $d_{F_0}$};
\draw [fill=black] (3.5,5)circle (1.5pt);
%\draw (4,5)[anchor=north west] node {\tiny $e_{F_0}$};
\draw [fill=black] (4,5)circle (1.5pt);
%\draw (3.8,4.7)[anchor=north] node {\tiny $f_{F_0}$};
\draw [fill=black] (3.8,4.7)circle (1.5pt);
%\draw (3.3,4.7)[anchor=north] node {\tiny $e_{F_0}$};
\draw [fill=black] (3.3,4.7)circle (1.5pt);

%
% Node y
%
\draw (6.5,2.35) node {$\hat y$};

%node path
\draw [line width=1pt,color=cyan] (6.5,2) to  (6.2,2.5); 
\draw [line width=1pt,color=cyan] (6.5,2) to  (6.8,2.5); 


%\draw (6.5,1.8) node {\tiny $e_y$};
%\draw (6,2.5) node {\tiny $f_y$};
%\draw (7,2.5) node {\tiny $f_y$};
\draw [fill=black] (6.5,2)circle (1.5pt);
\draw [fill=black] (6.2,2.5)circle (1.5pt);
\draw [fill=black] (6.8,2.5)circle (1.5pt);


%
% Node F_1
%
\draw (2.1,2) node {$\hat F_1$};
%node path
\draw [line width=1pt,color=red] (2.3,2.3) to  (2.3,1.7); 
\draw [line width=1pt,color=red] (1.7,2) to  (2.3,1.7); 

\draw [line width=1pt,color=red] (2.6,2.6) to  (2.3,2.3); 
\draw [line width=1pt,color=red] (2.6,1.3) to  (2.3,1.7); 
\draw [line width=1pt,color=red] (1.7,2) to  (1.3,2); 

% F_1 to F_2
\draw [line width=1pt,color=red] (.8,2) to  (1.3,2); 

%\draw (1.8,2.13) node {\tiny $b_{F_1}$};
\draw [fill=black] (1.7,2)circle (1.5pt);
%\draw (2.3,1.7)[anchor=north] node {\tiny $d_{F_1}$};
\draw [fill=black] (2.3,1.7)circle (1.5pt);
%\draw (2.25,2.2)[anchor=west] node {\tiny $c_{F_1}$};
\draw [fill=black] (2.3,2.3)circle (1.5pt);
%\draw (1.4,1.85) node {\tiny $b_3$};
\draw [fill=white] (1.3,2)circle (1.5pt);

%\draw (2.8,1.4) node {\tiny $d_3$};
\draw [fill=white] (2.6,1.3)circle (1.5pt);
\draw (2.75,2.55) node {\tiny $c_3$};
\draw [fill=white] (2.6,2.6)circle (1.5pt);

%
% Node F_2
%
\draw (0,2.2) node {$\hat F_2$};

\draw [line width=1pt,color=red] (-.8,2) to  (-.3,2); 
\draw [line width=1pt,color=red] (.8,2) to  (.3,2);   
\draw [line width=1pt,color=red] (-.3,2) to (.3,2);   

\draw [fill=black] (-.2,2) circle (1.5pt);
%\draw (-.2,1.8) node {\tiny $a_{F_2}$};
\draw [fill=black] (.2,2) circle (1.5pt);
%\draw (.3,1.8) node {\tiny $b_{F_2}$};
\draw (-.75,2.1) node {\tiny $a_1$};
\draw [fill=white] (-.8,2) circle (1.5pt);
%\draw (.8,2.16) node {\tiny $b_1$};
\draw [fill=white] (.8,2) circle (1.5pt);

%
% Node F_3
%
\draw (6.7,3.2) node {$\hat F_3$};
%\draw (6.3,3) node {\tiny $f_{F_3}$};
\draw [fill=black] (6.5,3)circle (1.5pt);

%\draw (5,2)[anchor=south] node {\tiny $e_0$};
\draw [fill=white] (5,2)circle (1.5pt);
%\draw (5.5,2)[anchor=south] node {\tiny $e_2$};
\draw [fill=white] (5.5,2)circle (1.5pt);
%\draw (5.25,2.25)[anchor=south] node {\tiny $e_3$};
\draw [fill=white] (5.25,2.25)circle (1.5pt);
%\draw (5.25,1.75)[anchor=west] node {\tiny $e_1$};
\draw [fill=white] (5.25,1.75)circle (1.5pt);

\end{scriptsize}
\end{tikzpicture}

}
\end{figure}
\end{frame}



\begin{frame}{Execução de Mudança de peso \MSFupdate($G$, $a$, $5$)}
\begin{figure}[htb]
\scalebox{1}{
\centering
\begin{tikzpicture}[line cap=round,line join=round,>=triangle 45,x=1cm,y=1cm]
\clip(-3,-.3) rectangle (10,5.6);

\begin{scriptsize}

%
% Node v
%
\draw (0,0) node {$\hat v$};
\draw [line width=1pt,color=cyan] (0,.5) to  (.35,.35); 
\draw [line width=1pt,color=cyan] (.35,.35) to  (.5,0); 

\draw [line width=1pt,color=cyan] (0,.5) to  (-1,1.8); 
\draw [line width=1pt,color=cyan] (.35,.35) to  (1.05,1.75); 
\draw [line width=1pt,color=cyan] (.5,0) to  (2.5,1); 


\draw (0,.5)[anchor=east] node {\tiny $a_v$};
\draw (.35,.35)[anchor=west] node {\tiny $b_v$};
\draw (.5,0)[anchor=north west] node {\tiny $d_v$};
\draw [fill=black] (0,.5) circle (1.5pt);
\draw [fill=black] (.35,.35) circle (1.5pt);
\draw [fill=black] (.5,0) circle (1.5pt);

\draw (-.6,1.9) node[anchor=north east] {\tiny $a_0$};
\draw [fill=white] (-1,1.8) circle (1.5pt);

% ^b
\draw [line width=1pt,color=cyan] (2.5,1) to (2.9,1.15); 
\draw (1,1.65)[anchor=west] node {\tiny $b_2$};
\draw [fill=white] (1.05,1.75) circle (1.5pt);
\draw (2.5,1)[anchor=north] node {\tiny $d_0$};
\draw [fill=white] (2.5,1) circle (1.5pt);

%
% Node u
%
\draw (0,4) node {$\hat u$};

\draw [line width=1pt,color=cyan] (0,3.5) to  (.3,3.7); 
\draw [line width=1pt,color=cyan] (.5,4) to  (.3,3.7); 

\draw [line width=1pt,color=cyan] (0,3.5) to  (-1,2.2); 
\draw [line width=1pt,color=cyan] (.3,3.7) to  (1.05,2.25); 
\draw [line width=1pt,color=cyan] (.5,4) to  (2.5,3); 

\draw (-1.1,2.35) node {\tiny $a_2$};
\draw [fill=white] (-1,2.2) circle (1.5pt);
\draw (1.2,2.33) node {\tiny $b_0$};
\draw [fill=white] (1.05,2.25) circle (1.5pt);

% u node path
\draw (-.2,3.5) node {\tiny $a_u$};
\draw [fill=black] (0,3.5) circle (1.5pt);
\draw (.5,4.15) node {\tiny $c_u$};
\draw [fill=black] (.5,4) circle (1.5pt);
\draw (.19,3.82) node {\tiny $b_u$};
\draw [fill=black] (.3,3.7) circle (1.5pt);

%
% Node z
%
\draw (3.95,2.1) node {$\hat z$};
\draw [line width=1pt,color=cyan] (4.3,2) to (6.5,2); 
\draw [line width=1pt,color=cyan] (3.7,2.3) to (2.5,3); 
\draw [line width=1pt,color=cyan] (3.7,1.7) to (2.9,1.15); 
\draw (2.4,2.9) node {\tiny $c_2$};
\draw [fill=white] (2.5,3) circle (1.5pt);


% node path
\draw [line width=1pt,color=cyan] (3.7,2.3) to  (3.7,1.7); 
\draw [line width=1pt,color=cyan] (3.7,1.7) to  (4.3,2); 


\draw (3.75,1.55) node {\tiny $d_z$};
\draw [fill=black] (3.7,1.7)circle (1.5pt);
\draw (3.7,2.43) node {\tiny $c_z$};
\draw [fill=black] (3.7,2.3)circle (1.5pt);
\draw (4.3,2.2) node {\tiny $g_z$};
\draw [fill=black] (4.3,2)circle (1.5pt);

\draw (3.2,2.7) node {\tiny $c_0$};
\draw [fill=white] (3,2.7)circle (1.5pt);
\draw (3.05,1.25)[anchor=north] node {\tiny $d_2$};
\draw [fill=white] (2.9,1.15)circle (1.5pt);

%eF0 -- e3
\draw [line width=1pt,color=red] (3.3,4.7) to  (5.25,2.25);


%
% Node F_0
%
\draw (3,5.2) node {$\hat F_0$};
\draw [line width=1pt,color=red] (2.7,4.7) to  (3.8,4.7); 
\draw [line width=1pt,color=red] (3.8,4.7) to  (4,5); 
\draw [line width=1pt,color=red] (4,5) to  (2.5,5); 

\draw [line width=1pt,color=red] (2.5,5) to[in=180,out=180,looseness=2]  (-1.2,2); 
\draw [line width=1pt,color=red] (4,5) to[in=-20,out=10,looseness=3]  (5.25,1.75); 
\draw [line width=1pt,color=red] (3.5,5) to[in=-30,out=20,looseness=4.5]  (2.8,.85); 

%^c
\draw [line width=1pt,color=red] (2.7,4.7) to  (2.9,3.1);

\draw (3.07,3.1) node {\tiny $c_1$};
\draw [fill=white] (2.9,3.1)circle (1.5pt);
\draw (2.9,.7) node {\tiny $d_1$};
\draw [fill=white] (2.8,.85)circle (1.5pt);


%fF3 -- 
\draw [line width=1pt,color=red] (6.5,3.9) to  (6.5,3);

% ^f
\draw [line width=1pt,color=cyan] (6.8,2.5) to[in=-60,out=60,looseness=1.5] (6.65,3.8);
\draw [line width=1pt,color=cyan] (6.2,2.5) to[in=-120,out=120,looseness=1.5] (6.35,3.8);

\draw [line width=1pt,color=red] (3.8,4.7) to  (6.5,3.95);
\draw (6.65,4.05) node {\tiny $f_0$};
\draw [fill=white] (6.5,3.95)circle (1.5pt);
\draw (6.65,3.55) node {\tiny $f_2$};
\draw [fill=white] (6.5,3.65)circle (1.5pt);
\draw (6.8,3.8) node {\tiny $f_3$};
\draw [fill=white] (6.65,3.8)circle (1.5pt);
\draw (6.2,3.8) node {\tiny $f_1$};
\draw [fill=white] (6.35,3.8)circle (1.5pt);


\draw (2.5,5)[anchor=south] node {\tiny $a_{F_0}$};
\draw [fill=black] (2.5,5)circle (1.5pt);
\draw (2.5,4.6) node {\tiny $c_{F_0}$};
\draw [fill=black] (2.7,4.7)circle (1.5pt);
\draw (3.5,5)[anchor=south] node {\tiny $d_{F_0}$};
\draw [fill=black] (3.5,5)circle (1.5pt);
\draw (4,5)[anchor=north west] node {\tiny $g_{F_0}$};
\draw [fill=black] (4,5)circle (1.5pt);
\draw (3.8,4.7)[anchor=north] node {\tiny $f_{F_0}$};
\draw [fill=black] (3.8,4.7)circle (1.5pt);
\draw (3.3,4.7)[anchor=north] node {\tiny $g_{F_0}$};
\draw [fill=black] (3.3,4.7)circle (1.5pt);

%
% Node y
%
\draw (6.5,2.35) node {$\hat y$};

%node path
\draw [line width=1pt,color=cyan] (6.5,2) to  (6.2,2.5); 
\draw [line width=1pt,color=cyan] (6.5,2) to  (6.8,2.5); 


\draw (6.5,1.8) node {\tiny $g_y$};
\draw (6,2.5) node {\tiny $f_y$};
\draw (7,2.5) node {\tiny $f_y$};
\draw [fill=black] (6.5,2)circle (1.5pt);
\draw [fill=black] (6.2,2.5)circle (1.5pt);
\draw [fill=black] (6.8,2.5)circle (1.5pt);


%
% Node F_1
%
\draw (2.1,2) node {$\hat F_1$};
%node path
\draw [line width=1pt,color=red] (2.3,2.3) to  (2.3,1.7); 
\draw [line width=1pt,color=red] (1.7,2) to  (2.3,1.7); 

\draw [line width=1pt,color=red] (2.6,2.6) to  (2.3,2.3); 
\draw [line width=1pt,color=red] (2.6,1.3) to  (2.3,1.7); 
\draw [line width=1pt,color=red] (1.7,2) to  (1.3,2); 

% F_1 to F_2
\draw [line width=1pt,color=red] (.8,2) to  (1.3,2); 

\draw (1.8,2.13) node {\tiny $b_{F_1}$};
\draw [fill=black] (1.7,2)circle (1.5pt);
\draw (2.3,1.7)[anchor=north] node {\tiny $d_{F_1}$};
\draw [fill=black] (2.3,1.7)circle (1.5pt);
\draw (2.25,2.2)[anchor=west] node {\tiny $c_{F_1}$};
\draw [fill=black] (2.3,2.3)circle (1.5pt);
\draw (1.4,1.85) node {\tiny $b_3$};
\draw [fill=white] (1.3,2)circle (1.5pt);

\draw (2.8,1.4) node {\tiny $d_3$};
\draw [fill=white] (2.6,1.3)circle (1.5pt);
\draw (2.75,2.55) node {\tiny $c_3$};
\draw [fill=white] (2.6,2.6)circle (1.5pt);

%
% Node F_2
%
\draw (0,2.2) node {$\hat F_2$};

\draw [line width=1pt,color=red] (-1.2,2) to  (-.3,2); 
\draw [line width=1pt,color=red] (.8,2) to  (.3,2);   
\draw [line width=1pt,color=red] (-.3,2) to (.3,2);   

\draw [fill=black] (-.2,2) circle (1.5pt);
\draw (-.2,1.8) node {\tiny $a_{F_2}$};
\draw [fill=black] (.2,2) circle (1.5pt);
\draw (.3,1.8) node {\tiny $b_{F_2}$};
\draw (-.75,2.1) node {\tiny $a_1$};
\draw [fill=white] (-.8,2) circle (1.5pt);
\draw (.8,2.16) node {\tiny $b_1$};
\draw [fill=white] (.8,2) circle (1.5pt);

\draw (-1.3,1.9) node {\tiny $a_3$};
\draw [fill=white] (-1.2,2)circle (1.5pt);

%
% Node F_3
%
\draw (6.7,3.2) node {$\hat F_3$};
\draw (6.3,3) node {\tiny $f_{F_3}$};
\draw [fill=black] (6.5,3)circle (1.5pt);

\draw (5,2)[anchor=south] node {\tiny $g_0$};
\draw [fill=white] (5,2)circle (1.5pt);
\draw (5.5,2)[anchor=south] node {\tiny $g_2$};
\draw [fill=white] (5.5,2)circle (1.5pt);
\draw (5.3,2.25)[anchor=south] node {\tiny $g_3$};
\draw [fill=white] (5.25,2.25)circle (1.5pt);
\draw (5.25,1.75)[anchor=north] node {\tiny $g_1$};
\draw [fill=white] (5.25,1.75)circle (1.5pt);

\end{scriptsize}
\end{tikzpicture}

}
\end{figure}
\end{frame}





\begin{frame}{Dois casos de remoção de aresta}
\begin{figure}[htb]
\scalebox{0.7}{
\begin{tikzpicture}[line cap=round,line join=round,x=1cm,y=1cm]
\clip(-1,-1) rectangle (6,3);


\draw [line width=1pt] (0,0) to[in=-135,out=135,looseness=1]  (0,2); % v -- u
\draw [line width=1pt] (0,0) to[in=-45,out=45,looseness=1]  (0,2);   % v -- u
\draw [line width=1pt] (2,1) to  (0,0); % 4 -- 6
\draw [line width=1pt] (2,1) to  (0,2); % 4 -- 6
\draw [line width=1pt] (2,1) to  (3,1); % 4 -- 6
\draw [line width=1pt] (3,1.05) to[out=45,in=-45,looseness=50] (3,.95); % 4 -- 4


\draw [line width=1pt] (1,2.5) to[out=-45,in=45,looseness=1] (1,1); % c: F0 -- F1
\draw [line width=1pt] (1,2.5) to[out=20,in=160,looseness=1] (4.2,2.5); % d: F0 -- F1
\draw [line width=1pt] (4.2,2.5) to[out=-20,in=20,looseness=1] (4.2,-.5); % d: F0 -- F1
\draw [line width=1pt] (1,1) to[out=-70,in=200,looseness=1] (4.2,-.5); % d: F0 -- F1


% laço F0 para F0
\draw [line width=1pt] (1,2.5) to[out=-30,in=90,looseness=1] (2.5,1); % F0 -- F1
\draw [line width=1pt] (2.5,1) to[out=-90,in=-90,looseness=2] (4.5,1); % F0 -- F1
\draw [line width=1pt] (1,2.5) to[out=0,in=90,looseness=1] (4.5,1); % F0 -- F1


\draw [line width=1pt] (1,2.5) to[out=170,in=180,looseness=2.5] (0,1); % F0 -- F2
\draw [line width=1pt] (1,2.5) to[out=0,in=90,looseness=1] (3.5,1); % F0 -- F3
\draw [line width=1pt] (1,1) to (0,1); % F1 -- F2




\begin{scriptsize}
\draw [fill=black] (0,2) circle (1.5pt);
\draw (0,2) node[anchor=south east] {$u$};


\draw [fill=black] (0,0) circle (1.5pt);
\draw (0,0) node[anchor=north east] {$v$};
\draw [fill=black] (2,1) circle (1.5pt);
\draw (1.95,1) node[anchor=south] {$z$};
\draw [fill=black] (3,1) circle (1.5pt);
\draw (2.95,1) node[anchor=south] {$y$};


\draw [fill=black] (1,2.5) circle (1.5pt);
\draw (1,2.5) node[anchor=south east] {$F_0$};
\draw [fill=black] (1,1) circle (1.5pt);
\draw (1,1) node[anchor=west] {$F_1$};
\draw [fill=black] (0,1) circle (1.5pt);
\draw (0,1) node[anchor=south] {$F_2$};
\draw [fill=black] (3.5,1) circle (1.5pt);
\draw (3.5,1) node[anchor=west] {$F_3$};

\draw (-0.5,1.05) node[anchor=north] {$a$};
%\draw (-0.5,1.1) node[anchor=north] {$2$};

\draw (0.5,1.35) node[anchor=north] {$b$};
%\draw (0.5,1.05) node[anchor=north] {$7$};

\draw (1.15,1.7) node[anchor=north] {$c$}; % u -- w
%\draw (1.4,1.7) node[anchor=north] {$3$}; % u -- w

\draw (1.1,.6) node[anchor=north] {$d$}; % v -- w
%\draw (1.2,.65) node[anchor=north] {$1$}; % v -- w

\draw (2.3,1) node[anchor=north] {$g$}; % v -- w
%\draw (2.3,.95) node[anchor=north] {$4$}; % v -- w

\draw (4.15,1) node {$f$}; % w -- w
%\draw (4.15,1) node {$2$}; % w -- w

\end{scriptsize}
\end{tikzpicture}

}
\end{figure}
\begin{minipage}[H]{0.4\textwidth}
\begin{figure}[htb]
\scalebox{0.7}{
\begin{tikzpicture}[line cap=round,line join=round,>=triangle 45,x=1cm,y=1cm]
\clip(-1,-1) rectangle (6,3);


\draw [line width=1pt] (0,0) to[in=-135,out=135,looseness=1]  (0,2); % v -- u
\draw [line width=1pt] (0,0) to[in=-45,out=45,looseness=1]  (0,2);   % v -- u
\draw [line width=1pt] (2,1) to  (0,0); % 4 -- 6
\draw [line width=1pt] (2,1) to  (0,2); % 4 -- 6
%\draw [line width=1pt] (2,1) to  (3,1); % 4 -- 6
\draw [line width=1pt] (3,1.05) to[out=45,in=-45,looseness=50] (3,.95); % 4 -- 4


\draw [line width=1pt] (1,2.5) to[out=-45,in=45,looseness=1] (1,1); % c: F0 -- F1
\draw [line width=1pt] (1,2.5) to[out=0,in=90,looseness=1] (2.5,1); % d: F0 -- F1
\draw [line width=1pt] (1,1) to[out=-55,in=-90,looseness=2] (2.5,1); % d: F0 -- F1


% laço F0 para F0
%\draw [line width=1pt] (1,2.5) to[out=-30,in=90,looseness=1] (2.5,1); % F0 -- F1
%\draw [line width=1pt] (2.5,1) to[out=-90,in=-90,looseness=2] (4.5,1); % F0 -- F1
%\draw [line width=1pt] (1,2.5) to[out=0,in=90,looseness=1] (4.5,1); % F0 -- F1


\draw [line width=1pt] (1,2.5) to[out=170,in=180,looseness=2.5] (0,1); % F0 -- F2
\draw [line width=1pt] (3.5,2) to (3.5,1); % F0 -- F3
\draw [line width=1pt] (1,1) to (0,1); % F1 -- F2




\begin{scriptsize}
\draw [fill=black] (0,2) circle (1.5pt);
\draw (0,2) node[anchor=south east] {$u$};


\draw [fill=black] (0,0) circle (1.5pt);
\draw (0,0) node[anchor=north east] {$v$};
\draw [fill=black] (2,1) circle (1.5pt);
\draw (1.95,1) node[anchor=south] {$z$};
\draw [fill=black] (3,1) circle (1.5pt);
\draw (2.95,1) node[anchor=south] {$y$};


\draw [fill=black] (1,2.5) circle (1.5pt);
\draw (1,2.5) node[anchor=south east] {$F_4$};
\draw [fill=black] (1,1) circle (1.5pt);
\draw (1,1) node[anchor=west] {$F_1$};
\draw [fill=black] (0,1) circle (1.5pt);
\draw (0,1) node[anchor=south] {$F_2$};
\draw [fill=black] (3.5,1) circle (1.5pt);
\draw (3.5,1) node[anchor=west] {$F_3$};
\draw [fill=black] (3.5,2) circle (1.5pt);
\draw (3.5,2) node[anchor=west] {$F_5$};

\draw (-0.5,1.05) node[anchor=north] {$a$};
%\draw (-0.5,1.1) node[anchor=north] {$2$};

\draw (0.5,1.35) node[anchor=north] {$b$};
%\draw (0.5,1.05) node[anchor=north] {$7$};

\draw (1.15,1.7) node[anchor=north] {$c$}; % u -- w
%\draw (1.4,1.7) node[anchor=north] {$3$}; % u -- w

\draw (1.1,.6) node[anchor=north] {$d$}; % v -- w
%\draw (1.2,.65) node[anchor=north] {$1$}; % v -- w

%\draw (2.3,1.3) node[anchor=north] {$e$}; % v -- w
%\draw (2.3,.95) node[anchor=north] {$4$}; % v -- w

\draw (4.15,1) node {$f$}; % w -- w
%\draw (4.15,1) node {$2$}; % w -- w

\end{scriptsize}
\end{tikzpicture}

}
\end{figure}
\end{minipage}
\begin{minipage}[H]{0.4\textwidth}
\begin{figure}[htb]
\scalebox{0.7}{
\begin{tikzpicture}[line cap=round,line join=round,>=triangle 45,x=1cm,y=1cm]
\clip(-1,-1) rectangle (6,3);


\draw [line width=1pt] (0,0) to[in=-135,out=135,looseness=1]  (0,2); % v -- u
%\draw [line width=1pt] (0,0) to[in=-45,out=45,looseness=1]  (0,2);   % v -- u
\draw [line width=1pt] (2,1) to  (0,0); % 4 -- 6
\draw [line width=1pt] (2,1) to  (0,2); % 4 -- 6
\draw [line width=1pt] (2,1) to  (3,1); % 4 -- 6
\draw [line width=1pt] (3,1.05) to[out=45,in=-45,looseness=50] (3,.95); % 4 -- 4


\draw [line width=1pt] (1,2.5) to[out=-45,in=45,looseness=1] (1,1); % c: F0 -- F1
\draw [line width=1pt] (1,2.5) to[out=20,in=160,looseness=1] (4.2,2.5); % d: F0 -- F1
\draw [line width=1pt] (4.2,2.5) to[out=-20,in=20,looseness=1] (4.2,-.5); % d: F0 -- F1
\draw [line width=1pt] (1,1) to[out=-70,in=200,looseness=1] (4.2,-.5); % d: F0 -- F1


% laço F0 para F0
\draw [line width=1pt] (1,2.5) to[out=-30,in=90,looseness=1] (2.5,1); % F0 -- F1
\draw [line width=1pt] (2.5,1) to[out=-90,in=-90,looseness=2] (4.5,1); % F0 -- F1
\draw [line width=1pt] (1,2.5) to[out=0,in=90,looseness=1] (4.5,1); % F0 -- F1


\draw [line width=1pt] (1,2.5) to[out=170,in=180,looseness=2.5] (0,1); % F0 -- F2
\draw [line width=1pt] (1,2.5) to[out=0,in=90,looseness=1] (3.5,1); % F0 -- F3
\draw [line width=1pt] (1,1) to (0,1); % F1 -- F2




\begin{scriptsize}
\draw [fill=black] (0,2) circle (1.5pt);
\draw (0,2) node[anchor=south east] {$u$};


\draw [fill=black] (0,0) circle (1.5pt);
\draw (0,0) node[anchor=north east] {$v$};
\draw [fill=black] (2,1) circle (1.5pt);
\draw (1.95,1) node[anchor=south] {$z$};
\draw [fill=black] (3,1) circle (1.5pt);
\draw (2.95,1) node[anchor=south] {$y$};


\draw [fill=black] (1,2.5) circle (1.5pt);
\draw (1,2.5) node[anchor=south east] {$F_0$};
\draw [fill=black] (1,1) circle (1.5pt);
\draw (1,1) node[anchor=west] {$F_6$};
%\draw [fill=black] (0,1) circle (1.5pt);
%\draw (0,1) node[anchor=south] {$F_2$};
\draw [fill=black] (3.5,1) circle (1.5pt);
\draw (3.5,1) node[anchor=west] {$F_3$};

\draw (-0.5,1.05) node[anchor=north] {$a$};
%\draw (-0.5,1.1) node[anchor=north] {$2$};

%\draw (0.5,1.35) node[anchor=north] {$b$};
%\draw (0.5,1.05) node[anchor=north] {$7$};

\draw (1.15,1.7) node[anchor=north] {$c$}; % u -- w
%\draw (1.4,1.7) node[anchor=north] {$3$}; % u -- w

\draw (1.1,.6) node[anchor=north] {$d$}; % v -- w
%\draw (1.2,.65) node[anchor=north] {$1$}; % v -- w

\draw (2.3,1.35) node[anchor=north] {$g$}; % v -- w
%\draw (2.3,.95) node[anchor=north] {$4$}; % v -- w

\draw (4.15,1) node {$f$}; % w -- w
%\draw (4.15,1) node {$2$}; % w -- w

\end{scriptsize}
\end{tikzpicture}

}
\end{figure}
\end{minipage}
\end{frame}

\begin{frame}{Remoção de ponte}
\begin{figure}
\begin{tikzpicture}[line cap=round,line join=round,x=1cm,y=1cm]
\clip(-1,-2) rectangle (6,3);

\draw (0,0) circle [radius=1cm];
\draw (4,0) circle [radius=1cm];
\draw (0,0) node[anchor=north] {$G_1$};
\draw (4,0) node[anchor=north] {$G_2$};


\draw [line width=1pt] (2,2.5) to[out=170,in=90,looseness=1] (-.5,.5); % F0 -- F2
\draw [line width=1pt] (2,2.5) to[out=175,in=90,looseness=1] (-.3,.3); % F0 -- F2
\draw [line width=1pt] (2,2.5) to[out=185,in=90,looseness=1] (0,0); % F0 -- F2
\draw [line width=1pt] (2,2.5) to[out=195,in=90,looseness=1] (.2,-.3); % F0 -- F2



%e*
\draw [line width=1pt] (2,2.5) to (2,0); % F0 -- F1
\draw [line width=1pt] (2,0) to[out=-90,in=-90,looseness=1.5] (5.5,0); % F0 -- F1
\draw [line width=1pt] (2,2.5) to[out=20,in=90,looseness=1] (5.5,0); % F0 -- F1


\draw [line width=1pt] (2,2.5) to[out=0,in=90,looseness=1] (4.3,.7); % F0 -- F3
\draw [line width=1pt] (2,2.5) to[out=0,in=90,looseness=1] (4.2,.5); % F0 -- F3
\draw [line width=1pt] (2,2.5) to[out=0,in=90,looseness=1] (3.9,.4); % F0 -- F3

\draw [line width=1pt] (2,2.5) to[out=-40,in=180,looseness=1] (3.5,.3); % F0 -- F3
\draw [line width=1pt] (2,2.5) to[out=-50,in=180,looseness=1] (3.4,.1); % F0 -- F3
\draw [line width=1pt] (2,2.5) to[out=-60,in=180,looseness=1] (3.3,.0); % F0 -- F3

% e
\draw [line width=1pt] (1,0) to (3,0); 
\draw (2,0) node[anchor=south east] {$e$};


\draw (2,1.5) node[anchor=south east] {$v_3$};
\draw (3,2.8) node[anchor=west] {$v_1$};


\draw (2,2.5) node[anchor=south] {$F_e$};
\draw [fill=black] (2,2.5) circle (1.5pt);


% Arcos
%\draw (2,2.5) circle [radius=1cm];
\draw[red,line width=1pt] (2.15,1.5) arc (-80:-50:1);
\draw[line width=1pt,->,red] (2.6,1.7) to (2.65,1.75);
\draw (2.75,1.5) node[red] {$\gamma$};


\draw[red,line width=1pt] (2.85,2) arc (-35:-5:1);
\draw[line width=1pt,->,red] (3.025,2.5) to (3.025,2.6);
\draw (3.3,2.4) node[red] {$\alpha$};

\draw[red,line width=1pt] (1,2.5) arc (180:220:1);
\draw[line width=1pt,->,red] (1.2,1.9) to (1.3,1.8);
\draw (1.2,1.55) node[red] {$\beta$};

\end{tikzpicture}

\end{figure}
\end{frame}

\begin{frame}{Remoção de ponte - \MSFdelEdge($G$, $g$)}
\begin{figure}
\begin{tikzpicture}[line cap=round,line join=round,x=1cm,y=1cm]
\clip(2,-.3) rectangle (10,5.6);

\begin{scriptsize}

%
% Node v
%
\draw (0,0) node {$\hat v$};
\draw [line width=1pt,color=cyan] (0,.5) to  (.35,.35); 
\draw [line width=1pt,color=cyan] (.35,.35) to  (.5,0); 

\draw [line width=1pt,color=cyan] (0,.5) to  (-1,1.8); 
\draw [line width=1pt,color=cyan] (.35,.35) to  (1.05,1.75); 
\draw [line width=1pt,color=cyan] (.5,0) to  (2.5,1); 


\draw (0,.5)[anchor=east] node {\tiny $a_v$};
\draw (.35,.35)[anchor=west] node {\tiny $b_v$};
%\draw (.5,0)[anchor=north west] node {\tiny $d_v$};
\draw [fill=black] (0,.5) circle (1.5pt);
\draw [fill=black] (.35,.35) circle (1.5pt);
\draw [fill=black] (.5,0) circle (1.5pt);

\draw (-.6,1.9) node[anchor=north east] {\tiny $a_0$};
\draw [fill=white] (-1,1.8) circle (1.5pt);

% ^b
\draw [line width=1pt,color=cyan] (2.5,1) to (2.9,1.15); 
\draw (1,1.65)[anchor=west] node {\tiny $b_2$};
\draw [fill=white] (1.05,1.75) circle (1.5pt);
%\draw (2.5,1)[anchor=north] node {\tiny $d_0$};
\draw [fill=white] (2.5,1) circle (1.5pt);

%
% Node u
%
\draw (0,4) node {$\hat u$};

\draw [line width=1pt,color=cyan] (0,3.5) to  (.3,3.7); 
\draw [line width=1pt,color=cyan] (.5,4) to  (.3,3.7); 

\draw [line width=1pt,color=cyan] (0,3.5) to  (-1,2.2); 
\draw [line width=1pt,color=cyan] (.3,3.7) to  (1.05,2.25); 
\draw [line width=1pt,color=cyan] (.5,4) to  (2.5,3); 

\draw (-1.1,2.35) node {\tiny $a_2$};
\draw [fill=white] (-1,2.2) circle (1.5pt);
\draw (1.2,2.33) node {\tiny $b_0$};
\draw [fill=white] (1.05,2.25) circle (1.5pt);

% u node path
\draw (-.2,3.5) node {\tiny $a_u$};
\draw [fill=black] (0,3.5) circle (1.5pt);
%\draw (.5,4.15) node {\tiny $c_u$};
\draw [fill=black] (.5,4) circle (1.5pt);
\draw (.19,3.82) node {\tiny $b_u$};
\draw [fill=black] (.3,3.7) circle (1.5pt);

%
% Node z
%
\draw (3.8,2) node {$\hat z$};
\draw [line width=1pt,color=cyan] (4.3,2) to (6.5,2); 
\draw [line width=1pt,color=cyan] (3.7,2.3) to (2.5,3); 
\draw [line width=1pt,color=cyan] (3.7,1.7) to (2.9,1.15); 
%\draw (2.4,2.9) node {\tiny $c_2$};
\draw [fill=white] (2.5,3) circle (1.5pt);


% node path
\draw [line width=1pt,color=cyan] (3.7,2.3) to  (3.7,1.7); 
%\draw [line width=1pt,color=cyan] (3.7,1.7) to  (4.3,2); 


%\draw (3.75,1.55) node {\tiny $d_z$};
\draw [fill=black] (3.7,1.7)circle (1.5pt);
%\draw (3.7,2.43) node {\tiny $c_z$};
\draw [fill=black] (3.7,2.3)circle (1.5pt);
\draw (4.3,2.2) node {\tiny $g_z$};
\draw [fill=black] (4.3,2)circle (1.5pt);

%\draw (3.2,2.7) node {\tiny $c_0$};
\draw [fill=white] (3,2.7)circle (1.5pt);
%\draw (3.05,1.25)[anchor=north] node {\tiny $d_2$};
\draw [fill=white] (2.9,1.15)circle (1.5pt);

%eF0 -- e3
\draw [line width=1pt,color=red] (3.3,4.7) to  (5.25,2.25);


%
% Node F_0
%
\draw (3,5.2) node {$\hat F_0$};
\draw [line width=1pt,color=red] (2.7,4.7) to  (3.8,4.7); 
\draw [line width=1pt,color=red] (2.5,5) to  (2.7,4.7); 
\draw [line width=1pt,color=red] (3.5,5) to  (2.5,5); 

\draw [line width=1pt,color=red] (2.5,5) to[in=180,out=180,looseness=2]  (-1.2,2); 
\draw [line width=1pt,color=red] (4,5) to[in=-20,out=10,looseness=3]  (5.25,1.75); 
\draw [line width=1pt,color=red] (3.5,5) to[in=-30,out=20,looseness=4.5]  (2.8,.85); 

%^c
\draw [line width=1pt,color=red] (2.7,4.7) to  (2.9,3.1);

%\draw (3.07,3.1) node {\tiny $c_1$};
\draw [fill=white] (2.9,3.1)circle (1.5pt);
%\draw (2.9,.7) node {\tiny $d_1$};
\draw [fill=white] (2.8,.85)circle (1.5pt);


%fF3 -- 
\draw [line width=1pt,color=red] (6.5,3.9) to  (6.5,3);

% ^f
\draw [line width=1pt,color=cyan] (6.8,2.5) to[in=-60,out=60,looseness=1.5] (6.65,3.8);
\draw [line width=1pt,color=cyan] (6.2,2.5) to[in=-120,out=120,looseness=1.5] (6.35,3.8);

\draw [line width=1pt,color=red] (3.8,4.7) to  (6.5,3.95);
%\draw (6.65,4.05) node {\tiny $f_0$};
\draw [fill=white] (6.5,3.95)circle (1.5pt);
%\draw (6.65,3.55) node {\tiny $f_2$};
\draw [fill=white] (6.5,3.65)circle (1.5pt);
%\draw (6.8,3.8) node {\tiny $f_3$};
\draw [fill=white] (6.65,3.8)circle (1.5pt);
%\draw (6.2,3.8) node {\tiny $f_1$};
\draw [fill=white] (6.35,3.8)circle (1.5pt);


\draw (2.5,5)[anchor=south] node {\tiny $a_{F_0}$};
\draw [fill=black] (2.5,5)circle (1.5pt);
\draw (2.5,4.6) node {\tiny $c_{F_0}$};
\draw [fill=black] (2.7,4.7)circle (1.5pt);
\draw (3.5,5)[anchor=south] node {\tiny $d_{F_0}$};
\draw [fill=black] (3.5,5)circle (1.5pt);
\draw (4,5)[anchor=north west] node {\tiny $g_{F_0}$};
\draw [fill=black] (4,5)circle (1.5pt);
\draw (3.8,4.7)[anchor=north] node {\tiny $f_{F_0}$};
\draw [fill=black] (3.8,4.7)circle (1.5pt);
\draw (3.3,4.7)[anchor=north] node {\tiny $g_{F_0}$};
\draw [fill=black] (3.3,4.7)circle (1.5pt);

%
% Node y
%
\draw (7,2.35) node {$\hat y$};

%node path
\draw [line width=1pt,color=cyan] (6.8,2.5) to  (6.2,2.5); 
%\draw [line width=1pt,color=cyan] (6.5,2) to  (6.8,2.5); 


\draw (6.5,1.8) node {\tiny $g_y$};
%\draw (6,2.5) node {\tiny $f_y$};
%\draw (7,2.5) node {\tiny $f_y$};
\draw [fill=black] (6.5,2)circle (1.5pt);
\draw [fill=black] (6.2,2.5)circle (1.5pt);
\draw [fill=black] (6.8,2.5)circle (1.5pt);


%
% Node F_1
%
%\draw (2.1,2) node {$\hat F_1$};
%node path
\draw [line width=1pt,color=red] (2.3,2.3) to  (2.3,1.7); 
\draw [line width=1pt,color=red] (1.7,2) to  (2.3,1.7); 

\draw [line width=1pt,color=red] (2.6,2.6) to  (2.3,2.3); 
\draw [line width=1pt,color=red] (2.6,1.3) to  (2.3,1.7); 
\draw [line width=1pt,color=red] (1.7,2) to  (1.3,2); 

% F_1 to F_2
\draw [line width=1pt,color=red] (.8,2) to  (1.3,2); 

\draw (1.8,2.13) node {\tiny $b_{F_1}$};
\draw [fill=black] (1.7,2)circle (1.5pt);
%\draw (2.3,1.7)[anchor=north] node {\tiny $d_{F_1}$};
\draw [fill=black] (2.3,1.7)circle (1.5pt);
%\draw (2.25,2.2)[anchor=west] node {\tiny $c_{F_1}$};
\draw [fill=black] (2.3,2.3)circle (1.5pt);
\draw (1.4,1.85) node {\tiny $b_3$};
\draw [fill=white] (1.3,2)circle (1.5pt);

%\draw (2.8,1.4) node {\tiny $d_3$};
\draw [fill=white] (2.6,1.3)circle (1.5pt);
%\draw (2.75,2.55) node {\tiny $c_3$};
\draw [fill=white] (2.6,2.6)circle (1.5pt);

%
% Node F_2
%
\draw (0,2.2) node {$\hat F_2$};

\draw [line width=1pt,color=red] (-1.2,2) to  (-.3,2); 
\draw [line width=1pt,color=red] (.8,2) to  (.3,2);   
\draw [line width=1pt,color=red] (-.3,2) to (.3,2);   

\draw [fill=black] (-.2,2) circle (1.5pt);
\draw (-.2,1.8) node {\tiny $a_{F_2}$};
\draw [fill=black] (.2,2) circle (1.5pt);
\draw (.3,1.8) node {\tiny $b_{F_2}$};
\draw (-.75,2.1) node {\tiny $a_1$};
\draw [fill=white] (-.8,2) circle (1.5pt);
\draw (.8,2.16) node {\tiny $b_1$};
\draw [fill=white] (.8,2) circle (1.5pt);

\draw (-1.3,1.9) node {\tiny $a_3$};
\draw [fill=white] (-1.2,2)circle (1.5pt);

%
% Node F_3
%
%\draw (6.7,3.2) node {$\hat F_3$};
%\draw (6.3,3) node {\tiny $f_{F_3}$};
\draw [fill=black] (6.5,3)circle (1.5pt);

\draw (5,2)[anchor=south] node {\tiny $g_0$};
\draw [fill=white] (5,2)circle (1.5pt);
\draw (5.5,2)[anchor=south] node {\tiny $g_2$};
\draw [fill=white] (5.5,2)circle (1.5pt);
\draw (5.25,2.25)[anchor=south] node {\tiny $g_3$};
\draw [fill=white] (5.25,2.25)circle (1.5pt);
\draw (5.25,1.75)[anchor=north] node {\tiny $g_1$};
\draw [fill=white] (5.25,1.75)circle (1.5pt);

\end{scriptsize}
\end{tikzpicture}

\end{figure}
\end{frame}

\begin{frame}{Remoção de não ponte}
\begin{minipage}{.45\textwidth}
\begin{figure}[h!]
\scalebox{.85}{
\begin{tikzpicture}[line cap=round,line join=round,x=1cm,y=1cm]
\clip(-1,-2.3) rectangle (6,3);

\draw (0,0) circle [radius=1cm];
\draw (4,0) circle [radius=1cm];
\draw (0,0) node[anchor=north] {$G$};
\draw (4,0) node[anchor=north] {$G$};


\draw [line width=1pt] (2,1.5) to[out=170,in=90,looseness=1] (-.5,.5); % F0 -- F2
\draw [line width=1pt] (2,1.5) to[out=175,in=90,looseness=1] (-.3,.3); % F0 -- F2
\draw [line width=1pt] (2,1.5) to[out=185,in=90,looseness=1] (0,0); % F0 -- F2
\draw [line width=1pt] (2,1.5) to[out=195,in=90,looseness=1] (.4,0); % F0 -- F2


\draw [line width=1pt] (2,1.5) to[out=0,in=90,looseness=1] (4.3,.7); % F0 -- F3
\draw [line width=1pt] (2,1.5) to[out=0,in=90,looseness=1] (4.2,.5); % F0 -- F3
\draw [line width=1pt] (2,1.5) to[out=0,in=90,looseness=1] (3.9,.4); % F0 -- F3


\draw [line width=1pt] (2,-1.5) to[out=-170,in=-90,looseness=1] (-.7,-.3); % F0 -- F2
\draw [line width=1pt] (2,-1.5) to[out=-175,in=-90,looseness=1] (-.3,-.5); % F0 -- F2
\draw [line width=1pt] (2,-1.5) to[out=-185,in=-90,looseness=1] (.3,-.7); % F0 -- F2
\draw [line width=1pt] (2,-1.5) to[out=-195,in=-90,looseness=1] (.7,-.6); % F0 -- F2

\draw [line width=1pt] (2,-1.5) to[out=0,in=-90,looseness=1] (4.3,-.7); % F0 -- F3
\draw [line width=1pt] (2,-1.5) to[out=0,in=-90,looseness=1] (4.2,-.5); % F0 -- F3
\draw [line width=1pt] (2,-1.5) to[out=0,in=-90,looseness=1] (3.9,-.4); % F0 -- F3
% e
\draw [line width=1pt] (1,0) to (3,0); 
\draw (2,0) node[anchor=south east] {$e$};


%e*
\draw [line width=1pt] (2,1.5) to (2,-1.5); 
%\draw (2.6,.5) node[anchor=south east] {$e^\star$};


\draw (2.1,.7) node[anchor=south east] {$e_3$};
\draw (2.1,-.9) node[anchor=east] {$e_1$};


\draw (2,1.5) node[anchor=south] {$F_e$};
\draw [fill=black] (2,1.5) circle (1.5pt);

\draw (2,-1.5) node[anchor=north] {$F'_e$};
\draw [fill=black] (2,-1.5) circle (1.5pt);

% Arcos
%\draw (2,-1.5) circle [radius=1cm];

\draw[red,line width=1pt,->] (2.85,1.2) arc (-35:-5:1);
\draw (3.1,2) node[red] {$\beta$};

\draw[red,line width=1pt,->] (1,1.7) arc (180:220:1);
\draw (1,1.9) node[red] {$\alpha$};

\draw[red,line width=1pt,->] (1.5,-.65) arc (120:200:1);
\draw (1.2,-.6) node[red] {$\delta$};

\draw[red,line width=1pt,<-] (2.5,-.65) arc (60:-20:1);
\draw (2.9,-.6) node[red] {$\gamma$};
\end{tikzpicture}

}
\end{figure}
\end{minipage}
\hfill
\begin{minipage}{.42\textwidth}
\begin{figure}[h!]
\scalebox{.85}{
\begin{tikzpicture}[line cap=round,line join=round,x=1cm,y=1cm]
\clip(-1,-2.3) rectangle (6,3);

\draw (0,0) circle [radius=1cm];
\draw (4,0) circle [radius=1cm];
\draw (0,0) node[anchor=north] {$G$};
\draw (4,.3) node[anchor=north] {$G$};


\draw [line width=1pt] (2,0) to[out=90,in=90,looseness=1] (-.5,.5); % F0 -- F2
\draw [line width=1pt] (2,0) to[out=100,in=90,looseness=1] (-.3,.3); % F0 -- F2
\draw [line width=1pt] (2,0) to[out=110,in=90,looseness=1] (0,0); % F0 -- F2
\draw [line width=1pt] (2,0) to[out=120,in=90,looseness=1] (.4,0); % F0 -- F2


\draw [line width=1pt] (2,0) to[out=90,in=90,looseness=1] (4.3,.7); % F0 -- F3
\draw [line width=1pt] (2,0) to[out=80,in=90,looseness=1] (4.2,.5); % F0 -- F3
\draw [line width=1pt] (2,0) to[out=70,in=90,looseness=1] (3.9,.4); % F0 -- F3


\draw [line width=1pt] (2,0) to[out=-90,in=-90,looseness=1] (-.7,-.5); % F0 -- F2
\draw [line width=1pt] (2,0) to[out=-100,in=-90,looseness=1] (-.3,-.5); % F0 -- F2
\draw [line width=1pt] (2,0) to[out=-110,in=-90,looseness=1] (.3,-.5); % F0 -- F2
\draw [line width=1pt] (2,0) to[out=-120,in=-90,looseness=1] (.7,-.3); % F0 -- F2

\draw [line width=1pt] (2,0) to[out=-90,in=-90,looseness=1] (4.3,-.7); % F0 -- F3
\draw [line width=1pt] (2,0) to[out=-90,in=-90,looseness=1] (4.2,-.5); % F0 -- F3
\draw [line width=1pt] (2,0) to[out=-90,in=-90,looseness=1] (3.9,-.4); % F0 -- F3

\draw (2,0) node[anchor=east] {$F_e$};
\draw [fill=black] (2,0) circle (1.5pt);

% Arcos
%\draw (2,0) circle [radius=1cm];

\draw[red,line width=1pt,<-] (2.33,.96) arc (70:30:1);
\draw (2.7,1.2) node[red] {$\beta$};

\draw[red,line width=1pt,->] (1.5,.85) arc (120:160:1);
\draw (1.4,1) node[red] {$\alpha$};

\draw[red,line width=1pt,->] (1.08,-.34) arc (200:250:1);
\draw (1.5,-1.2) node[red] {$\delta$};

\draw[red,line width=1pt,->] (2.5,-.86) arc (-60:-30:1);
\draw (2.7,-1.1) node[red] {$\gamma$};
\end{tikzpicture}

}
\end{figure}
\end{minipage}
\end{frame}

\begin{frame}{Remoção de não ponte - \MSFdelEdge($G$, $b$)}
\begin{figure}[H]
\scalebox{1.25}{
\begin{tikzpicture}[line cap=round,line join=round,>=triangle 45,x=1cm,y=1cm]
\clip(-1.2,-.3) rectangle (3,5.6);

\begin{scriptsize}

%
% Node v
%
%\draw (0,0) node {$\hat v$};
\draw [line width=1pt,color=cyan] (0,.5) to  (.35,.35); 
\draw [line width=1pt,color=cyan] (.35,.35) to  (.5,0); 

\draw [line width=1pt,color=cyan] (0,.5) to  (-1,1.8); 
\draw [line width=1pt,color=cyan] (.35,.35) to  (1.05,1.75); 
\draw [line width=1pt,color=cyan] (.5,0) to  (2.5,1); 


%\draw (0,.5)[anchor=east] node {\tiny $a_v$};
\draw (.35,.35)[anchor=west] node {\tiny $b_v$};
%\draw (.5,0)[anchor=north west] node {\tiny $d_v$};
\draw [fill=black] (0,.5) circle (1.5pt);
\draw [fill=black] (.35,.35) circle (1.5pt);
\draw [fill=black] (.5,0) circle (1.5pt);

\draw [line width=1pt,color=cyan] (-1,1.8) to  (-1,2.2); 
%\draw (-.6,1.9) node[anchor=north east] {\tiny $a_0$};
\draw [fill=white] (-1,1.8) circle (1.5pt);

% ^b
\draw [line width=1pt,color=cyan] (2.5,1) to (2.9,1.15); 
\draw (1,1.65)[anchor=west] node {\tiny $b_2$};
\draw [fill=white] (1.05,1.75) circle (1.5pt);
%\draw (2.5,1)[anchor=north] node {\tiny $d_0$};
\draw [fill=white] (2.5,1) circle (1.5pt);

%
% Node u
%
%\draw (0,4) node {$\hat u$};

\draw [line width=1pt,color=cyan] (0,3.5) to  (.3,3.7); 
\draw [line width=1pt,color=cyan] (.5,4) to  (.3,3.7); 

\draw [line width=1pt,color=cyan] (0,3.5) to  (-1,2.2); 
\draw [line width=1pt,color=cyan] (.3,3.7) to  (1.05,2.25); 
\draw [line width=1pt,color=cyan] (.5,4) to  (2.5,3); 

%\draw (-1.1,2.35) node {\tiny $a_2$};
\draw [fill=white] (-1,2.2) circle (1.5pt);
\draw (1.2,2.33) node {\tiny $b_0$};
\draw [fill=white] (1.05,2.25) circle (1.5pt);

%\draw (2.4,2.9) node {\tiny $c_2$};
\draw [fill=white] (2.5,3) circle (1.5pt);

% u node path
%\draw (-.2,3.5) node {\tiny $a_u$};
\draw [fill=black] (0,3.5) circle (1.5pt);
%\draw (.5,4.15) node {\tiny $c_u$};
\draw [fill=black] (.5,4) circle (1.5pt);
\draw (.19,3.82) node {\tiny $b_u$};
\draw [fill=black] (.3,3.7) circle (1.5pt);

%
% Node z
%
\draw (3.95,2.1) node {$\hat z$};
\draw [line width=1pt,color=cyan] (4.3,2) to (6.5,2); 
%\draw [line width=1pt,color=cyan] (3.7,2.3) to (3,2.7); 
\draw [line width=1pt,color=cyan] (3.7,1.7) to (2.9,1.15); 

% node path
\draw [line width=1pt,color=cyan] (3.7,2.3) to  (3.7,1.7); 
\draw [line width=1pt,color=cyan] (3.7,1.7) to  (4.3,2); 


%\draw (3.75,1.55) node {\tiny $d_z$};
\draw [fill=black] (3.7,1.7)circle (1.5pt);
%\draw (3.7,2.43) node {\tiny $c_z$};
\draw [fill=black] (3.7,2.3)circle (1.5pt);
\draw (4.3,2.2) node {\tiny $g_z$};
\draw [fill=black] (4.3,2)circle (1.5pt);

%\draw (3.2,2.7) node {\tiny $c_0$};
%\draw [fill=white] (3,2.7)circle (1.5pt);
%\draw (3.05,1.25)[anchor=north] node {\tiny $d_2$};
\draw [fill=white] (2.9,1.15)circle (1.5pt);

%eF0 -- e3
\draw [line width=1pt,color=red] (3.3,4.7) to  (5.25,2.25);


%
% Node F_0
%
%\draw (3,5.2) node {$\hat F_0$};
%\draw [line width=1pt,color=red] (2.7,4.7) to  (3.8,4.7); 
%\draw [line width=1pt,color=red] (3.8,4.7) to  (4,5); 
%\draw [line width=1pt,color=red] (4,5) to  (2.5,5); 

%\draw [line width=1pt,color=red] (2.5,5) to[in=180,out=180,looseness=2]  (-1.2,2); 
%\draw [line width=1pt,color=red] (4,5) to[in=-20,out=10,looseness=3]  (5.25,1.75); 
%\draw [line width=1pt,color=red] (3.5,5) to[in=-30,out=20,looseness=4.5]  (2.8,.85); 

%^c
%\draw [line width=1pt,color=red] (2.6,2.6) to  (2.9,3.1);
%\draw [line width=1pt,color=red] (2.7,4.7) to  (2.9,3.1);

%\draw (-1.3,1.9) node {\tiny $a_3$};
%\draw [fill=white] (-1.2,2)circle (1.5pt);
%\draw (3.07,3.1) node {\tiny $c_1$};
%\draw [fill=white] (2.9,3.1)circle (1.5pt);
%\draw (2.9,.7) node {\tiny $d_1$};
%\draw [fill=white] (2.8,.85)circle (1.5pt);


%fF3 -- 
%\draw [line width=1pt,color=red] (6.5,3.9) to  (6.5,3);

% ^f
%\draw [line width=1pt,color=cyan] (6.8,2.5) to[in=-60,out=60,looseness=1.5] (6.65,3.8);
%\draw [line width=1pt,color=cyan] (6.2,2.5) to[in=-120,out=120,looseness=1.5] (6.35,3.8);

\draw [line width=1pt,color=red] (3.8,4.7) to  (6.5,3.95);
%\draw (6.65,4.05) node {\tiny $f_0$};
%\draw [fill=white] (6.5,3.95)circle (1.5pt);
\draw (6.65,3.55) node {\tiny $f_2$};
%\draw [fill=white] (6.5,3.65)circle (1.5pt);
\draw (6.8,3.8) node {\tiny $f_3$};
%\draw [fill=white] (6.65,3.8)circle (1.5pt);
\draw (6.2,3.8) node {\tiny $f_1$};
%\draw [fill=white] (6.35,3.8)circle (1.5pt);


%\draw (2.5,5)[anchor=south] node {\tiny $a_{F_0}$};
%\draw [fill=black] (2.5,5)circle (1.5pt);
%\draw (2.5,4.6) node {\tiny $c_{F_0}$};
%\draw [fill=black] (2.7,4.7)circle (1.5pt);
%\draw (3.5,5)[anchor=south] node {\tiny $d_{F_0}$};
\draw [fill=black] (3.5,5)circle (1.5pt);
\draw (4,5)[anchor=north west] node {\tiny $g_{F_0}$};
\draw [fill=black] (4,5)circle (1.5pt);
\draw (3.8,4.7)[anchor=north] node {\tiny $f_{F_0}$};
\draw [fill=black] (3.8,4.7)circle (1.5pt);
\draw (3.3,4.7)[anchor=north] node {\tiny $g_{F_0}$};
\draw [fill=black] (3.3,4.7)circle (1.5pt);

%
% Node y
%
\draw (6.5,2.35) node {$\hat y$};

%node path
\draw [line width=1pt,color=cyan] (6.5,2) to  (6.2,2.5); 
\draw [line width=1pt,color=cyan] (6.5,2) to  (6.8,2.5); 


\draw (6.5,1.8) node {\tiny $g_y$};
\draw (6,2.5) node {\tiny $f_y$};
\draw (7,2.5) node {\tiny $f_y$};
\draw [fill=black] (6.5,2)circle (1.5pt);
\draw [fill=black] (6.2,2.5)circle (1.5pt);
\draw [fill=black] (6.8,2.5)circle (1.5pt);


%
% Node F_1
%
\draw (2.1,2) node {$\hat F_1$};
%node path
\draw [line width=1pt,color=red] (2.3,2.3) to  (2.3,1.7); 
\draw [line width=1pt,color=red] (1.7,2) to  (2.3,1.7); 

\draw [line width=1pt,color=red] (2.6,2.6) to  (2.3,2.3); 
\draw [line width=1pt,color=red] (2.6,1.3) to  (2.3,1.7); 
\draw [line width=1pt,color=red] (1.7,2) to  (1.3,2); 

% F_1 to F_2
%\draw [line width=1pt,color=red] (.8,2) to  (1.3,2); 

\draw (1.8,2.13) node {\tiny $b_{F_1}$};
\draw [fill=black] (1.7,2)circle (1.5pt);
\draw (2.3,1.7)[anchor=north] node {\tiny $d_{F_1}$};
\draw [fill=black] (2.3,1.7)circle (1.5pt);
\draw (2.25,2.2)[anchor=west] node {\tiny $c_{F_1}$};
\draw [fill=black] (2.3,2.3)circle (1.5pt);
\draw (1.4,1.85) node {\tiny $b_3$};
\draw [fill=white] (1.3,2)circle (1.5pt);

%\draw (2.8,1.4) node {\tiny $d_3$};
\draw [fill=white] (2.6,1.3)circle (1.5pt);
%\draw (2.75,2.55) node {\tiny $c_3$};
\draw [fill=white] (2.6,2.6)circle (1.5pt);

%
% Node F_2
%
\draw (0,2.2) node {$\hat F_2$};

\draw [line width=1pt,color=red] (-.8,2) to  (-.3,2); 
\draw [line width=1pt,color=red] (.8,2) to  (.3,2);   
\draw [line width=1pt,color=red] (-.3,2) to (.3,2);   

\draw [fill=black] (-.2,2) circle (1.5pt);
\draw (-.2,1.8) node {\tiny $a_{F_2}$};
\draw [fill=black] (.2,2) circle (1.5pt);
\draw (.3,1.8) node {\tiny $b_{F_2}$};
%\draw (-.75,2.1) node {\tiny $a_1$};
\draw [fill=white] (-.8,2) circle (1.5pt);
\draw (.8,2.16) node {\tiny $b_1$};
\draw [fill=white] (.8,2) circle (1.5pt);

%
% Node F_3
%
\draw (6.7,3.2) node {$\hat F_3$};
\draw (6.3,3) node {\tiny $f_{F_3}$};
\draw [fill=black] (6.5,3)circle (1.5pt);

\draw (5,2)[anchor=south] node {\tiny $g_0$};
\draw [fill=white] (5,2)circle (1.5pt);
\draw (5.5,2)[anchor=south] node {\tiny $g_2$};
\draw [fill=white] (5.5,2)circle (1.5pt);
\draw (5.3,2.25)[anchor=south] node {\tiny $g_3$};
\draw [fill=white] (5.25,2.25)circle (1.5pt);
\draw (5.25,1.75)[anchor=north] node {\tiny $g_1$};
\draw [fill=white] (5.25,1.75)circle (1.5pt);

\end{scriptsize}
\end{tikzpicture}

}
\end{figure}
\end{frame}
\begin{frame}{Remoção de não ponte - \MSFdelEdge($G$, $b$)}
\begin{figure}
\begin{tikzpicture}[line cap=round,line join=round,x=1cm,y=1cm]
\clip(-1.2,-.3) rectangle (3,5.6);

\begin{scriptsize}

%
% Node v
%
%\draw (0,0) node {$\hat v$};
\draw [line width=1pt,color=cyan] (0,.5) to  (.35,.35); 
\draw [line width=1pt,color=cyan] (.35,.35) to  (.5,0); 

\draw [line width=1pt,color=cyan] (0,.5) to  (-1,1.8); 
\draw [line width=1pt,color=cyan] (.35,.35) to  (1.05,.75); 
\draw [line width=1pt,color=cyan] (.5,0) to  (2.5,1); 


%\draw (0,.5)[anchor=east] node {\tiny $a_v$};
\draw (.35,.35)[anchor=west] node {\tiny $b_v$};
%\draw (.5,0)[anchor=north west] node {\tiny $d_v$};
\draw [fill=black] (0,.5) circle (1.5pt);
\draw [fill=black] (.35,.35) circle (1.5pt);
\draw [fill=black] (.5,0) circle (1.5pt);

\draw [line width=1pt,color=cyan] (-1,1.8) to  (-1,2.2); 
%\draw (-.6,1.9) node[anchor=north east] {\tiny $a_0$};
\draw [fill=white] (-1,1.8) circle (1.5pt);

% ^b
\draw [line width=1pt,color=cyan] (2.5,1) to (2.9,1.15); 
\draw (1,.65)[anchor=west] node {\tiny $b_2$};
\draw [fill=white] (1.05,.75) circle (1.5pt);
%\draw (2.5,1)[anchor=north] node {\tiny $d_0$};
\draw [fill=white] (2.5,1) circle (1.5pt);

%
% Node u
%
%\draw (0,4) node {$\hat u$};

\draw [line width=1pt,color=cyan] (0,3.5) to  (.3,3.7); 
\draw [line width=1pt,color=cyan] (.5,4) to  (.3,3.7); 

\draw [line width=1pt,color=cyan] (0,3.5) to  (-1,2.2); 
\draw [line width=1pt,color=cyan] (.3,3.7) to  (1.05,2.25); 
\draw [line width=1pt,color=cyan] (.5,4) to  (2.5,3); 

%\draw (-1.1,2.35) node {\tiny $a_2$};
\draw [fill=white] (-1,2.2) circle (1.5pt);
\draw (1.2,2.33) node {\tiny $b_0$};
\draw [fill=white] (1.05,2.25) circle (1.5pt);

%\draw (2.4,2.9) node {\tiny $c_2$};
\draw [fill=white] (2.5,3) circle (1.5pt);

% u node path
%\draw (-.2,3.5) node {\tiny $a_u$};
\draw [fill=black] (0,3.5) circle (1.5pt);
%\draw (.5,4.15) node {\tiny $c_u$};
\draw [fill=black] (.5,4) circle (1.5pt);
\draw (.19,3.82) node {\tiny $b_u$};
\draw [fill=black] (.3,3.7) circle (1.5pt);

%
% Node z
%
\draw (3.95,2.1) node {$\hat z$};
\draw [line width=1pt,color=cyan] (4.3,2) to (6.5,2); 
%\draw [line width=1pt,color=cyan] (3.7,2.3) to (3,2.7); 
\draw [line width=1pt,color=cyan] (3.7,1.7) to (2.9,1.15); 

% node path
\draw [line width=1pt,color=cyan] (3.7,2.3) to  (3.7,1.7); 
\draw [line width=1pt,color=cyan] (3.7,1.7) to  (4.3,2); 


%\draw (3.75,1.55) node {\tiny $d_z$};
\draw [fill=black] (3.7,1.7)circle (1.5pt);
%\draw (3.7,2.43) node {\tiny $c_z$};
\draw [fill=black] (3.7,2.3)circle (1.5pt);
\draw (4.3,2.2) node {\tiny $g_z$};
\draw [fill=black] (4.3,2)circle (1.5pt);

%\draw (3.2,2.7) node {\tiny $c_0$};
%\draw [fill=white] (3,2.7)circle (1.5pt);
%\draw (3.05,1.25)[anchor=north] node {\tiny $d_2$};
\draw [fill=white] (2.9,1.15)circle (1.5pt);

%eF0 -- e3
\draw [line width=1pt,color=red] (3.3,4.7) to  (5.25,2.25);


%
% Node F_0
%
%\draw (3,5.2) node {$\hat F_0$};
%\draw [line width=1pt,color=red] (2.7,4.7) to  (3.8,4.7); 
%\draw [line width=1pt,color=red] (3.8,4.7) to  (4,5); 
%\draw [line width=1pt,color=red] (4,5) to  (2.5,5); 

%\draw [line width=1pt,color=red] (2.5,5) to[in=180,out=180,looseness=2]  (-1.2,2); 
%\draw [line width=1pt,color=red] (4,5) to[in=-20,out=10,looseness=3]  (5.25,1.75); 
%\draw [line width=1pt,color=red] (3.5,5) to[in=-30,out=20,looseness=4.5]  (2.8,.85); 

%^c
%\draw [line width=1pt,color=red] (2.6,2.6) to  (2.9,3.1);
%\draw [line width=1pt,color=red] (2.7,4.7) to  (2.9,3.1);

%\draw (-1.3,1.9) node {\tiny $a_3$};
%\draw [fill=white] (-1.2,2)circle (1.5pt);
%\draw (3.07,3.1) node {\tiny $c_1$};
%\draw [fill=white] (2.9,3.1)circle (1.5pt);
%\draw (2.9,.7) node {\tiny $d_1$};
%\draw [fill=white] (2.8,.85)circle (1.5pt);


%fF3 -- 
%\draw [line width=1pt,color=red] (6.5,3.9) to  (6.5,3);

% ^f
%\draw [line width=1pt,color=cyan] (6.8,2.5) to[in=-60,out=60,looseness=1.5] (6.65,3.8);
%\draw [line width=1pt,color=cyan] (6.2,2.5) to[in=-120,out=120,looseness=1.5] (6.35,3.8);

\draw [line width=1pt,color=red] (3.8,4.7) to  (6.5,3.95);
%\draw (6.65,4.05) node {\tiny $f_0$};
%\draw [fill=white] (6.5,3.95)circle (1.5pt);
\draw (6.65,3.55) node {\tiny $f_2$};
%\draw [fill=white] (6.5,3.65)circle (1.5pt);
\draw (6.8,3.8) node {\tiny $f_3$};
%\draw [fill=white] (6.65,3.8)circle (1.5pt);
\draw (6.2,3.8) node {\tiny $f_1$};
%\draw [fill=white] (6.35,3.8)circle (1.5pt);


%\draw (2.5,5)[anchor=south] node {\tiny $a_{F_0}$};
%\draw [fill=black] (2.5,5)circle (1.5pt);
%\draw (2.5,4.6) node {\tiny $c_{F_0}$};
%\draw [fill=black] (2.7,4.7)circle (1.5pt);
%\draw (3.5,5)[anchor=south] node {\tiny $d_{F_0}$};
\draw [fill=black] (3.5,5)circle (1.5pt);
\draw (4,5)[anchor=north west] node {\tiny $g_{F_0}$};
\draw [fill=black] (4,5)circle (1.5pt);
\draw (3.8,4.7)[anchor=north] node {\tiny $f_{F_0}$};
\draw [fill=black] (3.8,4.7)circle (1.5pt);
\draw (3.3,4.7)[anchor=north] node {\tiny $g_{F_0}$};
\draw [fill=black] (3.3,4.7)circle (1.5pt);

%
% Node y
%
\draw (6.5,2.35) node {$\hat y$};

%node path
\draw [line width=1pt,color=cyan] (6.5,2) to  (6.2,2.5); 
\draw [line width=1pt,color=cyan] (6.5,2) to  (6.8,2.5); 


\draw (6.5,1.8) node {\tiny $g_y$};
\draw (6,2.5) node {\tiny $f_y$};
\draw (7,2.5) node {\tiny $f_y$};
\draw [fill=black] (6.5,2)circle (1.5pt);
\draw [fill=black] (6.2,2.5)circle (1.5pt);
\draw [fill=black] (6.8,2.5)circle (1.5pt);


\draw [line width=1pt,color=red] (.2,2) to[in=200,out=-30,looseness=.5]  (2.3,1.7); 

%
% Node F_1
%
%\draw (2.1,2) node {$\hat F_1$};
%node path
\draw [line width=1pt,color=red] (2.3,2.3) to  (2.3,1.7); 
\draw [line width=1pt,color=red] (1.7,2) to  (2.3,2.3); 

\draw [line width=1pt,color=red] (2.6,2.6) to  (2.3,2.3); 
\draw [line width=1pt,color=red] (2.6,1.3) to  (2.3,1.7); 
\draw [line width=1pt,color=red] (1.7,2) to  (1.3,2); 

% F_1 to F_2
%\draw [line width=1pt,color=red] (.8,2) to  (1.3,2); 

\draw (1.84,1.83) node {\tiny $b_{F_1}$};
\draw [fill=black] (1.7,2)circle (1.5pt);
\draw (2.3,1.7)[anchor=north] node {\tiny $d_{F_1}$};
\draw [fill=black] (2.3,1.7)circle (1.5pt);
\draw (2.25,2.2)[anchor=west] node {\tiny $c_{F_1}$};
\draw [fill=black] (2.3,2.3)circle (1.5pt);
\draw (1.4,1.85) node {\tiny $b_3$};
\draw [fill=white] (1.3,2)circle (1.5pt);

%\draw (2.8,1.4) node {\tiny $d_3$};
\draw [fill=white] (2.6,1.3)circle (1.5pt);
%\draw (2.75,2.55) node {\tiny $c_3$};
\draw [fill=white] (2.6,2.6)circle (1.5pt);

%
% Node F_2
%
%\draw (0,2.2) node {$\hat F_2$};

\draw [line width=1pt,color=red] (-.8,2) to  (-.3,2); 
\draw [line width=1pt,color=red] (.2,1.5) to  (.2,2);   
\draw [line width=1pt,color=red] (-.3,2) to (.2,2);   

\draw [fill=black] (-.2,2) circle (1.5pt);
\draw (-.2,1.8) node {\tiny $a_{F_2}$};
\draw [fill=black] (.2,2) circle (1.5pt);
\draw (.3,2.2) node {\tiny $b_{F_2}$};
%\draw (-.75,2.1) node {\tiny $a_1$};
\draw [fill=white] (-.8,2) circle (1.5pt);
\draw (.2,1.5) node[anchor=north] {\tiny $b_1$};
\draw [fill=white] (.2,1.5) circle (1.5pt);

%
% Node F_3
%
\draw (6.7,3.2) node {$\hat F_3$};
\draw (6.3,3) node {\tiny $f_{F_3}$};
\draw [fill=black] (6.5,3)circle (1.5pt);

\draw (5,2)[anchor=south] node {\tiny $g_0$};
\draw [fill=white] (5,2)circle (1.5pt);
\draw (5.5,2)[anchor=south] node {\tiny $g_2$};
\draw [fill=white] (5.5,2)circle (1.5pt);
\draw (5.3,2.25)[anchor=south] node {\tiny $g_3$};
\draw [fill=white] (5.25,2.25)circle (1.5pt);
\draw (5.25,1.75)[anchor=north] node {\tiny $g_1$};
\draw [fill=white] (5.25,1.75)circle (1.5pt);

\end{scriptsize}
\end{tikzpicture}

\end{figure}
\end{frame}

\begin{frame}{Remoção de não ponte - \MSFdelEdge($G$, $b$)}
\begin{figure}
\begin{tikzpicture}[line cap=round,line join=round,x=1cm,y=1cm]
\clip(-1.2,-.3) rectangle (3,5.6);

\begin{scriptsize}

%
% Node v
%
%\draw (0,0) node {$\hat v$};
\draw [line width=1pt,color=cyan] (0,.5) to  (.5,0); 
%\draw [line width=1pt,color=cyan] (.35,.35) to  (.5,0); 

\draw [line width=1pt,color=cyan] (0,.5) to  (-1,1.8); 
%\draw [line width=1pt,color=cyan] (.35,.35) to  (1.05,1.75); 
\draw [line width=1pt,color=cyan] (.5,0) to  (2.5,1); 


\draw (0,.5)[anchor=east] node {\tiny $a_v$};
%\draw (.35,.35)[anchor=west] node {\tiny $b_v$};
\draw (.5,0)[anchor=north west] node {\tiny $d_v$};
\draw [fill=black] (0,.5) circle (1.5pt);
%\draw [fill=black] (.35,.35) circle (1.5pt);
\draw [fill=black] (.5,0) circle (1.5pt);

\draw [line width=1pt,color=cyan] (-1,1.8) to  (-1,2.2); 
%\draw (-.6,1.9) node[anchor=north east] {\tiny $a_0$};
\draw [fill=white] (-1,1.8) circle (1.5pt);

% ^b
\draw [line width=1pt,color=cyan] (2.5,1) to (2.9,1.15); 
%\draw (1,1.65)[anchor=west] node {\tiny $b_2$};
%\draw [fill=white] (1.05,1.75) circle (1.5pt);
%\draw (2.5,1)[anchor=north] node {\tiny $d_0$};
\draw [fill=white] (2.5,1) circle (1.5pt);

%
% Node u
%
%\draw (0,4) node {$\hat u$};

\draw [line width=1pt,color=cyan] (0,3.5) to  (.5,4); 
%\draw [line width=1pt,color=cyan] (.5,4) to  (.3,3.7); 

\draw [line width=1pt,color=cyan] (0,3.5) to  (-1,2.2); 
%\draw [line width=1pt,color=cyan] (.3,3.7) to  (1.05,2.25); 
\draw [line width=1pt,color=cyan] (.5,4) to  (2.5,3); 

%\draw (-1.1,2.35) node {\tiny $a_2$};
\draw [fill=white] (-1,2.2) circle (1.5pt);
%\draw (1.2,2.33) node {\tiny $b_0$};
%\draw [fill=white] (1.05,2.25) circle (1.5pt);

%\draw (2.4,2.9) node {\tiny $c_2$};
\draw [fill=white] (2.5,3) circle (1.5pt);

% u node path
\draw (-.2,3.5) node {\tiny $a_u$};
\draw [fill=black] (0,3.5) circle (1.5pt);
\draw (.5,4.15) node {\tiny $c_u$};
\draw [fill=black] (.5,4) circle (1.5pt);
%\draw (.19,3.82) node {\tiny $b_u$};
%\draw [fill=black] (.3,3.7) circle (1.5pt);

%
% Node z
%
\draw (3.95,2.1) node {$\hat z$};
\draw [line width=1pt,color=cyan] (4.3,2) to (6.5,2); 
%\draw [line width=1pt,color=cyan] (3.7,2.3) to (3,2.7); 
\draw [line width=1pt,color=cyan] (3.7,1.7) to (2.9,1.15); 

% node path
\draw [line width=1pt,color=cyan] (3.7,2.3) to  (3.7,1.7); 
\draw [line width=1pt,color=cyan] (3.7,1.7) to  (4.3,2); 


%\draw (3.75,1.55) node {\tiny $d_z$};
\draw [fill=black] (3.7,1.7)circle (1.5pt);
%\draw (3.7,2.43) node {\tiny $c_z$};
\draw [fill=black] (3.7,2.3)circle (1.5pt);
\draw (4.3,2.2) node {\tiny $g_z$};
\draw [fill=black] (4.3,2)circle (1.5pt);

%\draw (3.2,2.7) node {\tiny $c_0$};
%\draw [fill=white] (3,2.7)circle (1.5pt);
%\draw (3.05,1.25)[anchor=north] node {\tiny $d_2$};
\draw [fill=white] (2.9,1.15)circle (1.5pt);

%eF0 -- e3
\draw [line width=1pt,color=red] (3.3,4.7) to  (5.25,2.25);


%
% Node F_0
%
%\draw (3,5.2) node {$\hat F_0$};
%\draw [line width=1pt,color=red] (2.7,4.7) to  (3.8,4.7); 
%\draw [line width=1pt,color=red] (3.8,4.7) to  (4,5); 
%\draw [line width=1pt,color=red] (4,5) to  (2.5,5); 

%\draw [line width=1pt,color=red] (2.5,5) to[in=180,out=180,looseness=2]  (-1.2,2); 
%\draw [line width=1pt,color=red] (4,5) to[in=-20,out=10,looseness=3]  (5.25,1.75); 
%\draw [line width=1pt,color=red] (3.5,5) to[in=-30,out=20,looseness=4.5]  (2.8,.85); 

%^c
%\draw [line width=1pt,color=red] (2.6,2.6) to  (2.9,3.1);
%\draw [line width=1pt,color=red] (2.7,4.7) to  (2.9,3.1);

%\draw (-1.3,1.9) node {\tiny $a_3$};
%\draw [fill=white] (-1.2,2)circle (1.5pt);
%\draw (3.07,3.1) node {\tiny $c_1$};
%\draw [fill=white] (2.9,3.1)circle (1.5pt);
%\draw (2.9,.7) node {\tiny $d_1$};
%\draw [fill=white] (2.8,.85)circle (1.5pt);


%fF3 -- 
%\draw [line width=1pt,color=red] (6.5,3.9) to  (6.5,3);

% ^f
%\draw [line width=1pt,color=cyan] (6.8,2.5) to[in=-60,out=60,looseness=1.5] (6.65,3.8);
%\draw [line width=1pt,color=cyan] (6.2,2.5) to[in=-120,out=120,looseness=1.5] (6.35,3.8);

\draw [line width=1pt,color=red] (3.8,4.7) to  (6.5,3.95);
%\draw (6.65,4.05) node {\tiny $f_0$};
%\draw [fill=white] (6.5,3.95)circle (1.5pt);
\draw (6.65,3.55) node {\tiny $f_2$};
%\draw [fill=white] (6.5,3.65)circle (1.5pt);
\draw (6.8,3.8) node {\tiny $f_3$};
%\draw [fill=white] (6.65,3.8)circle (1.5pt);
\draw (6.2,3.8) node {\tiny $f_1$};
%\draw [fill=white] (6.35,3.8)circle (1.5pt);


%\draw (2.5,5)[anchor=south] node {\tiny $a_{F_0}$};
%\draw [fill=black] (2.5,5)circle (1.5pt);
%\draw (2.5,4.6) node {\tiny $c_{F_0}$};
%\draw [fill=black] (2.7,4.7)circle (1.5pt);
%\draw (3.5,5)[anchor=south] node {\tiny $d_{F_0}$};
\draw [fill=black] (3.5,5)circle (1.5pt);
\draw (4,5)[anchor=north west] node {\tiny $g_{F_0}$};
\draw [fill=black] (4,5)circle (1.5pt);
\draw (3.8,4.7)[anchor=north] node {\tiny $f_{F_0}$};
\draw [fill=black] (3.8,4.7)circle (1.5pt);
\draw (3.3,4.7)[anchor=north] node {\tiny $g_{F_0}$};
\draw [fill=black] (3.3,4.7)circle (1.5pt);

%
% Node y
%
\draw (6.5,2.35) node {$\hat y$};

%node path
\draw [line width=1pt,color=cyan] (6.5,2) to  (6.2,2.5); 
\draw [line width=1pt,color=cyan] (6.5,2) to  (6.8,2.5); 


\draw (6.5,1.8) node {\tiny $g_y$};
\draw (6,2.5) node {\tiny $f_y$};
\draw (7,2.5) node {\tiny $f_y$};
\draw [fill=black] (6.5,2)circle (1.5pt);
\draw [fill=black] (6.2,2.5)circle (1.5pt);
\draw [fill=black] (6.8,2.5)circle (1.5pt);


%
% Node F_1
%
%\draw (2.1,2) node {$\hat F_1$};
%node path
\draw [line width=1pt,color=red] (2.3,2.3) to  (2.3,1.7); 
%\draw [line width=1pt,color=red] (1.7,2) to  (2.3,1.7); 

\draw [line width=1pt,color=red] (2.6,2.6) to  (2.3,2.3); 
\draw [line width=1pt,color=red] (2.6,1.3) to  (2.3,1.7); 
%\draw [line width=1pt,color=red] (1.7,2) to  (1.3,2); 

% F_1 to F_2
%\draw [line width=1pt,color=red] (.8,2) to  (1.3,2); 

%\draw (1.8,2.13) node {\tiny $b_{F_1}$};
%\draw [fill=black] (1.7,2)circle (1.5pt);
\draw (2.25,2.2)[anchor=west] node {\tiny $c_{F_1}$};
\draw [fill=black] (2.3,2.3)circle (1.5pt);
%\draw (1.4,1.85) node {\tiny $b_3$};
%\draw [fill=white] (1.3,2)circle (1.5pt);

%\draw (2.8,1.4) node {\tiny $d_3$};
\draw [fill=white] (2.6,1.3)circle (1.5pt);
%\draw (2.75,2.55) node {\tiny $c_3$};
\draw [fill=white] (2.6,2.6)circle (1.5pt);

%
% Node F_2
%
%\draw (0,2.2) node {$\hat F_2$};

\draw [line width=1pt,color=red] (-.8,2) to  (-.2,2); 
\draw [line width=1pt,color=red] (-.2,2) to  (2.3,1.7);   
%\draw [line width=1pt,color=red] (-.3,2) to (.2,2);   
\draw (2.3,1.7)[anchor=north] node {\tiny $d_{F_1}$};
\draw [fill=black] (2.3,1.7)circle (1.5pt);

\draw [fill=black] (-.2,2) circle (1.5pt);
\draw (-.2,1.8) node {\tiny $a_{F_2}$};
%\draw [fill=black] (.2,2) circle (1.5pt);
%\draw (.3,1.8) node {\tiny $b_{F_2}$};
%\draw (-.75,2.1) node {\tiny $a_1$};
\draw [fill=white] (-.8,2) circle (1.5pt);
%\draw (.8,2.16) node {\tiny $b_1$};
%\draw [fill=white] (.8,2) circle (1.5pt);

%
% Node F_3
%
\draw (6.7,3.2) node {$\hat F_3$};
\draw (6.3,3) node {\tiny $f_{F_3}$};
\draw [fill=black] (6.5,3)circle (1.5pt);

\draw (5,2)[anchor=south] node {\tiny $g_0$};
\draw [fill=white] (5,2)circle (1.5pt);
\draw (5.5,2)[anchor=south] node {\tiny $g_2$};
\draw [fill=white] (5.5,2)circle (1.5pt);
\draw (5.3,2.25)[anchor=south] node {\tiny $g_3$};
\draw [fill=white] (5.25,2.25)circle (1.5pt);
\draw (5.25,1.75)[anchor=north] node {\tiny $g_1$};
\draw [fill=white] (5.25,1.75)circle (1.5pt);

\end{scriptsize}
\end{tikzpicture}

\end{figure}
\end{frame}


\begin{frame}{Dois casos de adição de aresta}
\begin{minipage}{0.4\textwidth}
\begin{figure}
\scalebox{0.7}{
\begin{tikzpicture}[line cap=round,line join=round,x=1cm,y=1cm]
\clip(-1,-1) rectangle (6,3.5);


\draw [line width=1pt] (0,0) to[in=-135,out=135,looseness=1]  (0,2); % v -- u
\draw [line width=1pt] (0,0) to[in=-45,out=45,looseness=1]  (0,2);   % v -- u
\draw [line width=1pt] (2,1) to  (0,0); % 4 -- 6
\draw [line width=1pt] (2,1) to  (0,2); % 4 -- 6
%\draw [line width=1pt] (2,1) to  (3,1); % 4 -- 6
\draw [line width=1pt] (3,1.05) to[out=45,in=-45,looseness=50] (3,.95); % 4 -- 4


\draw [line width=1pt] (1,2.5) to[out=-45,in=45,looseness=1] (1,1); % c: F0 -- F1
\draw [line width=1pt] (1,2.5) to[out=0,in=90,looseness=1] (2.5,1); % d: F0 -- F1
\draw [line width=1pt] (1,1) to[out=-55,in=-90,looseness=2] (2.5,1); % d: F0 -- F1


% laço F0 para F0
%\draw [line width=1pt] (1,2.5) to[out=-30,in=90,looseness=1] (2.5,1); % F0 -- F1
%\draw [line width=1pt] (2.5,1) to[out=-90,in=-90,looseness=2] (4.5,1); % F0 -- F1
%\draw [line width=1pt] (1,2.5) to[out=0,in=90,looseness=1] (4.5,1); % F0 -- F1


\draw [line width=1pt] (1,2.5) to[out=170,in=180,looseness=2.5] (0,1); % F0 -- F2
\draw [line width=1pt] (3.5,2) to (3.5,1); % F0 -- F3
\draw [line width=1pt] (1,1) to (0,1); % F1 -- F2



\draw (2.5,3) node {\Large ANTES};

\begin{scriptsize}
\draw [fill=black] (0,2) circle (1.5pt);
%\draw (0,2) node[anchor=south east] {\Large $u$};


\draw [fill=black] (0,0) circle (1.5pt);
\draw (0,0) node[anchor=north east] {\Large $v$};
\draw [fill=black] (2,1) circle (1.5pt);
%\draw (1.95,1) node[anchor=south] {\Large $z$};
\draw [fill=black] (3,1) circle (1.5pt);
\draw (2.95,1) node[anchor=south] {\Large $y$};


\draw [fill=black] (1,2.5) circle (1.5pt);
\draw (1,2.5) node[anchor=south east] {\Large $F_4$};
\draw [fill=black] (1,1) circle (1.5pt);
%\draw (1,1) node[anchor=west] {\Large $F_1$};
\draw [fill=black] (0,1) circle (1.5pt);
%\draw (0,1) node[anchor=south] {\Large $F_2$};
\draw [fill=black] (3.5,1) circle (1.5pt);
%\draw (3.5,1) node[anchor=west] {\Large $F_3$};
\draw [fill=black] (3.5,2) circle (1.5pt);
\draw (3.5,2) node[anchor=west] {\Large $F_5$};

%\draw (-0.5,1.05) node[anchor=north] {\Large $a$};
%\draw (-0.5,1.1) node[anchor=north] {\Large $2$};

%\draw (0.5,1.35) node[anchor=north] {\Large $b$};
%\draw (0.5,1.05) node[anchor=north] {\Large $7$};

%\draw (1.15,1.7) node[anchor=north] {\Large $c$}; % u -- w
%\draw (1.4,1.7) node[anchor=north] {\Large $3$}; % u -- w

%\draw (1.1,.6) node[anchor=north] {\Large $d$}; % v -- w
%\draw (1.2,.65) node[anchor=north] {\Large $1$}; % v -- w

%\draw (2.3,1.3) node[anchor=north] {\Large $e$}; % v -- w
%\draw (2.3,.95) node[anchor=north] {\Large $4$}; % v -- w

%\draw (4.15,1) node {\Large $f$}; % w -- w
%\draw (4.15,1) node {\Large $2$}; % w -- w

\end{scriptsize}
\end{tikzpicture}

}
\end{figure}
\end{minipage}
\begin{minipage}{0.4\textwidth}
\begin{figure}
\scalebox{0.7}{
\begin{tikzpicture}[line cap=round,line join=round,x=1cm,y=1cm]
\clip(-3,-1) rectangle (6,3);


\draw [line width=1pt] (0,0) to[in=-135,out=135,looseness=1]  (0,2); % v -- u
\draw [line width=1pt] (0,0) to[in=-45,out=45,looseness=1]  (0,2);   % v -- u
\draw [line width=1pt] (2,1) to  (0,0); % 4 -- 6
\draw [line width=1pt] (2,1) to  (0,2); % 4 -- 6
%\draw [line width=1pt] (2,1) to  (3,1); % 4 -- 6
\draw [line width=1pt] (3,1.05) to[out=45,in=-45,looseness=50] (3,.95); % 4 -- 4


\draw [line width=1pt] (1,2.5) to[out=-45,in=45,looseness=1] (1,1); % c: F0 -- F1
\draw [line width=1pt] (1,2.5) to[out=0,in=90,looseness=1] (2.3,1.5); % d: F0 -- F1
\draw [line width=1pt] (1,1) to[out=-55,in=-90,looseness=2] (2.3,1.5); % d: F0 -- F1


% laço F0 para F0
%\draw [line width=1pt] (1,2.5) to[out=-30,in=90,looseness=1] (2.5,1); % F0 -- F1
%\draw [line width=1pt] (2.5,1) to[out=-90,in=-90,looseness=2] (4.5,1); % F0 -- F1
%\draw [line width=1pt] (1,2.5) to[out=0,in=90,looseness=1] (4.5,1); % F0 -- F1


\draw [line width=1pt] (1,2.5) to[out=170,in=180,looseness=2.5] (0,1); % F0 -- F2
\draw [line width=1pt] (1,2.5) to[out=30,in=90,looseness=1] (3.5,1); % F0 -- F3
\draw [line width=1pt] (1,1) to (0,1); % F1 -- F2


\draw [line width=1pt] (1,2.5) to[out=70,in=90,looseness=.8] (4.7,1); % F0 -- F3
\draw [line width=1pt] (4.7,1) to[out=-90,in=-90,looseness=2] (2.5,1); % F0 -- F3
\draw [line width=1pt] (1,2.5) to[out=20,in=90,looseness=1.3] (2.5,1); % F0 -- F3

\draw [line width=1pt] (0,0) to[out=0,in=-90,looseness=1] (3,1); % F0 -- F1


\begin{scriptsize}
\draw [fill=black] (0,2) circle (1.5pt);
\draw (0,2) node[anchor=south east] {$u$};


\draw [fill=black] (0,0) circle (1.5pt);
\draw (0,0) node[anchor=north east] {$v$};
\draw [fill=black] (2,1) circle (1.5pt);
\draw (1.95,1) node[anchor=south] {$z$};
\draw [fill=black] (3,1) circle (1.5pt);
\draw (2.95,1) node[anchor=south] {$y$};


\draw [fill=black] (1,2.5) circle (1.5pt);
\draw (1,2.5) node[anchor=south east] {$F_4$};
\draw [fill=black] (1,1) circle (1.5pt);
\draw (1,1) node[anchor=west] {$F_1$};
\draw [fill=black] (0,1) circle (1.5pt);
\draw (0,1) node[anchor=south] {$F_2$};
\draw [fill=black] (3.5,1) circle (1.5pt);
\draw (3.5,1) node[anchor=west] {$F_3$};
%\draw [fill=black] (3.5,2) circle (1.5pt);
%\draw (3.5,2) node[anchor=west] {$F_5$};

\draw (-0.5,1.05) node[anchor=north] {$a$};
%\draw (-0.5,1.1) node[anchor=north] {$2$};

\draw (0.5,1.35) node[anchor=north] {$b$};
%\draw (0.5,1.05) node[anchor=north] {$7$};

\draw (1.15,1.7) node[anchor=north] {$c$}; % u -- w
%\draw (1.4,1.7) node[anchor=north] {$3$}; % u -- w

\draw (1.1,.6) node[anchor=north] {$d$}; % v -- w
%\draw (1.2,.65) node[anchor=north] {$1$}; % v -- w

%\draw (2.3,1.3) node[anchor=north] {$e$}; % v -- w
%\draw (2.3,.95) node[anchor=north] {$4$}; % v -- w

\draw (4.15,1) node {$f$}; % w -- w
%\draw (4.15,1) node {$2$}; % w -- w

\draw (3,.2) node {$h$}; % w -- w
\end{scriptsize}
\end{tikzpicture}

}
\end{figure}
\end{minipage}
\begin{minipage}{0.4\textwidth}
\begin{figure}
\scalebox{0.7}{
\begin{tikzpicture}[line cap=round,line join=round,x=1cm,y=1cm]
\clip(-1,-1) rectangle (6,3);


\draw [line width=1pt] (0,0) to[in=-135,out=135,looseness=1]  (0,2); % v -- u
\draw [line width=1pt] (0,0) to[in=-45,out=45,looseness=1]  (0,2);   % v -- u
\draw [line width=1pt] (2,1) to  (0,0); % 4 -- 6
\draw [line width=1pt] (2,1) to  (0,2); % 4 -- 6
\draw [line width=1pt] (2,1) to  (3,1); % 4 -- 6
\draw [line width=1pt] (3,1.05) to[out=45,in=-45,looseness=50] (3,.95); % 4 -- 4


\draw [line width=1pt] (1,2.5) to[out=-45,in=45,looseness=1] (1,1); % c: F0 -- F1
\draw [line width=1pt] (1,2.5) to[out=20,in=160,looseness=1] (4.2,2.5); % d: F0 -- F1
\draw [line width=1pt] (4.2,2.5) to[out=-20,in=20,looseness=1] (4.2,-.5); % d: F0 -- F1
\draw [line width=1pt] (1,1) to[out=-70,in=200,looseness=1] (4.2,-.5); % d: F0 -- F1


% laço F0 para F0
\draw [line width=1pt] (1,2.5) to[out=-30,in=90,looseness=1] (2.5,1); % F0 -- F1
\draw [line width=1pt] (2.5,1) to[out=-90,in=-90,looseness=2] (4.5,1); % F0 -- F1
\draw [line width=1pt] (1,2.5) to[out=0,in=90,looseness=1] (4.5,1); % F0 -- F1


\draw [line width=1pt] (1,2.5) to[out=170,in=180,looseness=2.5] (0,1); % F0 -- F2
\draw [line width=1pt] (1,2.5) to[out=0,in=90,looseness=1] (3.5,1); % F0 -- F3
\draw [line width=1pt] (1,1) to (0,1); % F1 -- F2




\begin{scriptsize}
\draw [fill=black] (0,2) circle (1.5pt);
%\draw (0,2) node[anchor=south east] {{\Large $u$}};


\draw [fill=black] (0,0) circle (1.5pt);
\draw (0,0) node[anchor=north east] {{\Large $v$}};
\draw [fill=black] (2,1) circle (1.5pt);
%\draw (2,1.1) node[anchor=south] {{\Large $z$}};
\draw [fill=black] (3,1) circle (1.5pt);
\draw (2.95,1) node[anchor=south] {{\Large $y$}};


\draw [fill=black] (1,2.5) circle (1.5pt);
\draw (1,2.5) node[anchor=south east] {{\Large $F_0$}};
\draw [fill=black] (1,1) circle (1.5pt);
%\draw (1.1,1) node[anchor=west] {{\Large $F_1$}};
\draw [fill=black] (0,1) circle (1.5pt);
%\draw (0,1) node[anchor=south] {{\Large $F_2$}};
\draw [fill=black] (3.5,1) circle (1.5pt);
%\draw (3.5,1) node[anchor=west] {{\Large $F_3$}};

%\draw (-.6,1.05) node[anchor=north] {{\Large $a$}};
%\draw (-0.5,1.1) node[anchor=north] {{\Large $2$}};

%\draw (0.7,1.4) node[anchor=north] {{\Large $b$}};
%\draw (0.5,1.05) node[anchor=north] {{\Large $7$}};

%\draw (1.15,1.85) node[anchor=north] {{\Large $c$}}; % u -- w
%\draw (1.4,1.7) node[anchor=north] {{\Large $3$}}; % u -- w

\draw (1.1,.5) node[anchor=north] {{\Large $d$}}; % v -- w
%\draw (1.2,.65) node[anchor=north] {{\Large $1$}}; % v -- w

\draw (2.3,1) node[anchor=north] {{\Large $g$}}; % v -- w
%\draw (2.3,.95) node[anchor=north] {{\Large $4$}}; % v -- w

%\draw (4.15,1) node {{\Large $f$}}; % w -- w
%\draw (4.15,1) node {{\Large $2$}}; % w -- w

\end{scriptsize}
\end{tikzpicture}

}
\end{figure}
\end{minipage}
\begin{minipage}{0.4\textwidth}
\begin{figure}
\scalebox{0.7}{
\begin{tikzpicture}[line cap=round,line join=round,x=1cm,y=1cm]
\clip(-3,-1) rectangle (4.6,3);


\draw [line width=1pt] (0,0) to[in=-135,out=135,looseness=1]  (0,2); % v -- u
\draw [line width=1pt] (0,0) to[in=-45,out=45,looseness=1]  (0,2);   % v -- u
\draw [line width=1pt] (2,1) to  (0,0); % 4 -- 6
\draw [line width=1pt] (2,1) to  (0,2); % 4 -- 6
\draw [line width=1pt] (2,1) to  (3,1); % 4 -- 6
\draw [line width=1pt] (3,1.05) to[out=45,in=-45,looseness=50] (3,.95); % 4 -- 4


\draw [line width=1pt] (1,2.5) to[out=-45,in=45,looseness=1] (1,1); % c: F0 -- F1
%\draw [line width=1pt] (1,2.5) to[out=20,in=160,looseness=1] (4.2,2.5); % d: F0 -- F1
%\draw [line width=1pt] (4.2,2.5) to[out=-20,in=20,looseness=1] (4.2,-.5); % d: F0 -- F1
%\draw [line width=1pt] (1,1) to[out=-70,in=200,looseness=1] (4.2,-.5); % d: F0 -- F1


% laço F0 para F0
\draw [line width=1pt] (1,2.5) to[out=-30,in=90,looseness=1] (2.5,1); % F0 -- F1
%\draw [line width=1pt] (2.5,1) to[out=-90,in=-90,looseness=2] (4.5,1); % F0 -- F1
\draw [line width=1pt] (1,2.5) to[out=0,in=90,looseness=1] (4.5,1); % F0 -- F1
\draw [line width=1pt] (1.5,.5) to[out=-90,in=-90,looseness=1.5] (4.5,1); % F0 -- F1


% a
\draw [line width=1pt] (1,2.5) to[out=170,in=180,looseness=2.5] (0,1); % F0 -- F2
\draw [line width=1pt] (1,2.5) to[out=0,in=90,looseness=1] (3.5,1); % F0 -- F3
\draw [line width=1pt] (1,1) to (0,1); % F1 -- F2


\draw [line width=1pt,color=red] (0,0) to[out=0,in=220,looseness=1] (3,1); % F1 -- F2
\draw [line width=1pt] (2.5,1) to[out=-90,in=0,looseness=1] (1.5,.5); % F0 -- F1


\draw [line width=1pt] (1,1) to (1.5,.5); % F1 -- F2


\begin{scriptsize}
\draw [fill=black] (0,2) circle (1.5pt);
%\draw (0,2) node[anchor=south east] {\Large $u$};


\draw [fill=black] (0,0) circle (1.5pt);
\draw (0,0) node[anchor=north east] {\Large $v$};
\draw [fill=black] (2,1) circle (1.5pt);
%\draw (1.95,1) node[anchor=south] {\Large $z$};
\draw [fill=black] (3,1) circle (1.5pt);
\draw (2.95,1) node[anchor=south] {\Large $y$};


\draw [fill=black] (1,2.5) circle (1.5pt);
\draw (1,2.5) node[anchor=south east] {\Large $F_0$};
\draw [fill=black] (1,1) circle (1.5pt);
%\draw (1,1) node[anchor=west] {\Large $F_1$};
\draw [fill=black] (0,1) circle (1.5pt);
%\draw (0,1) node[anchor=south] {\Large $F_2$};
\draw [fill=black] (3.5,1) circle (1.5pt);
%\draw (3.5,1) node[anchor=west] {\Large $F_3$};
\draw [fill=black] (1.5,.5) circle (1.5pt);

%\draw (-0.5,1.05) node[anchor=north] {\Large $a$};
%\draw (-0.5,1.1) node[anchor=north] {\Large $2$};

%\draw (0.5,1.35) node[anchor=north] {\Large $b$};
%\draw (0.5,1.05) node[anchor=north] {\Large $7$};

%\draw (1.15,1.7) node[anchor=north] {\Large $c$}; % u -- w
%\draw (1.4,1.7) node[anchor=north] {\Large $3$}; % u -- w

\draw (1.1,.6) node[anchor=north] {\Large $d$}; % v -- w
%\draw (1.2,.65) node[anchor=north] {\Large $1$}; % v -- w

\draw (2.3,1.4) node[anchor=north] {\Large $g$}; % v -- w
%\draw (2.3,.95) node[anchor=north] {\Large $4$}; % v -- w

%\draw (4.15,1) node {\Large $f$}; % w -- w
%\draw (4.15,1) node {\Large $2$}; % w -- w

\draw (2.35,.5) node[anchor=north] {\Large $h$}; % v -- w
\end{scriptsize}
\end{tikzpicture}

}
\end{figure}
\end{minipage}
\end{frame}


\begin{frame}{Adição de aresta - \MSFaddEdge($G$, $h$, $v$, $a$, $y$, $f$, $7$)}
\begin{figure}
\scalebox{1}{
\begin{tikzpicture}[line cap=round,line join=round,x=1cm,y=1cm]
\clip(-2.8,-.3) rectangle (7.3,5.6);

\begin{scriptsize}

%
% Node v
%
\draw (0,0) node {$\hat v$};
\draw [line width=1pt,color=cyan] (0,.5) to  (.35,.35); 
\draw [line width=1pt,color=cyan] (.35,.35) to  (.5,0); 

\draw [line width=1pt,color=cyan] (0,.5) to  (-1,1.8); 
\draw [line width=1pt,color=cyan] (.35,.35) to  (1.05,1.75); 
\draw [line width=1pt,color=cyan] (.5,0) to  (2.5,1); 


\draw (0,.5)[anchor=east] node {\tiny $a_v$};
\draw (.35,.35)[anchor=west] node {\tiny $b_v$};
\draw (.5,0)[anchor=north west] node {\tiny $d_v$};
\draw [fill=black] (0,.5) circle (1.5pt);
\draw [fill=black] (.35,.35) circle (1.5pt);
\draw [fill=black] (.5,0) circle (1.5pt);

\draw (-.6,1.9) node[anchor=north east] {\tiny $a_0$};
\draw [fill=white] (-1,1.8) circle (1.5pt);

% ^b
\draw [line width=1pt,color=cyan] (2.5,1) to (2.9,1.15); 
\draw (1,1.65)[anchor=west] node {\tiny $b_2$};
\draw [fill=white] (1.05,1.75) circle (1.5pt);
%\draw (2.5,1)[anchor=north] node {\tiny $d_0$};
\draw [fill=white] (2.5,1) circle (1.5pt);

%
% Node u
%
\draw (0,4) node {$\hat u$};

\draw [line width=1pt,color=cyan] (0,3.5) to  (.3,3.7); 
\draw [line width=1pt,color=cyan] (.5,4) to  (.3,3.7); 

\draw [line width=1pt,color=cyan] (0,3.5) to  (-1,2.2); 
\draw [line width=1pt,color=cyan] (.3,3.7) to  (1.05,2.25); 
\draw [line width=1pt,color=cyan] (.5,4) to  (2.5,3); 

\draw (-1.1,2.35) node {\tiny $a_2$};
\draw [fill=white] (-1,2.2) circle (1.5pt);
\draw (1.2,2.33) node {\tiny $b_0$};
\draw [fill=white] (1.05,2.25) circle (1.5pt);

% u node path
\draw (-.2,3.5) node {\tiny $a_u$};
\draw [fill=black] (0,3.5) circle (1.5pt);
\draw (.5,4.15) node {\tiny $c_u$};
\draw [fill=black] (.5,4) circle (1.5pt);
\draw (.19,3.82) node {\tiny $b_u$};
\draw [fill=black] (.3,3.7) circle (1.5pt);

%
% Node z
%
\draw (3.8,2) node {$\hat z$};
%\draw [line width=1pt,color=cyan] (4.3,2) to (6.5,2); 
\draw [line width=1pt,color=cyan] (3.7,2.3) to (2.5,3); 
\draw [line width=1pt,color=cyan] (3.7,1.7) to (2.9,1.15); 
%\draw (2.4,2.9) node {\tiny $c_2$};
\draw [fill=white] (2.5,3) circle (1.5pt);


% node path
\draw [line width=1pt,color=cyan] (3.7,2.3) to  (3.7,1.7); 
%\draw [line width=1pt,color=cyan] (3.7,1.7) to  (4.3,2); 


\draw (3.75,1.55) node {\tiny $d_z$};
\draw [fill=black] (3.7,1.7)circle (1.5pt);
\draw (3.7,2.43) node {\tiny $c_z$};
\draw [fill=black] (3.7,2.3)circle (1.5pt);
%\draw (4.3,2.2) node {\tiny $g_z$};
%\draw [fill=black] (4.3,2)circle (1.5pt);

%\draw (3.2,2.7) node {\tiny $c_0$};
\draw [fill=white] (3,2.7)circle (1.5pt);
%\draw (3.05,1.25)[anchor=north] node {\tiny $d_2$};
\draw [fill=white] (2.9,1.15)circle (1.5pt);

%eF0 -- e3
%\draw [line width=1pt,color=red] (3.3,4.7) to  (5.25,2.25);


%
% Node F_0
%
\draw (3,5.2) node {$\hat F_0$};
%\draw [line width=1pt,color=red] (2.7,4.7) to  (3.8,4.7); 
\draw [line width=1pt,color=red] (2.5,5) to  (2.7,4.7); 
\draw [line width=1pt,color=red] (3.5,5) to  (2.5,5); 

\draw [line width=1pt,color=red] (2.5,5) to[in=180,out=180,looseness=2]  (-1.2,2); 
%\draw [line width=1pt,color=red] (4,5) to[in=-20,out=10,looseness=3]  (5.25,1.75); 
\draw [line width=1pt,color=red] (3.5,5) to[in=-30,out=20,looseness=2]  (2.8,.85); 

%^c
\draw [line width=1pt,color=red] (2.7,4.7) to  (2.9,3.1);

%\draw (3.07,3.1) node {\tiny $c_1$};
\draw [fill=white] (2.9,3.1)circle (1.5pt);
%\draw (2.9,.7) node {\tiny $d_1$};
\draw [fill=white] (2.8,.85)circle (1.5pt);


%fF3 -- 
\draw [line width=1pt,color=red] (6.5,3.9) to  (6.5,3);

% ^f
\draw [line width=1pt,color=cyan] (6.8,2.5) to[in=-60,out=60,looseness=1.5] (6.65,3.8);
\draw [line width=1pt,color=cyan] (6.2,2.5) to[in=-120,out=120,looseness=1.5] (6.35,3.8);

\draw [line width=1pt,color=red] (6.5,4.7) to  (6.5,3.95);
\draw (6.65,4.05) node {\tiny $f_1$};
\draw [fill=white] (6.5,3.95)circle (1.5pt);
\draw (6.65,3.55) node {\tiny $f_3$};
\draw [fill=white] (6.5,3.65)circle (1.5pt);
\draw (6.8,3.8) node {\tiny $f_0$};
\draw [fill=white] (6.65,3.8)circle (1.5pt);
\draw (6.2,3.8) node {\tiny $f_2$};
\draw [fill=white] (6.35,3.8)circle (1.5pt);


\draw (2.5,5)[anchor=south] node {\tiny $a_{F_0}$};
\draw [fill=black] (2.5,5)circle (1.5pt);
\draw (2.5,4.6) node {\tiny $c_{F_0}$};
\draw [fill=black] (2.7,4.7)circle (1.5pt);
\draw (3.5,5)[anchor=south] node {\tiny $d_{F_0}$};
\draw [fill=black] (3.5,5)circle (1.5pt);
%\draw (4,5)[anchor=north west] node {\tiny $g_{F_0}$};
%\draw [fill=black] (4,5)circle (1.5pt);
\draw (6.5,4.7)[anchor=south] node {\tiny $f_{F_4}$};
\draw [fill=black] (6.5,4.7)circle (1.5pt);
%\draw (3.3,4.7)[anchor=north] node {\tiny $g_{F_0}$};
%\draw [fill=black] (3.3,4.7)circle (1.5pt);

%
% Node y
%
\draw (6.5,2.3) node {$\hat y$};

%node path
\draw [line width=1pt,color=cyan] (6.8,2.5) to  (6.2,2.5); 
%\draw [line width=1pt,color=cyan] (6.5,2) to  (6.8,2.5); 


%\draw (6.5,1.8) node {\tiny $g_y$};
\draw (6,2.5) node {\tiny $f_y$};
\draw (7,2.5) node {\tiny $f_y$};
%\draw [fill=black] (6.5,2)circle (1.5pt);
\draw [fill=black] (6.2,2.5)circle (1.5pt);
\draw [fill=black] (6.8,2.5)circle (1.5pt);


%
% Node F_1
%
%\draw (2.1,2) node {$\hat F_1$};
%node path
\draw [line width=1pt,color=red] (2.3,2.3) to  (2.3,1.7); 
\draw [line width=1pt,color=red] (1.7,2) to  (2.3,1.7); 

\draw [line width=1pt,color=red] (2.6,2.6) to  (2.3,2.3); 
\draw [line width=1pt,color=red] (2.6,1.3) to  (2.3,1.7); 
\draw [line width=1pt,color=red] (1.7,2) to  (1.3,2); 

% F_1 to F_2
\draw [line width=1pt,color=red] (.8,2) to  (1.3,2); 

\draw (1.8,2.13) node {\tiny $b_{F_1}$};
\draw [fill=black] (1.7,2)circle (1.5pt);
%\draw (2.3,1.7)[anchor=north] node {\tiny $d_{F_1}$};
\draw [fill=black] (2.3,1.7)circle (1.5pt);
%\draw (2.25,2.2)[anchor=west] node {\tiny $c_{F_1}$};
\draw [fill=black] (2.3,2.3)circle (1.5pt);
\draw (1.4,1.85) node {\tiny $b_3$};
\draw [fill=white] (1.3,2)circle (1.5pt);

%\draw (2.8,1.4) node {\tiny $d_3$};
\draw [fill=white] (2.6,1.3)circle (1.5pt);
%\draw (2.75,2.55) node {\tiny $c_3$};
\draw [fill=white] (2.6,2.6)circle (1.5pt);

%
% Node F_2
%
\draw (0,2.2) node {$\hat F_2$};

\draw [line width=1pt,color=red] (-1.2,2) to  (-.3,2); 
\draw [line width=1pt,color=red] (.8,2) to  (.3,2);   
\draw [line width=1pt,color=red] (-.3,2) to (.3,2);   

\draw [fill=black] (-.2,2) circle (1.5pt);
\draw (-.2,1.8) node {\tiny $a_{F_2}$};
\draw [fill=black] (.2,2) circle (1.5pt);
\draw (.3,1.8) node {\tiny $b_{F_2}$};
\draw (-.75,2.1) node {\tiny $a_1$};
\draw [fill=white] (-.8,2) circle (1.5pt);
\draw (.8,2.16) node {\tiny $b_1$};
\draw [fill=white] (.8,2) circle (1.5pt);

\draw (-1.3,1.9) node {\tiny $a_3$};
\draw [fill=white] (-1.2,2)circle (1.5pt);

%
% Node F_3
%
%\draw (6.7,3.2) node {$\hat F_3$};
\draw (6.3,3) node {\tiny $f_{F_3}$};
\draw [fill=black] (6.5,3)circle (1.5pt);

%\draw (5,2)[anchor=south] node {\tiny $g_0$};
%\draw [fill=white] (5,2)circle (1.5pt);
%\draw (5.5,2)[anchor=south] node {\tiny $g_2$};
%\draw [fill=white] (5.5,2)circle (1.5pt);
%\draw (5.25,2.25)[anchor=south] node {\tiny $g_3$};
%\draw [fill=white] (5.25,2.25)circle (1.5pt);
%\draw (5.25,1.75)[anchor=north] node {\tiny $g_1$};
%\draw [fill=white] (5.25,1.75)circle (1.5pt);

\end{scriptsize}
\end{tikzpicture}

}
\end{figure}
\end{frame}

\begin{frame}{Adição de aresta - \MSFaddEdge($G$, $h$, $v$, $a$, $y$, $f$, $7$)}
\begin{figure}[htb]
\scalebox{1}{
\begin{tikzpicture}[line cap=round,line join=round,x=1cm,y=1cm]
\clip(-2.8,-1) rectangle (8,6);

\begin{scriptsize}

%
% Node v
%
%\draw (0,0) node {$\hat v$};
\draw [line width=1pt,color=cyan] (0,.5) to  (.35,.35); 
\draw [line width=1pt,color=cyan] (.35,.35) to  (.5,0); 
\draw [line width=1pt,color=cyan] (0,.5) to  (.5,-.5); 

\draw [line width=1pt,color=cyan] (0,.5) to  (-1,1.8); 
\draw [line width=1pt,color=cyan] (.35,.35) to  (1.05,1.75); 
\draw [line width=1pt,color=cyan] (.5,0) to  (2.5,1); 
\draw [line width=1pt,color=cyan] (.5,-.5) to  (2,-.5); 


\draw (0,.5)[anchor=east] node {\normalsize $a_v$};
\draw (.35,.35)[anchor=west] node {\normalsize $b_v$};
\draw (.5,0)[anchor=north west] node {\normalsize $d_v$};
\draw (.5,-.5)[anchor=east] node {\normalsize $h_v$};
\draw [fill=black] (0,.5) circle (1.5pt);
\draw [fill=black] (.35,.35) circle (1.5pt);
\draw [fill=black] (.5,0) circle (1.5pt);
\draw [fill=black] (.5,-.5) circle (1.5pt);

\draw (-.6,1.9) node[anchor=north east] {\normalsize $a_0$};
\draw [fill=white] (-1,1.8) circle (1.5pt);

\draw (2,-.5)[anchor=south] node {\normalsize $h_0$};
\draw (2,-.5)[fill=white] circle (1.5pt);
% ^b
\draw [line width=1pt,color=cyan] (2.5,1) to (2.9,1.15); 
%\draw (1,1.65)[anchor=west] node {\normalsize $b_2$};
\draw [fill=white] (1.05,1.75) circle (1.5pt);
%\draw (2.5,1)[anchor=north] node {\normalsize $d_0$};
\draw [fill=white] (2.5,1) circle (1.5pt);

%
% Node u
%
%\draw (0,4) node {$\hat u$};

\draw [line width=1pt,color=cyan] (0,3.5) to  (.3,3.7); 
\draw [line width=1pt,color=cyan] (.5,4) to  (.3,3.7); 

\draw [line width=1pt,color=cyan] (0,3.5) to  (-1,2.2); 
\draw [line width=1pt,color=cyan] (.3,3.7) to  (1.05,2.25); 
\draw [line width=1pt,color=cyan] (.5,4) to  (2.5,3); 

\draw (-1.1,2.35) node {\normalsize $a_2$};
\draw [fill=white] (-1,2.2) circle (1.5pt);
%\draw (1.2,2.33) node {\normalsize $b_0$};
\draw [fill=white] (1.05,2.25) circle (1.5pt);

% u node path
%\draw (-.2,3.5) node {\normalsize $a_u$};
\draw [fill=black] (0,3.5) circle (1.5pt);
%\draw (.5,4.15) node {\normalsize $c_u$};
\draw [fill=black] (.5,4) circle (1.5pt);
%\draw (.19,3.82) node {\normalsize $b_u$};
\draw [fill=black] (.3,3.7) circle (1.5pt);

%
% Node z
%
%\draw (3.8,2) node {$\hat z$};
%\draw [line width=1pt,color=cyan] (4.3,2) to (6.5,2); 
\draw [line width=1pt,color=cyan] (3.7,2.3) to (2.5,3); 
\draw [line width=1pt,color=cyan] (3.7,1.7) to (2.9,1.15); 
%\draw (2.4,2.9) node {\normalsize $c_2$};
\draw [fill=white] (2.5,3) circle (1.5pt);


% node path
\draw [line width=1pt,color=cyan] (3.7,2.3) to  (3.7,1.7); 
%\draw [line width=1pt,color=cyan] (3.7,1.7) to  (4.3,2); 


%\draw (3.75,1.55) node {\normalsize $d_z$};
\draw [fill=black] (3.7,1.7)circle (1.5pt);
%\draw (3.7,2.43) node {\normalsize $c_z$};
\draw [fill=black] (3.7,2.3)circle (1.5pt);
%\draw (4.3,2.2) node {\normalsize $g_z$};
%\draw [fill=black] (4.3,2)circle (1.5pt);

%\draw (3.2,2.7) node {\normalsize $c_0$};
\draw [fill=white] (3,2.7)circle (1.5pt);
%\draw (3.05,1.25)[anchor=north] node {\normalsize $d_2$};
\draw [fill=white] (2.9,1.15)circle (1.5pt);

%eF0 -- e3
%\draw [line width=1pt,color=red] (3.3,4.7) to  (5.25,2.25);


%
% Node F_0
%
%\draw (3,5.2) node {$\hat F_0$};
%\draw [line width=1pt,color=red] (2.7,4.7) to  (3.8,4.7); 
\draw [line width=1pt,color=red] (2.5,5) to  (2.7,4.7); 
\draw [line width=1pt,color=red] (3.5,5) to  (2.7,4.7); 

\draw [line width=1pt,color=red] (2.5,5) to[in=180,out=180,looseness=2]  (-1.2,2); 
%\draw [line width=1pt,color=red] (4,5) to[in=-20,out=10,looseness=3]  (5.25,1.75); 
\draw [line width=1pt,color=red] (3.5,5) to[in=-30,out=20,looseness=2]  (2.8,.85); 

%^c
\draw [line width=1pt,color=red] (2.7,4.7) to  (2.9,3.1);

%\draw (3.07,3.1) node {\normalsize $c_1$};
\draw [fill=white] (2.9,3.1)circle (1.5pt);
%\draw (2.9,.7) node {\normalsize $d_1$};
\draw [fill=white] (2.8,.85)circle (1.5pt);


%fF3 -- 
\draw [line width=1pt,color=red] (6.5,4.7) to  (7,4.7);
\draw [line width=1pt,color=red] (6.5,3.9) to  (6.5,3);
\draw [line width=1pt,color=red] (3.5,5.5) to  (4.5,5.5);

% ^f
\draw [line width=1pt,color=cyan] (6.8,2.5) to[in=-60,out=60,looseness=1.5] (6.65,3.8);
\draw [line width=1pt,color=cyan] (6.2,2.5) to[in=-120,out=120,looseness=1.5] (6.35,3.8);
\draw [line width=1pt,color=red] (6.5,4.7) to  (6.5,3.95);
\draw [line width=1pt,color=red] (3.5,5) to  (3.5,5.5);
\draw [line width=1pt,color=red] (7.5,4) to  (7,4.7);

%\draw (6.65,4.05) node {\normalsize $f_3$};
\draw [fill=white] (6.5,3.95)circle (1.5pt);
%\draw (6.65,3.55) node {\normalsize $f_1$};
\draw [fill=white] (6.5,3.65)circle (1.5pt);
%\draw (6.8,3.8) node {\normalsize $f_2$};
\draw [fill=white] (6.65,3.8)circle (1.5pt);
%\draw (6.2,3.8) node {\normalsize $f_0$};
\draw [fill=white] (6.35,3.8)circle (1.5pt);
\draw (4.7,5.5) node {\normalsize $h_3$};
\draw [fill=white] (4.5,5.5)circle (1.5pt);



\draw (2.5,5)[anchor=south] node {\normalsize $a_{F_0}$};
\draw [fill=black] (2.5,5)circle (1.5pt);
\draw (2.5,4.5) node {\normalsize $c_{F_0}$};
\draw [fill=black] (2.7,4.7)circle (1.5pt);
\draw (3.5,5)[anchor=north] node {\normalsize $d_{F_0}$};
\draw [fill=black] (3.5,5)circle (1.5pt);
\draw (3.5,5.5)[anchor=east] node {\normalsize $h_{F_0}$};
\draw [fill=black] (3.5,5.5)circle (1.5pt);
\draw (6.5,4.7)[anchor=south] node {\normalsize $f_{F_4}$};
\draw [fill=black] (6.5,4.7)circle (1.5pt);
%\draw (3.3,4.7)[anchor=north] node {\normalsize $g_{F_0}$};
%\draw [fill=black] (3.3,4.7)circle (1.5pt);
\draw (7,4.7)[anchor=south] node {\normalsize $h_{F_4}$};
\draw [fill=black] (7,4.7)circle (1.5pt);
\draw (7.5,4)[anchor=north] node {\normalsize $h_1$};
\draw [fill=white] (7.5,4)circle (1.5pt);

%
% Node y
%
%\draw (6.5,2.3) node {$\hat y$};

%node path
\draw [line width=1pt,color=cyan] (6.8,2.5) to  (6.2,2.5); 
\draw [line width=1pt,color=cyan] (6.5,2) to  (6.2,2.5); 

\draw [line width=1pt,color=cyan] (6.5,2) to  (6.5,1); 

\draw (6.7,2) node {\normalsize $h_y$};
\draw (6,2.5) node {\normalsize $f_y$};
\draw (7,2.5) node {\normalsize $f_y$};
\draw [fill=black] (6.5,2)circle (1.5pt);
\draw [fill=black] (6.2,2.5)circle (1.5pt);
\draw [fill=black] (6.8,2.5)circle (1.5pt);


\draw (6.5,.8) node {\normalsize $h_2$};
\draw [fill=white] (6.5,1)circle (1.5pt);
%
% Node F_1
%
%\draw (2.1,2) node {$\hat F_1$};
%node path
\draw [line width=1pt,color=red] (2.3,2.3) to  (2.3,1.7); 
\draw [line width=1pt,color=red] (1.7,2) to  (2.3,1.7); 

\draw [line width=1pt,color=red] (2.6,2.6) to  (2.3,2.3); 
\draw [line width=1pt,color=red] (2.6,1.3) to  (2.3,1.7); 
\draw [line width=1pt,color=red] (1.7,2) to  (1.3,2); 

% F_1 to F_2
\draw [line width=1pt,color=red] (.8,2) to  (1.3,2); 

%\draw (1.8,2.13) node {\normalsize $b_{F_1}$};
\draw [fill=black] (1.7,2)circle (1.5pt);
%\draw (2.3,1.7)[anchor=north] node {\normalsize $d_{F_1}$};
\draw [fill=black] (2.3,1.7)circle (1.5pt);
%\draw (2.25,2.2)[anchor=west] node {\normalsize $c_{F_1}$};
\draw [fill=black] (2.3,2.3)circle (1.5pt);
%\draw (1.4,1.85) node {\normalsize $b_3$};
\draw [fill=white] (1.3,2)circle (1.5pt);

%\draw (2.8,1.4) node {\normalsize $d_3$};
\draw [fill=white] (2.6,1.3)circle (1.5pt);
%\draw (2.75,2.55) node {\normalsize $c_3$};
\draw [fill=white] (2.6,2.6)circle (1.5pt);

%
% Node F_2
%
%\draw (0,2.2) node {$\hat F_2$};

\draw [line width=1pt,color=red] (-1.2,2) to  (-.3,2); 
\draw [line width=1pt,color=red] (.8,2) to  (.3,2);   
\draw [line width=1pt,color=red] (-.3,2) to (.3,2);   

\draw [fill=black] (-.2,2) circle (1.5pt);
%\draw (-.2,1.8) node {\normalsize $a_{F_2}$};
\draw [fill=black] (.2,2) circle (1.5pt);
%\draw (.3,1.8) node {\normalsize $b_{F_2}$};
\draw (-.75,2.1) node {\normalsize $a_1$};
\draw [fill=white] (-.8,2) circle (1.5pt);
%\draw (.8,2.16) node {\normalsize $b_1$};
\draw [fill=white] (.8,2) circle (1.5pt);

\draw (-1.3,1.9) node {\normalsize $a_3$};
\draw [fill=white] (-1.2,2)circle (1.5pt);

%
% Node F_3
%
%\draw (6.7,3.2) node {$\hat F_3$};
%\draw (6.3,3) node {\normalsize $f_{F_3}$};
\draw [fill=black] (6.5,3)circle (1.5pt);

%\draw (5,2)[anchor=south] node {\normalsize $g_0$};
%\draw [fill=white] (5,2)circle (1.5pt);
%\draw (5.5,2)[anchor=south] node {\normalsize $g_2$};
%\draw [fill=white] (5.5,2)circle (1.5pt);
%\draw (5.25,2.25)[anchor=south] node {\normalsize $g_3$};
%\draw [fill=white] (5.25,2.25)circle (1.5pt);
%\draw (5.25,1.75)[anchor=north] node {\normalsize $g_1$};
%\draw [fill=white] (5.25,1.75)circle (1.5pt);

\draw [fill=white] (-1.2,2)circle (1.5pt);

\end{scriptsize}
\end{tikzpicture}

	}
\end{figure}
\end{frame}

\begin{frame}{Adição de aresta - \MSFaddEdge($G$, $h$, $v$, $a$, $y$, $f$, $7$)}
\begin{figure}[htb]
\scalebox{1}{
\begin{tikzpicture}[line cap=round,line join=round,x=1cm,y=1cm]
\clip(-2.8,-1) rectangle (8,5.6);

\begin{scriptsize}

%
% Node v
%
%\draw (0,0) node {$\hat v$};
\draw [line width=1pt,color=cyan] (0,.5) to  (.35,.35); 
\draw [line width=1pt,color=cyan] (.35,.35) to  (.5,0); 
\draw [line width=1pt,color=cyan] (0,.5) to  (.5,-.5); 

\draw [line width=1pt,color=cyan] (0,.5) to  (-1,1.8); 
\draw [line width=1pt,color=cyan] (.35,.35) to  (1.05,1.75); 
\draw [line width=1pt,color=cyan] (.5,0) to  (2.5,1); 
\draw [line width=1pt,color=cyan] (.5,-.5) to  (5,.5); 


\draw (0,.5)[anchor=east] node {\normalsize $a_v$};
\draw (.35,.35)[anchor=west] node {\normalsize $b_v$};
\draw (.5,0)[anchor=north west] node {\normalsize $d_v$};
\draw (.5,-.5)[anchor=east] node {\normalsize $h_v$};
\draw [fill=black] (0,.5) circle (1.5pt);
\draw [fill=black] (.35,.35) circle (1.5pt);
\draw [fill=black] (.5,0) circle (1.5pt);
\draw [fill=black] (.5,-.5) circle (1.5pt);

%\draw (-.6,1.9) node[anchor=north east] {\normalsize $a_0$};
\draw [fill=white] (-1,1.8) circle (1.5pt);

%h 
\draw [line width=1pt,color=cyan] (5.5,1) to  (5,.5); 
\draw (5,0)[anchor=south] node {\normalsize $h_0$};
\draw (5,.5)[fill=white] circle (1.5pt);
% ^b
\draw [line width=1pt,color=cyan] (2.5,1) to (2.9,1.15); 
%\draw (1,1.65)[anchor=west] node {\normalsize $b_2$};
\draw [fill=white] (1.05,1.75) circle (1.5pt);
%\draw (2.5,1)[anchor=north] node {\normalsize $d_0$};
\draw [fill=white] (2.5,1) circle (1.5pt);

%
% Node u
%
%\draw (0,4) node {$\hat u$};

\draw [line width=1pt,color=cyan] (0,3.5) to  (.3,3.7); 
\draw [line width=1pt,color=cyan] (.5,4) to  (.3,3.7); 

\draw [line width=1pt,color=cyan] (0,3.5) to  (-1,2.2); 
\draw [line width=1pt,color=cyan] (.3,3.7) to  (1.05,2.25); 
\draw [line width=1pt,color=cyan] (.5,4) to  (2.5,3); 

%\draw (-1.1,2.35) node {\normalsize $a_2$};
\draw [fill=white] (-1,2.2) circle (1.5pt);
%\draw (1.2,2.33) node {\normalsize $b_0$};
\draw [fill=white] (1.05,2.25) circle (1.5pt);

% u node path
\draw (-.25,3.5) node {\normalsize $a_u$};
\draw [fill=black] (0,3.5) circle (1.5pt);
\draw (.5,4.15) node {\normalsize $c_u$};
\draw [fill=black] (.5,4) circle (1.5pt);
\draw (.17,3.87) node {\normalsize $b_u$};
\draw [fill=black] (.3,3.7) circle (1.5pt);

%
% Node z
%
%\draw (3.8,2) node {$\hat z$};
%\draw [line width=1pt,color=cyan] (4.3,2) to (6.5,2); 
\draw [line width=1pt,color=cyan] (3.7,2.3) to (2.5,3); 
\draw [line width=1pt,color=cyan] (3.7,1.7) to (2.9,1.15); 
%\draw (2.4,2.9) node {\normalsize $c_2$};
\draw [fill=white] (2.5,3) circle (1.5pt);


% node path
\draw [line width=1pt,color=cyan] (3.7,2.3) to  (3.7,1.7); 
%\draw [line width=1pt,color=cyan] (3.7,1.7) to  (4.3,2); 


%\draw (3.75,1.55) node {\normalsize $d_z$};
\draw [fill=black] (3.7,1.7)circle (1.5pt);
%\draw (3.7,2.43) node {\normalsize $c_z$};
\draw [fill=black] (3.7,2.3)circle (1.5pt);
%\draw (4.3,2.2) node {\normalsize $g_z$};
%\draw [fill=black] (4.3,2)circle (1.5pt);

%\draw (3.2,2.7) node {\normalsize $c_0$};
\draw [fill=white] (3,2.7)circle (1.5pt);
%\draw (3.05,1.25)[anchor=north] node {\normalsize $d_2$};
\draw [fill=white] (2.9,1.15)circle (1.5pt);

%eF0 -- e3
%\draw [line width=1pt,color=red] (3.3,4.7) to  (5.25,2.25);


%
% Node F_0
%
%\draw (3,5.2) node {$\hat F_0$};
%\draw [line width=1pt,color=red] (2.7,4.7) to  (3.8,4.7); 
\draw [line width=1pt,color=red] (2.5,5) to  (2.7,4.7); 
\draw [line width=1pt,color=red] (3.5,5) to  (2.7,4.7); 

\draw [line width=1pt,color=red] (2.5,5) to[in=180,out=180,looseness=2]  (-1.2,2); 
%\draw [line width=1pt,color=red] (4,5) to[in=-20,out=10,looseness=3]  (5.25,1.75); 
\draw [line width=1pt,color=red] (3.5,5) to[in=-0,out=-60,looseness=1.4]  (2.8,.85); 

%^c
\draw [line width=1pt,color=red] (2.7,4.7) to  (2.9,3.1);

%\draw (3.07,3.1) node {\normalsize $c_1$};
\draw [fill=white] (2.9,3.1)circle (1.5pt);
%\draw (2.9,.7) node {\normalsize $d_1$};
\draw [fill=white] (2.8,.85)circle (1.5pt);


%fF3 -- 
\draw [line width=1pt,color=red] (6.5,4.7) to  (7,4.7);
\draw [line width=1pt,color=red] (6.5,3.9) to  (6.5,3);
\draw [line width=1pt,color=red] (4.5,5) to  (5,1);

% ^f
\draw [line width=1pt,color=cyan] (6.8,2.5) to[in=-60,out=60,looseness=1.5] (6.65,3.8);
\draw [line width=1pt,color=cyan] (6.2,2.5) to[in=-120,out=120,looseness=1.5] (6.35,3.8);
\draw [line width=1pt,color=red] (6.5,4.7) to  (6.5,3.95);
\draw [line width=1pt,color=red] (3.5,5) to  (4.5,5);
\draw [line width=1pt,color=red] (5.5,.5) to[in=0,out=-20,looseness=1]  (7,4.7);

\draw [line width=1pt,color=red] (6.5,4.7) to  (4.5,5);

%\draw (6.65,4.05) node {\normalsize $f_3$};
\draw [fill=white] (6.5,3.95)circle (1.5pt);
%\draw (6.65,3.55) node {\normalsize $f_1$};
\draw [fill=white] (6.5,3.65)circle (1.5pt);
%\draw (6.8,3.8) node {\normalsize $f_2$};
\draw [fill=white] (6.65,3.8)circle (1.5pt);
%\draw (6.2,3.8) node {\normalsize $f_0$};
\draw [fill=white] (6.35,3.8)circle (1.5pt);
\draw (4.8,1) node {\normalsize $h_3$};
\draw [fill=white] (5,1)circle (1.5pt);



\draw (2.5,5)[anchor=south] node {\normalsize $a_{F_0}$};
\draw [fill=black] (2.5,5)circle (1.5pt);
\draw (2.5,4.5) node {\normalsize $c_{F_0}$};
\draw [fill=black] (2.7,4.7)circle (1.5pt);
\draw (3.5,5)[anchor=south] node {\normalsize $d_{F_0}$};
\draw [fill=black] (3.5,5)circle (1.5pt);
\draw (4.5,5)[anchor=south] node {\normalsize $h_{F_0}$};
\draw [fill=black] (4.5,5)circle (1.5pt);
\draw (6.5,4.7)[anchor=south] node {\normalsize $f_{F_4}$};
\draw [fill=black] (6.5,4.7)circle (1.5pt);
%\draw (3.3,4.7)[anchor=north] node {\normalsize $g_{F_0}$};
%\draw [fill=black] (3.3,4.7)circle (1.5pt);
\draw (7,4.7)[anchor=south] node {\normalsize $h_{F_4}$};
\draw [fill=black] (7,4.7)circle (1.5pt);
\draw (5.5,.5)[anchor=north] node {\normalsize $h_1$};
\draw [fill=white] (5.5,.5)circle (1.5pt);

%
% Node y
%
%\draw (6.5,2.3) node {$\hat y$};

%node path
\draw [line width=1pt,color=cyan] (6.8,2.5) to  (6.2,2.5); 
\draw [line width=1pt,color=cyan] (6.5,2) to  (6.2,2.5); 

\draw [line width=1pt,color=cyan] (6.5,2) to  (5.5,1); 


\draw (6.7,2) node {\normalsize $h_y$};
\draw (6,2.5) node {\normalsize $f_y$};
\draw (7,2.5) node {\normalsize $f_y$};
\draw [fill=black] (6.5,2)circle (1.5pt);
\draw [fill=black] (6.2,2.5)circle (1.5pt);
\draw [fill=black] (6.8,2.5)circle (1.5pt);


\draw (5.45,1.2) node {\normalsize $h_2$};
\draw [fill=white] (5.5,1)circle (1.5pt);
%
% Node F_1
%
%\draw (2.1,2) node {$\hat F_1$};
%node path
\draw [line width=1pt,color=red] (2.3,2.3) to  (2.3,1.7); 
\draw [line width=1pt,color=red] (1.7,2) to  (2.3,1.7); 

\draw [line width=1pt,color=red] (2.6,2.6) to  (2.3,2.3); 
\draw [line width=1pt,color=red] (2.6,1.3) to  (2.3,1.7); 
\draw [line width=1pt,color=red] (1.7,2) to  (1.3,2); 

% F_1 to F_2
\draw [line width=1pt,color=red] (.8,2) to  (1.3,2); 

%\draw (1.8,2.13) node {\normalsize $b_{F_1}$};
\draw [fill=black] (1.7,2)circle (1.5pt);
%\draw (2.3,1.7)[anchor=north] node {\normalsize $d_{F_1}$};
\draw [fill=black] (2.3,1.7)circle (1.5pt);
%\draw (2.25,2.2)[anchor=west] node {\normalsize $c_{F_1}$};
\draw [fill=black] (2.3,2.3)circle (1.5pt);
%\draw (1.4,1.85) node {\normalsize $b_3$};
\draw [fill=white] (1.3,2)circle (1.5pt);

%\draw (2.8,1.4) node {\normalsize $d_3$};
\draw [fill=white] (2.6,1.3)circle (1.5pt);
%\draw (2.75,2.55) node {\normalsize $c_3$};
\draw [fill=white] (2.6,2.6)circle (1.5pt);

%
% Node F_2
%
%\draw (0,2.2) node {$\hat F_2$};

\draw [line width=1pt,color=red] (-1.2,2) to  (-.3,2); 
\draw [line width=1pt,color=red] (.8,2) to  (.3,2);   
\draw [line width=1pt,color=red] (-.3,2) to (.3,2);   

\draw [fill=black] (-.2,2) circle (1.5pt);
%\draw (-.2,1.8) node {\normalsize $a_{F_2}$};
\draw [fill=black] (.2,2) circle (1.5pt);
%\draw (.3,1.8) node {\normalsize $b_{F_2}$};
%\draw (-.75,2.1) node {\normalsize $a_1$};
\draw [fill=white] (-.8,2) circle (1.5pt);
%\draw (.8,2.16) node {\normalsize $b_1$};
\draw [fill=white] (.8,2) circle (1.5pt);

%\draw (-1.3,1.9) node {\normalsize $a_3$};
\draw [fill=white] (-1.2,2)circle (1.5pt);

%
% Node F_3
%
%\draw (6.7,3.2) node {$\hat F_3$};
%\draw (6.3,3) node {\normalsize $f_{F_3}$};
\draw [fill=black] (6.5,3)circle (1.5pt);

%\draw (5,2)[anchor=south] node {\normalsize $g_0$};
%\draw [fill=white] (5,2)circle (1.5pt);
%\draw (5.5,2)[anchor=south] node {\normalsize $g_2$};
%\draw [fill=white] (5.5,2)circle (1.5pt);
%\draw (5.25,2.25)[anchor=south] node {\normalsize $g_3$};
%\draw [fill=white] (5.25,2.25)circle (1.5pt);
%\draw (5.25,1.75)[anchor=north] node {\normalsize $g_1$};
%\draw [fill=white] (5.25,1.75)circle (1.5pt);

\draw [fill=white] (-1.2,2)circle (1.5pt);

\end{scriptsize}
\end{tikzpicture}

}
\end{figure}
\end{frame}

\begin{frame}{Adição de aresta - \MSFaddEdge($G$, $h$, $v$, $a$, $y$, $f$, $7$)}
\begin{figure}[htb]
\scalebox{1}{
\begin{tikzpicture}[line cap=round,line join=round,x=1cm,y=1cm]
\clip(-3,-.3) rectangle (10,5.6);

\begin{scriptsize}

%
% Node v
%
\draw (0,0) node {$\hat v$};
\draw [line width=1pt,color=cyan] (0,.5) to  (.35,.35); 
\draw [line width=1pt,color=cyan] (.35,.35) to  (.5,0); 

\draw [line width=1pt,color=cyan] (0,.5) to  (-1,1.8); 
\draw [line width=1pt,color=cyan] (.35,.35) to  (1.05,1.75); 
\draw [line width=1pt,color=cyan] (.5,0) to  (2.5,1); 


\draw (0,.5)[anchor=east] node {\tiny $a_v$};
\draw (.35,.35)[anchor=west] node {\tiny $b_v$};
\draw (.5,0)[anchor=north west] node {\tiny $d_v$};
\draw [fill=black] (0,.5) circle (1.5pt);
\draw [fill=black] (.35,.35) circle (1.5pt);
\draw [fill=black] (.5,0) circle (1.5pt);

\draw [line width=1pt,color=cyan] (-1,1.8) to  (-1,2.2); 
\draw (-.6,1.9) node[anchor=north east] {\tiny $a_0$};
\draw [fill=white] (-1,1.8) circle (1.5pt);

% ^b
\draw [line width=1pt,color=cyan] (2.5,1) to (2.9,1.15); 
\draw (1,1.65)[anchor=west] node {\tiny $b_2$};
\draw [fill=white] (1.05,1.75) circle (1.5pt);
\draw (2.5,1)[anchor=north] node {\tiny $d_0$};
\draw [fill=white] (2.5,1) circle (1.5pt);

%
% Node u
%
\draw (0,4) node {$\hat u$};

\draw [line width=1pt,color=cyan] (0,3.5) to  (.3,3.7); 
\draw [line width=1pt,color=cyan] (.5,4) to  (.3,3.7); 

\draw [line width=1pt,color=cyan] (0,3.5) to  (-1,2.2); 
\draw [line width=1pt,color=cyan] (.3,3.7) to  (1.05,2.25); 
\draw [line width=1pt,color=cyan] (.5,4) to  (2.5,3); 

\draw (-1.1,2.35) node {\tiny $a_2$};
\draw [fill=white] (-1,2.2) circle (1.5pt);
\draw (1.2,2.33) node {\tiny $b_0$};
\draw [fill=white] (1.05,2.25) circle (1.5pt);

\draw (2.4,2.9) node {\tiny $c_2$};
\draw [fill=white] (2.5,3) circle (1.5pt);

% u node path
\draw (-.2,3.5) node {\tiny $a_u$};
\draw [fill=black] (0,3.5) circle (1.5pt);
\draw (.5,4.15) node {\tiny $c_u$};
\draw [fill=black] (.5,4) circle (1.5pt);
\draw (.19,3.82) node {\tiny $b_u$};
\draw [fill=black] (.3,3.7) circle (1.5pt);

%
% Node z
%
\draw (3.95,2.1) node {$\hat z$};
\draw [line width=1pt,color=cyan] (4.3,2) to (6.5,2); 
\draw [line width=1pt,color=cyan] (3.7,2.3) to (3,2.7); 
\draw [line width=1pt,color=cyan] (3.7,1.7) to (2.9,1.15); 

% node path
\draw [line width=1pt,color=cyan] (3.7,2.3) to  (3.7,1.7); 
\draw [line width=1pt,color=cyan] (3.7,1.7) to  (4.3,2); 


\draw (3.75,1.55) node {\tiny $d_z$};
\draw [fill=black] (3.7,1.7)circle (1.5pt);
\draw (3.7,2.43) node {\tiny $c_z$};
\draw [fill=black] (3.7,2.3)circle (1.5pt);
\draw (4.3,2.2) node {\tiny $g_z$};
\draw [fill=black] (4.3,2)circle (1.5pt);

\draw (3.2,2.7) node {\tiny $c_0$};
\draw [fill=white] (3,2.7)circle (1.5pt);
\draw (3.05,1.25)[anchor=north] node {\tiny $d_2$};
\draw [fill=white] (2.9,1.15)circle (1.5pt);

%eF0 -- e3
\draw [line width=1pt,color=red] (3.3,4.7) to  (5.25,2.25);


%
% Node F_0
%
\draw (3,5.2) node {$\hat F_0$};
\draw [line width=1pt,color=red] (2.7,4.7) to  (3.8,4.7); 
\draw [line width=1pt,color=red] (3.8,4.7) to  (4,5); 
\draw [line width=1pt,color=red] (4,5) to  (2.5,5); 

\draw [line width=1pt,color=red] (2.5,5) to[in=180,out=180,looseness=2]  (-1.2,2); 
\draw [line width=1pt,color=red] (4,5) to[in=-20,out=10,looseness=3]  (5.25,1.75); 
\draw [line width=1pt,color=red] (3.5,5) to[in=-30,out=20,looseness=4.5]  (2.8,.85); 

%^c
\draw [line width=1pt,color=red] (2.6,2.6) to  (2.9,3.1);
\draw [line width=1pt,color=red] (2.7,4.7) to  (2.9,3.1);

\draw (-1.3,1.9) node {\tiny $a_3$};
\draw [fill=white] (-1.2,2)circle (1.5pt);
\draw (3.07,3.1) node {\tiny $c_1$};
\draw [fill=white] (2.9,3.1)circle (1.5pt);
\draw (2.9,.7) node {\tiny $d_1$};
\draw [fill=white] (2.8,.85)circle (1.5pt);


%fF3 -- 
\draw [line width=1pt,color=red] (6.5,3.9) to  (6.5,3);

% ^f
\draw [line width=1pt,color=cyan] (6.8,2.5) to[in=-60,out=60,looseness=1.5] (6.65,3.8);
\draw [line width=1pt,color=cyan] (6.2,2.5) to[in=-120,out=120,looseness=1.5] (6.35,3.8);

\draw [line width=1pt,color=red] (3.8,4.7) to  (6.5,3.95);
\draw (6.65,4.05) node {\tiny $f_0$};
\draw [fill=white] (6.5,3.95)circle (1.5pt);
\draw (6.65,3.55) node {\tiny $f_2$};
\draw [fill=white] (6.5,3.65)circle (1.5pt);
\draw (6.8,3.8) node {\tiny $f_3$};
\draw [fill=white] (6.65,3.8)circle (1.5pt);
\draw (6.2,3.8) node {\tiny $f_1$};
\draw [fill=white] (6.35,3.8)circle (1.5pt);


\draw (2.5,5)[anchor=south] node {\tiny $a_{F_0}$};
\draw [fill=black] (2.5,5)circle (1.5pt);
\draw (2.5,4.6) node {\tiny $c_{F_0}$};
\draw [fill=black] (2.7,4.7)circle (1.5pt);
\draw (3.5,5)[anchor=south] node {\tiny $d_{F_0}$};
\draw [fill=black] (3.5,5)circle (1.5pt);
\draw (4,5)[anchor=north west] node {\tiny $g_{F_0}$};
\draw [fill=black] (4,5)circle (1.5pt);
\draw (3.8,4.7)[anchor=north] node {\tiny $f_{F_0}$};
\draw [fill=black] (3.8,4.7)circle (1.5pt);
\draw (3.3,4.7)[anchor=north] node {\tiny $g_{F_0}$};
\draw [fill=black] (3.3,4.7)circle (1.5pt);

%
% Node y
%
\draw (6.5,2.35) node {$\hat y$};

%node path
\draw [line width=1pt,color=cyan] (6.5,2) to  (6.2,2.5); 
\draw [line width=1pt,color=cyan] (6.5,2) to  (6.8,2.5); 


\draw (6.5,1.8) node {\tiny $g_y$};
\draw (6,2.5) node {\tiny $f_y$};
\draw (7,2.5) node {\tiny $f_y$};
\draw [fill=black] (6.5,2)circle (1.5pt);
\draw [fill=black] (6.2,2.5)circle (1.5pt);
\draw [fill=black] (6.8,2.5)circle (1.5pt);


%
% Node F_1
%
\draw (2.1,2) node {$\hat F_1$};
%node path
\draw [line width=1pt,color=red] (2.3,2.3) to  (2.3,1.7); 
\draw [line width=1pt,color=red] (1.7,2) to  (2.3,1.7); 

\draw [line width=1pt,color=red] (2.6,2.6) to  (2.3,2.3); 
\draw [line width=1pt,color=red] (2.6,1.3) to  (2.3,1.7); 
\draw [line width=1pt,color=red] (1.7,2) to  (1.3,2); 

% F_1 to F_2
\draw [line width=1pt,color=red] (.8,2) to  (1.3,2); 

\draw (1.8,2.13) node {\tiny $b_{F_1}$};
\draw [fill=black] (1.7,2)circle (1.5pt);
\draw (2.3,1.7)[anchor=north] node {\tiny $d_{F_1}$};
\draw [fill=black] (2.3,1.7)circle (1.5pt);
\draw (2.25,2.2)[anchor=west] node {\tiny $c_{F_1}$};
\draw [fill=black] (2.3,2.3)circle (1.5pt);
\draw (1.4,1.85) node {\tiny $b_3$};
\draw [fill=white] (1.3,2)circle (1.5pt);

\draw (2.8,1.4) node {\tiny $d_3$};
\draw [fill=white] (2.6,1.3)circle (1.5pt);
\draw (2.75,2.55) node {\tiny $c_3$};
\draw [fill=white] (2.6,2.6)circle (1.5pt);

%
% Node F_2
%
\draw (0,2.2) node {$\hat F_2$};

\draw [line width=1pt,color=red] (-.8,2) to  (-.3,2); 
\draw [line width=1pt,color=red] (.8,2) to  (.3,2);   
\draw [line width=1pt,color=red] (-.3,2) to (.3,2);   

\draw [fill=black] (-.2,2) circle (1.5pt);
\draw (-.2,1.8) node {\tiny $a_{F_2}$};
\draw [fill=black] (.2,2) circle (1.5pt);
\draw (.3,1.8) node {\tiny $b_{F_2}$};
\draw (-.75,2.1) node {\tiny $a_1$};
\draw [fill=white] (-.8,2) circle (1.5pt);
\draw (.8,2.16) node {\tiny $b_1$};
\draw [fill=white] (.8,2) circle (1.5pt);

%
% Node F_3
%
\draw (6.7,3.2) node {$\hat F_3$};
\draw (6.3,3) node {\tiny $f_{F_3}$};
\draw [fill=black] (6.5,3)circle (1.5pt);

\draw (5,2)[anchor=south] node {\tiny $g_0$};
\draw [fill=white] (5,2)circle (1.5pt);
\draw (5.5,2)[anchor=south] node {\tiny $g_2$};
\draw [fill=white] (5.5,2)circle (1.5pt);
\draw (5.3,2.25)[anchor=south] node {\tiny $g_3$};
\draw [fill=white] (5.25,2.25)circle (1.5pt);
\draw (5.25,1.75)[anchor=north] node {\tiny $g_1$};
\draw [fill=white] (5.25,1.75)circle (1.5pt);

\end{scriptsize}
\end{tikzpicture}

}
\end{figure}
\end{frame}

\begin{frame}{Adição de aresta - \MSFaddEdge($G$, $h$, $v$, $a$, $y$, $f$, $7$)}
\begin{figure}[htb]
\scalebox{.9}{
\begin{tikzpicture}[line cap=round,line join=round,x=1cm,y=1cm]
\clip(-3,-1) rectangle (8.7,6);

\begin{scriptsize}

%
% Node v
%
\draw (0,0) node {$\hat v$};
\draw [line width=1pt,color=cyan] (0,.5) to  (.35,.35); 
\draw [line width=1pt,color=cyan] (.35,.35) to  (.5,0); 
\draw [line width=1pt,color=cyan] (0,.5) to  (.5,-.5); 

\draw [line width=1pt,color=cyan] (0,.5) to  (-1,1.8); 
\draw [line width=1pt,color=cyan] (.35,.35) to  (1.05,1.75); 
\draw [line width=1pt,color=cyan] (.5,0) to  (2.5,1); 
\draw [line width=1pt,color=cyan] (.5,-.5) to  (2,-.5); 


\draw (0,.5)[anchor=east] node {\tiny $a_v$};
\draw (.35,.35)[anchor=west] node {\tiny $b_v$};
\draw (.5,0)[anchor=north west] node {\tiny $d_v$};
\draw (.5,-.5)[anchor=east] node {\tiny $h_v$};
\draw [fill=black] (0,.5) circle (1.5pt);
\draw [fill=black] (.35,.35) circle (1.5pt);
\draw [fill=black] (.5,0) circle (1.5pt);
\draw [fill=black] (.5,-.5) circle (1.5pt);

\draw [line width=1pt,color=cyan] (-1,1.8) to  (-1,2.2); 
\draw (-.6,1.9) node[anchor=north east] {\tiny $a_0$};
\draw [fill=white] (-1,1.8) circle (1.5pt);

\draw (2,-.5)[anchor=south] node {\tiny $h_0$};
\draw (2,-.5)[fill=white] circle (1.5pt);
% ^b
\draw [line width=1pt,color=cyan] (2.5,1) to (2.9,1.15); 
\draw (1,1.65)[anchor=west] node {\tiny $b_2$};
\draw [fill=white] (1.05,1.75) circle (1.5pt);
\draw (2.5,1)[anchor=north] node {\tiny $d_0$};
\draw [fill=white] (2.5,1) circle (1.5pt);

%
% Node u
%
\draw (0,4) node {$\hat u$};

\draw [line width=1pt,color=cyan] (0,3.5) to  (.3,3.7); 
\draw [line width=1pt,color=cyan] (.5,4) to  (.3,3.7); 

\draw [line width=1pt,color=cyan] (0,3.5) to  (-1,2.2); 
\draw [line width=1pt,color=cyan] (.3,3.7) to  (1.05,2.25); 
\draw [line width=1pt,color=cyan] (.5,4) to  (2.5,3); 

\draw (-1.1,2.35) node {\tiny $a_2$};
\draw [fill=white] (-1,2.2) circle (1.5pt);
\draw (1.2,2.33) node {\tiny $b_0$};
\draw [fill=white] (1.05,2.25) circle (1.5pt);

\draw (2.4,2.9) node {\tiny $c_2$};
\draw [fill=white] (2.5,3) circle (1.5pt);

% u node path
\draw (-.2,3.5) node {\tiny $a_u$};
\draw [fill=black] (0,3.5) circle (1.5pt);
\draw (.5,4.15) node {\tiny $c_u$};
\draw [fill=black] (.5,4) circle (1.5pt);
\draw (.19,3.82) node {\tiny $b_u$};
\draw [fill=black] (.3,3.7) circle (1.5pt);

%
% Node z
%
\draw (3.95,2.1) node {$\hat z$};
\draw [line width=1pt,color=cyan] (4.3,2) to (6.5,2); 
\draw [line width=1pt,color=cyan] (3.7,2.3) to (3,2.7); 
\draw [line width=1pt,color=cyan] (3.7,1.7) to (2.9,1.15); 

% node path
\draw [line width=1pt,color=cyan] (3.7,2.3) to  (3.7,1.7); 
\draw [line width=1pt,color=cyan] (3.7,1.7) to  (4.3,2); 


\draw (3.75,1.55) node {\tiny $d_z$};
\draw [fill=black] (3.7,1.7)circle (1.5pt);
\draw (3.7,2.43) node {\tiny $c_z$};
\draw [fill=black] (3.7,2.3)circle (1.5pt);
\draw (4.3,2.2) node {\tiny $g_z$};
\draw [fill=black] (4.3,2)circle (1.5pt);

\draw (3.2,2.7) node {\tiny $c_0$};
\draw [fill=white] (3,2.7)circle (1.5pt);
\draw (3.05,1.25)[anchor=north] node {\tiny $d_2$};
\draw [fill=white] (2.9,1.15)circle (1.5pt);

%eF0 -- e3
\draw [line width=1pt,color=red] (3.3,4.7) to  (5.25,2.25);


%
% Node F_0
%
\draw (3,5) node {$\hat F_0$};
\draw [line width=1pt,color=red] (2.7,4.7) to  (3.8,4.7); 
\draw [line width=1pt,color=red] (3.8,4.7) to  (4.3,4.8); 
\draw [line width=1pt,color=red] (3.5,5.4) to  (3,5.3); 
\draw [line width=1pt,color=red] (3,5.3) to  (2.5,5); 
\draw [line width=1pt,color=red] (2.5,5) to  (2.7,4.7); 
\draw [line width=1pt,color=red] (3.5,5.4) to  (4,5.2); 

\draw [line width=1pt,color=red] (2.5,5) to[in=100,out=180,looseness=1.5]  (-1.2,2); 
\draw [line width=1pt,color=red] (4,5.2) to[in=-30,out=20,looseness=3.5]  (5.25,1.75); 
\draw [line width=1pt,color=red] (5,5.8) to[in=90,out=180,looseness=1]  (3.5,5.4); 
\draw [line width=1pt,color=red] (5,5.8) to[in=-30,out=0,looseness=3]  (2.8,.85); 
\draw [line width=1pt,color=red] (4.3,4.8) to (4.8,4.8); 

%^c
\draw [line width=1pt,color=red] (2.6,2.6) to  (2.9,3.1);
\draw [line width=1pt,color=red] (2.7,4.7) to  (2.9,3.1);

\draw (-1.3,1.9) node {\tiny $a_3$};
\draw [fill=white] (-1.2,2)circle (1.5pt);
\draw (3.07,3.1) node {\tiny $c_1$};
\draw [fill=white] (2.9,3.1)circle (1.5pt);
\draw (2.9,.7) node {\tiny $d_1$};
\draw [fill=white] (2.8,.85)circle (1.5pt);
\draw (4.8,4.8)[anchor=south] node {\tiny $h_1$};
\draw [fill=white] (4.8,4.8)circle (1.5pt);


%fF3 -- 
\draw [line width=1pt,color=red] (6.5,3.9) to  (6.5,3);

% ^f
\draw [line width=1pt,color=cyan] (6.8,2.5) to[in=-60,out=60,looseness=1.5] (6.65,3.8);
\draw [line width=1pt,color=cyan] (6.2,2.5) to[in=-120,out=120,looseness=1.5] (6.35,3.8);

\draw [line width=1pt,color=red] (3.8,4.7) to  (6.5,3.95);



\draw (6.65,4.05) node {\tiny $f_0$};
\draw [fill=white] (6.5,3.95)circle (1.5pt);
\draw (6.65,3.55) node {\tiny $f_2$};
\draw [fill=white] (6.5,3.65)circle (1.5pt);
\draw (6.8,3.8) node {\tiny $f_3$};
\draw [fill=white] (6.65,3.8)circle (1.5pt);
\draw (6.2,3.8) node {\tiny $f_1$};
\draw [fill=white] (6.35,3.8)circle (1.5pt);



\draw [line width=1pt,color=red] (2.5,5.5) to  (3,5.3);

\draw (3,5.3)[anchor=south] node {\tiny $h_{F_0}$};
\draw [fill=black] (3,5.3)circle (1.5pt);
\draw (2.5,5)[anchor=south] node {\tiny $a_{F_0}$};
\draw [fill=black] (2.5,5)circle (1.5pt);
\draw (2.5,4.6) node {\tiny $c_{F_0}$};
\draw [fill=black] (2.7,4.7)circle (1.5pt);
\draw (3.4,5.4)[anchor=south] node {\tiny $d_{F_0}$};
\draw [fill=black] (3.5,5.4)circle (1.5pt);
\draw (4.3,4.8)[anchor=south] node {\tiny $h_{F_0}$};
\draw [fill=black] (4.3,4.8)circle (1.5pt);
\draw (4,5.2)[anchor=south] node {\tiny $g_{F_0}$};
\draw [fill=black] (4,5.2)circle (1.5pt);
\draw (3.8,4.7)[anchor=north] node {\tiny $f_{F_0}$};
\draw [fill=black] (3.8,4.7)circle (1.5pt);
\draw (3.3,4.7)[anchor=north] node {\tiny $g_{F_0}$};
\draw [fill=black] (3.3,4.7)circle (1.5pt);

\draw (2.5,5.5)[anchor=south] node {\tiny $h_3$};
\draw [fill=white] (2.5,5.5)circle (1.5pt);
%
% Node y
%
\draw (6.7,2.3) node {$\hat y$};

%node path
\draw [line width=1pt,color=cyan] (6.5,2) to  (6.2,2.5); 
\draw [line width=1pt,color=cyan] (6.2,2.5) to  (6.8,2.5); 
\draw [line width=1pt,color=cyan] (6.5,2) to  (7.1,2); 


\draw [line width=1pt,color=cyan] (7.1,2) to  (6.6,1.5); 

\draw (6.5,1.8) node {\tiny $g_y$};
\draw (6,2.5) node {\tiny $f_y$};
\draw (7,2.5) node {\tiny $f_y$};
\draw (7.3,2) node {\tiny $h_y$};
\draw [fill=black] (6.5,2)circle (1.5pt);
\draw [fill=black] (6.2,2.5)circle (1.5pt);
\draw [fill=black] (6.8,2.5)circle (1.5pt);
\draw [fill=black] (7.1,2)circle (1.5pt);



\draw (6.8,1.5) node {\tiny $h_2$};
\draw [fill=white] (6.6,1.5)circle (1.5pt);

%
% Node F_1
%
\draw (2.1,2) node {$\hat F_1$};
%node path
\draw [line width=1pt,color=red] (2.3,2.3) to  (2.3,1.7); 
\draw [line width=1pt,color=red] (1.7,2) to  (2.3,1.7); 

\draw [line width=1pt,color=red] (2.6,2.6) to  (2.3,2.3); 
\draw [line width=1pt,color=red] (2.6,1.3) to  (2.3,1.7); 
\draw [line width=1pt,color=red] (1.7,2) to  (1.3,2); 

% F_1 to F_2
\draw [line width=1pt,color=red] (.8,2) to  (1.3,2); 

\draw (1.8,2.13) node {\tiny $b_{F_1}$};
\draw [fill=black] (1.7,2)circle (1.5pt);
\draw (2.3,1.7)[anchor=north] node {\tiny $d_{F_1}$};
\draw [fill=black] (2.3,1.7)circle (1.5pt);
\draw (2.25,2.2)[anchor=west] node {\tiny $c_{F_1}$};
\draw [fill=black] (2.3,2.3)circle (1.5pt);
\draw (1.4,1.85) node {\tiny $b_3$};
\draw [fill=white] (1.3,2)circle (1.5pt);

\draw (2.8,1.4) node {\tiny $d_3$};
\draw [fill=white] (2.6,1.3)circle (1.5pt);
\draw (2.75,2.55) node {\tiny $c_3$};
\draw [fill=white] (2.6,2.6)circle (1.5pt);

%
% Node F_2
%
\draw (0,2.2) node {$\hat F_2$};

\draw [line width=1pt,color=red] (-.8,2) to  (-.3,2); 
\draw [line width=1pt,color=red] (.8,2) to  (.3,2);   
\draw [line width=1pt,color=red] (-.3,2) to (.3,2);   

\draw [fill=black] (-.2,2) circle (1.5pt);
\draw (-.2,1.8) node {\tiny $a_{F_2}$};
\draw [fill=black] (.2,2) circle (1.5pt);
\draw (.3,1.8) node {\tiny $b_{F_2}$};
\draw (-.75,2.1) node {\tiny $a_1$};
\draw [fill=white] (-.8,2) circle (1.5pt);
\draw (.8,2.16) node {\tiny $b_1$};
\draw [fill=white] (.8,2) circle (1.5pt);

%
% Node F_3
%
\draw (6.7,3.2) node {$\hat F_3$};
\draw (6.3,3) node {\tiny $f_{F_3}$};
\draw [fill=black] (6.5,3)circle (1.5pt);

\draw (5,2)[anchor=south] node {\tiny $g_0$};
\draw [fill=white] (5,2)circle (1.5pt);
\draw (5.5,2)[anchor=south] node {\tiny $g_2$};
\draw [fill=white] (5.5,2)circle (1.5pt);
\draw (5.3,2.25)[anchor=south] node {\tiny $g_3$};
\draw [fill=white] (5.25,2.25)circle (1.5pt);
\draw (5.25,1.75)[anchor=north] node {\tiny $g_1$};
\draw [fill=white] (5.25,1.75)circle (1.5pt);

\end{scriptsize}
\end{tikzpicture}

}
\label{fig:MSF-adiciona-nao-ponte-1}
\end{figure}
\end{frame}


\begin{frame}{Adição de aresta - \MSFaddEdge($G$, $h$, $v$, $a$, $y$, $f$, $7$)}
\begin{figure}
\scalebox{.95}{
\begin{tikzpicture}[line cap=round,line join=round,x=1cm,y=1cm]
\clip(-3,-1) rectangle (8.7,6);

\begin{scriptsize}

%
% Node v
%
%\draw (0,0) node {$\hat v$};
\draw [line width=1pt,color=cyan] (0,.5) to  (.35,.35); 
\draw [line width=1pt,color=cyan] (.35,.35) to  (.5,0); 
\draw [line width=1pt,color=cyan] (0,.5) to  (.5,-.5); 

\draw [line width=1pt,color=cyan] (0,.5) to  (-1,1.8); 
\draw [line width=1pt,color=cyan] (.35,.35) to  (1.05,1.75); 
\draw [line width=1pt,color=cyan] (.5,0) to  (2.5,1); 
\draw [line width=1pt,color=cyan] (.5,-.5) to  (5.2,0); 


%\draw (0,.5)[anchor=east] node {\normalsize $a_v$};
%\draw (.35,.35)[anchor=west] node {\normalsize $b_v$};
%\draw (.5,0)[anchor=north west] node {\normalsize $d_v$};
\draw (.5,-.5)[anchor=east] node {\normalsize $h_v$};
\draw [fill=black] (0,.5) circle (1.5pt);
\draw [fill=black] (.35,.35) circle (1.5pt);
\draw [fill=black] (.5,0) circle (1.5pt);
\draw [fill=black] (.5,-.5) circle (1.5pt);

\draw [line width=1pt,color=cyan] (-1,1.8) to  (-1,2.2); 
%\draw (-.6,1.9) node[anchor=north east] {\normalsize $a_0$};
\draw [fill=white] (-1,1.8) circle (1.5pt);

\draw (5.2,0)[anchor=south] node {\normalsize $h_0$};
\draw (5.2,0)[fill=white] circle (1.5pt);
% ^b
\draw [line width=1pt,color=cyan] (2.5,1) to (2.9,1.15); 
%\draw (1,1.65)[anchor=west] node {\normalsize $b_2$};
\draw [fill=white] (1.05,1.75) circle (1.5pt);
\draw (2.5,1)[anchor=north] node {\normalsize $d_0$};
\draw [fill=white] (2.5,1) circle (1.5pt);

%
% Node u
%
%\draw (0,4) node {$\hat u$};

\draw [line width=1pt,color=cyan] (0,3.5) to  (.3,3.7); 
\draw [line width=1pt,color=cyan] (.5,4) to  (.3,3.7); 

\draw [line width=1pt,color=cyan] (0,3.5) to  (-1,2.2); 
\draw [line width=1pt,color=cyan] (.3,3.7) to  (1.05,2.25); 
\draw [line width=1pt,color=cyan] (.5,4) to  (2.5,3); 

%\draw (-1.1,2.35) node {\normalsize $a_2$};
\draw [fill=white] (-1,2.2) circle (1.5pt);
%\draw (1.2,2.33) node {\normalsize $b_0$};
\draw [fill=white] (1.05,2.25) circle (1.5pt);

%\draw (2.4,2.9) node {\normalsize $c_2$};
\draw [fill=white] (2.5,3) circle (1.5pt);

% u node path
%\draw (-.2,3.5) node {\normalsize $a_u$};
\draw [fill=black] (0,3.5) circle (1.5pt);
%\draw (.5,4.15) node {\normalsize $c_u$};
\draw [fill=black] (.5,4) circle (1.5pt);
%\draw (.19,3.82) node {\normalsize $b_u$};
\draw [fill=black] (.3,3.7) circle (1.5pt);

%
% Node z
%
%\draw (3.95,2.1) node {$\hat z$};
\draw [line width=1pt,color=cyan] (4.3,2) to (6.5,2); 
\draw [line width=1pt,color=cyan] (3.7,2.3) to (3,2.7); 
\draw [line width=1pt,color=cyan] (3.7,1.7) to (2.9,1.15); 

% node path
\draw [line width=1pt,color=cyan] (3.7,2.3) to  (3.7,1.7); 
\draw [line width=1pt,color=cyan] (3.7,1.7) to  (4.3,2); 


%\draw (3.75,1.55) node {\normalsize $d_z$};
\draw [fill=black] (3.7,1.7)circle (1.5pt);
%\draw (3.7,2.43) node {\normalsize $c_z$};
\draw [fill=black] (3.7,2.3)circle (1.5pt);
%\draw (4.3,2.2) node {\normalsize $g_z$};
\draw [fill=black] (4.3,2)circle (1.5pt);

%\draw (3.2,2.7) node {\normalsize $c_0$};
\draw [fill=white] (3,2.7)circle (1.5pt);
\draw (3.05,1.25)[anchor=north] node {\normalsize $d_2$};
\draw [fill=white] (2.9,1.15)circle (1.5pt);

%eF0 -- e3
\draw [line width=1pt,color=red] (3.3,4.7) to  (5.25,2.25);


%
% Node F_0
%
%\draw (3,5) node {$\hat F_0$};
\draw [line width=1pt,color=red] (2.7,4.7) to  (3.8,4.7); 
\draw [line width=1pt,color=red] (3.8,4.7) to  (4.3,4.8); 
\draw [line width=1pt,color=red] (2.5,5) to  (2.7,4.7); 

\draw [line width=1pt,color=red] (2.5,5) to[in=100,out=180,looseness=1.5]  (-1.2,2); 
\draw [line width=1pt,color=red] (4.3,4.8) to[in=0,out=20,looseness=2] (5.7,-.1); 

%^c
\draw [line width=1pt,color=red] (2.6,2.6) to  (2.9,3.1);
\draw [line width=1pt,color=red] (2.7,4.7) to  (2.9,3.1);

%\draw (-1.3,1.9) node {\normalsize $a_3$};
\draw [fill=white] (-1.2,2)circle (1.5pt);
%\draw (3.07,3.1) node {\normalsize $c_1$};
\draw [fill=white] (2.9,3.1)circle (1.5pt);
\draw (5.7,-.1)[anchor=north] node {\normalsize $h_1$};
\draw [fill=white] (5.7,-.1)circle (1.5pt);


%fF3 -- 
\draw [line width=1pt,color=red] (6.5,3.9) to  (6.5,3);

% ^f
\draw [line width=1pt,color=cyan] (6.8,2.5) to[in=-60,out=60,looseness=1.5] (6.65,3.8);
\draw [line width=1pt,color=cyan] (6.2,2.5) to[in=-120,out=120,looseness=1.5] (6.35,3.8);

\draw [line width=1pt,color=red] (3.8,4.7) to  (6.5,3.95);



%\draw (6.65,4.05) node {\normalsize $f_0$};
\draw [fill=white] (6.5,3.95)circle (1.5pt);
%\draw (6.65,3.55) node {\normalsize $f_2$};
\draw [fill=white] (6.5,3.65)circle (1.5pt);
%\draw (6.8,3.8) node {\normalsize $f_3$};
\draw [fill=white] (6.65,3.8)circle (1.5pt);
%\draw (6.2,3.8) node {\normalsize $f_1$};
\draw [fill=white] (6.35,3.8)circle (1.5pt);




%\draw (2.5,5)[anchor=south] node {\normalsize $a_{F_0}$};
\draw [fill=black] (2.5,5)circle (1.5pt);
%\draw (2.5,4.6) node {\normalsize $c_{F_0}$};
\draw [fill=black] (2.7,4.7)circle (1.5pt);
\draw (4.3,4.8)[anchor=south] node {\normalsize $h_{F_0}$};
\draw [fill=black] (4.3,4.8)circle (1.5pt);
\draw (3.3,4.7)[anchor=south] node {\normalsize $g_{F_0}$};
\draw [fill=black] (3.3,4.7)circle (1.5pt);
\draw (3.8,4.7)[anchor=north] node {\normalsize $f_{F_0}$};
\draw [fill=black] (3.8,4.7)circle (1.5pt);



%
% F4
%
\draw [line width=1pt,color=red] (4.5,1) to  (3.7,.5);
\draw [line width=1pt,color=red] (3.7,.5) to  (4.5,.5);

\draw [line width=1pt,color=red] (2.8,.85) to  (3.7,.5); % d
\draw [line width=1pt,color=red] (5.5,.3) to  (4.5,.5); % h
\draw [line width=1pt,color=red] (5.25,1.75) to  (4.5,1); % g

%\draw (4,.2) node {$\hat F_4$};
\draw (3.6,.5)[anchor=south] node {\normalsize $d_{F_0}$};
\draw [fill=black] (3.7,.5)circle (1.5pt);
\draw (4.5,.5)[anchor=south] node {\normalsize $h_{F_0}$};
\draw [fill=black] (4.5,.5)circle (1.5pt);
\draw (4.5,1)[anchor=south east] node {\normalsize $g_{F_0}$};
\draw [fill=black] (4.5,1)circle (1.5pt);


\draw (2.9,.7) node {\normalsize $d_1$};
\draw [fill=white] (2.8,.85)circle (1.5pt);
\draw (5.25,1.75)[anchor=north] node {\normalsize $g_1$};
\draw [fill=white] (5.25,1.75)circle (1.5pt);
\draw (5.5,.3)[anchor=south] node {\normalsize $h_3$};
\draw [fill=white] (5.5,.3)circle (1.5pt);

%
% Node y
%
%\draw (6.7,2.3) node {$\hat y$};

%node path
\draw [line width=1pt,color=cyan] (6.5,2) to  (6.2,2.5); 
\draw [line width=1pt,color=cyan] (6.2,2.5) to  (6.8,2.5); 
\draw [line width=1pt,color=cyan] (6.5,2) to  (7.1,2); 


\draw [line width=1pt,color=cyan] (7.1,2) to  (5.9,.2); 

%\draw (6.5,1.8) node {\normalsize $g_y$};
%\draw (6,2.5) node {\normalsize $f_y$};
%\draw (7,2.5) node {\normalsize $f_y$};
\draw (7.3,2) node {\normalsize $h_y$};
\draw [fill=black] (6.5,2)circle (1.5pt);
\draw [fill=black] (6.2,2.5)circle (1.5pt);
\draw [fill=black] (6.8,2.5)circle (1.5pt);
\draw [fill=black] (7.1,2)circle (1.5pt);

\draw (5.85,.4) node {\normalsize $h_2$};
\draw [fill=white] (5.9,.2)circle (1.5pt);

%
% Node F_1
%
%\draw (2.1,2) node {$\hat F_1$};
%node path
\draw [line width=1pt,color=red] (2.3,2.3) to  (2.3,1.7); 
\draw [line width=1pt,color=red] (1.7,2) to  (2.3,1.7); 

\draw [line width=1pt,color=red] (2.6,2.6) to  (2.3,2.3); 
\draw [line width=1pt,color=red] (2.6,1.3) to  (2.3,1.7); 
\draw [line width=1pt,color=red] (1.7,2) to  (1.3,2); 

% F_1 to F_2
\draw [line width=1pt,color=red] (.8,2) to  (1.3,2); 

%\draw (1.8,2.13) node {\normalsize $b_{F_1}$};
\draw [fill=black] (1.7,2)circle (1.5pt);
%\draw (2.3,1.7)[anchor=north] node {\normalsize $d_{F_1}$};
\draw [fill=black] (2.3,1.7)circle (1.5pt);
%\draw (2.25,2.2)[anchor=west] node {\normalsize $c_{F_1}$};
\draw [fill=black] (2.3,2.3)circle (1.5pt);
%\draw (1.4,1.85) node {\normalsize $b_3$};
\draw [fill=white] (1.3,2)circle (1.5pt);

\draw (2.8,1.4) node {\normalsize $d_3$};
\draw [fill=white] (2.6,1.3)circle (1.5pt);
%\draw (2.75,2.55) node {\normalsize $c_3$};
\draw [fill=white] (2.6,2.6)circle (1.5pt);

%
% Node F_2
%
%\draw (0,2.2) node {$\hat F_2$};

\draw [line width=1pt,color=red] (-.8,2) to  (-.3,2); 
\draw [line width=1pt,color=red] (.8,2) to  (.3,2);   
\draw [line width=1pt,color=red] (-.3,2) to (.3,2);   

\draw [fill=black] (-.2,2) circle (1.5pt);
%\draw (-.2,1.8) node {\normalsize $a_{F_2}$};
\draw [fill=black] (.2,2) circle (1.5pt);
%\draw (.3,1.8) node {\normalsize $b_{F_2}$};
%\draw (-.75,2.1) node {\normalsize $a_1$};
\draw [fill=white] (-.8,2) circle (1.5pt);
%\draw (.8,2.16) node {\normalsize $b_1$};
\draw [fill=white] (.8,2) circle (1.5pt);

%
% Node F_3
%
%\draw (6.7,3.2) node {$\hat F_3$};
%\draw (6.3,3) node {\normalsize $f_{F_3}$};
\draw [fill=black] (6.5,3)circle (1.5pt);

\draw (5,2)[anchor=south] node {\normalsize $g_0$};
\draw [fill=white] (5,2)circle (1.5pt);
\draw (5.5,2)[anchor=south] node {\normalsize $g_2$};
\draw [fill=white] (5.5,2)circle (1.5pt);
\draw (5.3,2.25)[anchor=south] node {\normalsize $g_3$};
\draw [fill=white] (5.25,2.25)circle (1.5pt);

\end{scriptsize}
\end{tikzpicture}

}
\end{figure}
\end{frame}






\section[$\Omega(\lg n)$]{O limitante inferior de~$\Omega(\lg n)$}

\begin{frame}{O limitante inferior de~$\Omega(\lg n)$}
Mihai Patrascu e Erik D. Demaine provaram o seguinte limitante inferior:
\begin{teorema}
Seja $t_m$ o consumo de tempo de \dymGraphAddEdge{} ou \dymGraphDelEdge{} e~$t_c$ o consumo de tempo de \dymGraphQuery{}, então
$$
\min\{ t_m, t_c\}\lg \left( \frac{\max\{ t_m, t_c\}}{\min\{ t_m,t_c\}}\right) = \Omega(\lg n).
$$
Esse limitante é válido mesmo para implementações aleatorizadas e/ou amortizadas de \dymGraphAddEdge{}, \dymGraphDelEdge{} e \dymGraphQuery{} e mesmo restringindo a classe de grafos do problema para caminhos.
\end{teorema}
\end{frame}

\section{Bibliografia}
\nocite{*}
\begin{frame}[allowframebreaks]
\frametitle{Bibliografia}
\bibliographystyle{plain}
    \bibliography{bib.bib}
\end{frame}

\end{document}
